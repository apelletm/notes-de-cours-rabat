\documentclass[11pt]{exam}
%%%%%%%%%%%%%%%%%%%%%%%%%%%%%%%%
%\noprintanswers % pour enlever les réponses
%\printanswers

\unframedsolutions
\SolutionEmphasis{\itshape\small}
\renewcommand{\solutiontitle}{\noindent\textbf{A: }}
%%%%%%%%%%%%%%%%%%%%%%%%%%%%%%%%

\usepackage[utf8]{inputenc}
\usepackage[english]{babel}


\usepackage[margin=0.73in]{geometry}
%\usepackage[top=1in, bottom=1in, left=1in, right=1in]{geometry}

%\usepackage{fullpage}


\usepackage{hyperref}
\usepackage{appendix}
\usepackage{enumerate}


\usepackage{times,graphicx,epsfig,amsmath,latexsym,amssymb,verbatim}%,revsymb}
\usepackage{algorithmicx, enumitem, algpseudocode, algorithm, caption}


%%%%%%%%%%%%%%%%%%%%%
% Handling comments and versions %%%
%%%%%%%%%%%%%%%%%%%%%
\newcommand{\extra}[1]{}

\renewcommand{\comment}[1]{\texttt{[#1]}}


%%%%%%%%%%%%%%%%%%%%%%%%%%%
%% THEOREMS
%%%%%%%%%%%%%%%%%%%%%%%%%%%

\usepackage{amsmath,amssymb,amsfonts}
\usepackage{amsthm}

% Landau 
\newcommand{\bigO}{\mathcal{O}}
\newcommand*{\OLandau}{\bigO}
\newcommand*{\WLandau}{\Omega}
\newcommand*{\xOLandau}{\widetilde{\OLandau}}
\newcommand*{\xWLandau}{\widetilde{\WLandau}}
\newcommand*{\TLandau}{\Theta}
\newcommand*{\xTLandau}{\widetilde{\TLandau}}
\newcommand{\smallo}{o} %technically, an omicron
\newcommand{\softO}{\widetilde{\bigO}}
\newcommand{\wLandau}{\omega}
\newcommand{\negl}{\mathrm{negl}} 


\newtheorem{theorem}{Theorem}
\theoremstyle{definition}
\newtheorem{definition}[theorem]{Definition}


\newcommand{\nc}{\newcommand}
\nc{\eps}{\varepsilon}
\nc{\RR}{{{\mathbb R}}}
\nc{\CC}{{{\mathbb C}}}
\nc{\FF}{{{\mathbb F}}}
\nc{\NN}{{{\mathbb N}}}
\nc{\ZZ}{{{\mathbb Z}}}
\nc{\PP}{{{\mathbb P}}}
\nc{\QQ}{{{\mathbb Q}}}
\nc{\UU}{{{\mathbb U}}}
\nc{\OO}{{{\mathbb O}}}
\nc{\EE}{{{\mathbb E}}}

\let\vec\mathbf

\pretolerance=1000

%%%%%%%%%%%%%%%%%%%%%%%%%%%%%%%%
%%%%%%%%%%%%%%%%%%%%%%%%%%%%%%%%
%% DOCUMENT STARTS
%%%%%%%%%%%%%%%%%%%%%%%%%%%%%%%%
%%%%%%%%%%%%%%%%%%%%%%%%%%%%%%%%
\usepackage{tikz}
\usetikzlibrary{automata}



% Custom colors
\usepackage{color}
\definecolor{deepblue}{rgb}{0,0,0.5}
\definecolor{deepred}{rgb}{0.6,0,0}
\definecolor{deepgreen}{rgb}{0,0.5,0}

\usepackage{pythonhighlight}

\begin{document}
	{\noindent
		\textsc{CIMPA School}
		\hfill {October 23 -- November 4, 2023\\}
	\hrule
	\begin{center}
		{\Large\textbf{
				\textsc{Lab 2: Breaking Merkle-Hellman Cryptosystem} \\[5pt]
		} } 
	\end{center}
	}
	\hrule \vspace{5mm}
	
	\thispagestyle{empty}
	
	\vspace{0.2cm}
	

An encryption scheme proposed proposed by Merkle-Hellman in 1978~\cite{MH78} was shortly after broken by Adi Shamir~\cite{Shamir}.
Your task will be to implement the attack and break concrete instances. Let us start with a description of the encryption scheme.

\begin{definition}[Superincreasing sequence]
	Superincreasing sequence is a tuple $\vec{r} = (r_1, \ldots, r_n)$ of positive numbers such that
	\[
	r_i > 2r_{i-1}, \quad 2\leq i < n.
	\] 
\end{definition}

From the definition it follows that $r_k > r_{k-1} + \ldots r_1$ for all $2 \leq k \leq n$. For a superincreasing sequence, the following knapsack problem has an efficient solution: given $S \in \ZZ$ such that there exists $\vec b \in \{0,1\}^n$ with the property
\[
	S = \sum_{i=1}^n b_i r_i,
\]
find $\vec b$. Such $\vec b$ can be found via the following algorithm
\begin{algorithm}[H] \caption{Superincreasing knapsack}
	\begin{tabular}{l l}
		\textbf{Input:} 
		& $\vec{r} = (r_1, \ldots, r_n), S \in \ZZ$ \\
		\textbf{Output:} 
		& $\vec b$ s.t.\ $S = \sum_{i=1}^n b_i r_i$ \\
\end{tabular}
	\begin{algorithmic}[1]
		\State $\vec b = \vec 0$
		\For{$i$ from $n$ to $1$}
			\If{$S>r_i$}
				\State $b_i = 1$
				\State $S = S-r_i$
			\EndIf
		\EndFor \\
		\Return $\vec b$
	\end{algorithmic}
\label{alg:knapsack_superenc}
\caption{Solving superincreasing knapsack}
\end{algorithm}

Let $n$ be a public parameter. Key generation function $\mathsf{KeyGen}$, encryption function  $\mathsf{Enc}$, and decryption function $ \mathsf{Dec}$ are defined as follows.

\medskip

$\mathsf{KeyGen}(n)$.
\begin{enumerate}
	\item Generate a random superincreasing sequence $\vec r$ (the precise method of `random' is irrelevant for this discussion).
	\item Choose $A, q$ such that $q>r_n$ and $\gcd(A,q) = 1$.
	\item Compute the sequence $M_i = Ar_i~\bmod q$
	\item Set $\mathrm{pk} = M, \mathrm{sk} = (A^{-1}\bmod~q, q, \vec r)$.
\end{enumerate}

\medskip

$\mathsf{Enc}(\mathrm{pk}, \vec m \in \{ 0,1\}^n)$.
\begin{enumerate}
	\item Return ciphertext $c = \sum_{i=1}^n M_i m_i \in \ZZ$
\end{enumerate}

\medskip
$\mathsf{Dec}(\mathrm{sk}, c)$.
\begin{enumerate}
	\item Compute $c' = c A^{-1}\bmod~q$
	\item Using Algorithm~\ref{alg:knapsack_superenc} with input $c', \vec r$ find $\vec m' \in \{0,1\}^n$.
\end{enumerate}

You can convince yourself in the correctness of the scheme. There is no need to convince yourself in its security since now you'll see an efficient attack on it.

\medskip

\paragraph{Shamir's attack} recovers message $\vec m$ from the knowledge of the corresponding ciphertext and the public key. Consider the lattice generated by the \emph{rows} of the following $(n+1) \times (n+1)$ matrix
\[
	B = 
	\begin{pmatrix}
		2 & 0 & 0 & \ldots & 0 & M_1 \\
		0 & 2 & 0 & \ldots & 0 & M_2 \\
		0 & 0 & 2 & \ldots & 0 & M_3 \\
		\vdots & \vdots & \vdots & \ddots & \vdots & \vdots \\
		-1 & -1 & -1 & \ldots & -1 & -c 
	\end{pmatrix}
\]

Note that if $\vec m$ is the message for the ciphertext $c$, i.e., $c =\sum_{i=1}^n M_i m_i $, then $L(B)$ contains $\vec t = (2m_1 -1, 2m_2-1, 2m_3 - 1, \ldots, 0) \in \ZZ^{n+1}$. For a binary $\vec m$, this vector is the shortest of $L(B)$ with high probability and, for many interesting parameters, can be efficiently found via LLL.

\medskip 
\paragraph{Task}

\begin{enumerate}
	\item Download script $\mathsf{merkle-hellman.sage}$. There you'll find implementations of $\mathsf{KeyGen}$,   $\mathsf{Enc}$, $ \mathsf{Dec}$. You do not need to modify them.
	\item Implement Shamir's attack in the function $\mathsf{attack()}$.
	\item Check the correctness of your implementation by running
	\[
		\textsf{sage -t merkle\_hellman.sage}
	\]
	All three tests should pass.
\end{enumerate}


\begin{thebibliography}{9}
	\bibitem{MH78} Ralph Merkle and Martin Hellman,
	\textit{Hiding Information and Signatures in Trapdoor Knapsacks}. IEEE Trans. Information Theory. 1978
	
	\bibitem{Shamir} Adi Shamir, 
	\textit{A Polynomial Time Algorithm for Breaking the Basic Merkle-Hellman Cryptosystem}. CRYPTO. 1982
	
\end{thebibliography}


\end{document}