% ==================================================================
% Definitions for this paper
% ==================================================================
\mathchardef\hyphen="2D

\usepackage{multirow}
\usepackage{multicol} % For multiple coloumn environments
%\usepackage{stmaryrd} % For set brackets
% \setlength{\columnsep}{15pt} % Defining the coloumn seperation
% \setlength{\columnseprule}{1pt} % Place a line between coloumns
% \newcommand{\tab}{\hspace*{2em}}

%subscripts

\newcommand*\SmallTextScript[2]{{\mathchoice{\displaystyle #2}
		{\textstyle #2}%dito
		{\scalebox{#1}{\ensuremath{\scriptstyle #2}}}%
		{\scalebox{#1}{\ensuremath{\scriptscriptstyle #2}}}%
}}


% ADVERSARIES AND SUCH
\newcommand*{\poly}{\ensuremath{\mathrm{poly}}}
\newcommand*{\eps}{\ensuremath{\varepsilon}}

% GROUPS/DISTRIBUTIONS/SETS/LISTS
\newcommand{\N}{{{\mathbb N}}}
\newcommand{\Z}{{{\mathbb Z}}}
\newcommand*{\IZ}{\ensuremath{\mathbb{Z}}}
\newcommand*{\IN}{\ensuremath{\mathbb{N}}}
\newcommand*{\IQ}{\ensuremath{\mathbb{Q}}}
\newcommand{\R}{{{\mathbb R}}}
\newcommand*{\IR}{{{\mathbb R}}}
\newcommand{\Zp}{\ints_p} % Integers modulo p
\newcommand{\Zq}{\ints_q} % Integers modulo q
\newcommand{\Zn}{\ints_N} % Integers modulo N
\newcommand{\F}{\ensuremath{\mathbb{F}}}

\newcommand{\GF}{\ensuremath{\mathbb{F}_2}}
\newcommand{\GFn}{\ensuremath{\mathbb{F}^n_2}}

%%% ALGORITHMS/PROCEDURES %%%
\newcommand{\bigO}{\mathcal{O}}
\newcommand*{\OLandau}{\bigO}
\newcommand*{\WLandau}{\Omega}
\newcommand*{\xOLandau}{\widetilde{\OLandau}}
\newcommand*{\xWLandau}{\widetilde{\WLandau}}
\newcommand*{\TLandau}{\Theta}
\newcommand*{\xTLandau}{\widetilde{\TLandau}}
\newcommand{\smallo}{o} %technically, an omicron
\newcommand{\wLandau}{\omega}
\newcommand{\negl}{\mathrm{negl}}
\newcommand*\PROB\Pr 
\DeclareMathOperator*{\EXPECT}{\mathbb{E}}
\DeclareMathOperator*{\VARIANCE}{\mathbb{V}}
\DeclareMathOperator*{\LOGBIAS}{\mathbb{LB}}

% Lattices

% \newcommand{\coset}{\Lambda} % Lambda Lattice
% \newcommand{\cosetPerp}{\Lambda^{\bot}} % Lambda_Perp Lattice
% \newcommand{\gadget}{\textbf{G}} %Gaget matrix
% \newcommand{\mes}{\textbf{m}} %message vector
% \newcommand{\AMat}{\textbf{A}} %A matrices
% \newcommand{\BMat}{\textbf{B}} %B matrices
% \newcommand{\RMat}{\textbf{R}} %R matrices
% \newcommand{\HMat}{\textbf{H}} %H matrices
% \newcommand{\XMat}{\textbf{X}} %H matrices
% \newcommand{\mbar}{\bar{m}} %mBar dimension
% % \newcommand{\gauss}{\mathcal{D}} % gaussian distribution
% \newcommand{\Id}{\textbf{I}} % Identity matrix
% \newcommand{\er}{\textbf{e}} % gaussian distr. vectors
% % \newcommand{\cipher}{\textit{c}} % ciphertext
% \newcommand{\Olwe}{\mathcal{O}_{\textsf{LWE}}} %LWE oracle
% \newcommand{\OSample}{\mathcal{O}_{Sample}} %LWE oracle
% \newcommand{\SigmaB}{\boldsymbol{\Sigma}} %semi-deifinite matrix Sigma%
% % \newcommand{\mods}{\text{ mod}}


%Vectors and Matrices

\newcommand{\AMat}{\mathbf{A}} %A matrices
\newcommand{\BMat}{\mathbf{B}} %B matrices
\newcommand{\DMat}{\mathbf{D}} %Diagonal


\newcommand{\HMat}{\ensuremath{\mathbf{H}}}
\newcommand{\QMat}{\ensuremath{\mathbf{Q}}}
\newcommand{\Id}{\ensuremath{\mathbf{I}}}
\newcommand{\ZeroM}{\textbf{0}} % Zero matrix

\newcommand{\bvec}{\ensuremath{\mathbf{b}}}
\newcommand{\evec}{\ensuremath{\mathbf{e}}}
\newcommand{\svec}{\ensuremath{\mathbf{s}}}
\newcommand{\vvec}{\ensuremath{\mathbf{v}}}
\newcommand{\zvec}{\ensuremath{\mathbf{0}}}
\newcommand{\xvec}{\ensuremath{\mathbf{x}}}
\newcommand{\yvec}{\ensuremath{\mathbf{y}}}
\newcommand{\uvec}{\ensuremath{\mathbf{u}}}
\newcommand{\tvec}{\ensuremath{\mathbf{t}}}
\newcommand{\zerovec}{\ensuremath{\mathbf{0}}}

\newcommand{\nth}{^{\mathrm{th}}}

\newcommand{\RepMMT}{\ensuremath{\mathcal{R}_{\protect\SmallTextScript{0.70}{\texttt{MMT}}}}}
\newcommand{\RepBJMM}{\ensuremath{\mathcal{R}_{\protect\SmallTextScript{0.70}{\texttt{BJMM}}}}}
\newcommand{\XOR}{\ensuremath{\mathtt{3XOR}}}


% % % % % \newcommand{\mb}[1]{\mathbf{#1}} % does not compile otherwise
%%% Removed by Gotti; this is just asking to screw up with packages that (properly) define \mb (mathbold)

% \newcommand{\bL}{\|\bvec_1\|} % b1 length that appears way too often
% \newcommand{\dL}{\|\dvec_1\|} % b1 length that appears way too oftend

%Norms and Scalar products

\newcommand*\abs[1]{\left\lvert#1\right\rvert}
\newcommand*\norm[1]{\left\lVert#1\right\rVert}
\newcommand*\normalabs[1]{\lvert#1\rvert} 
\newcommand*\normalnorm[1]{\lVert#1\rVert}
\newcommand*\bignorm[1]{\bigl\lVert#1\bigr\rVert}
\newcommand*\bigabs[1]{\bigl\lvert#1\bigr\rvert}
\newcommand*\Bigabs[1]{\Bigl\lvert#1\Bigr\rvert}
\newcommand*{\ScProd}[2]{\ensuremath{\langle#1\mathbin{,}#2\rangle}} %Scalar Product
% \newcommand*{\ScProd}[2]{\ensuremath{\langle#1 \:{,}\:#2\rangle}} %Scalar Product
\newcommand*{\bigScProd}[2]{\ensuremath{\bigl\langle#1\mathbin{,}#2\bigr\rangle}} %Scalar Product
\newcommand*{\BigScProd}[2]{\ensuremath{\Bigl\langle#1\mathbin{,}#2\Bigr\rangle}} %Scalar Product


%Some other math operators

\DeclareMathOperator{\Span}{Span} %span of vectors
\DeclareMathOperator{\vol}{\mathrm{vol}} %volume
\DeclareMathOperator{\LW}{LambertW} %Lambert W function
\DeclareMathOperator{\SD}{SD}
\DeclareMathOperator{\gradient}{grad}
\DeclareMathOperator{\TRACE}{Tr}
\newcommand*{\dDR}{\mathrm{d}} %de-Rham-Differential (the d in dx, dy, dz and so on)


%Lists
\renewcommand{\L}{\ensuremath{\mathcal{L}}}

\renewcommand{\P}{\ensuremath{\mathcal{P}}}

\newcommand*{\Lout}{\ensuremath{\L_{\mkern-0.5mu\protect\SmallTextScript{0.85}{\textup{out}}}}}
\newcommand*{\Sout}{\ensuremath{S_{\mkern-0.5mu\protect\SmallTextScript{0.85}{\textup{out}}}}}
\newcommand{\wt}{\ensuremath{\mathit{wt}}}


\newcommand*{\softO}{\widetilde{\bigO}}

\newcommand{\const}{\mathsf{c}} 


\newcommand{\transpose}{\mkern0.7mu^{\mathsf{ t}}}

%proper overline reduced by 1.5mu
\newcommand{\overbar}[1]{\mkern 1.5mu\overline{\mkern-1.5mu#1\mkern-1.5mu}\mkern 1.5mu}

\DeclareMathOperator{\erf}{erf} %error function
\DeclareMathOperator{\erfc}{erfc} %complementary error function
\newcommand{\Er}{\ensuremath{\mathrm{Er}}} %complementary error function


% LATTICES

\newcommand{\Lat}{\ensuremath{\mathcal{L}}}
\newcommand*{\Sphere}[1]{\ensuremath{\mathsf{S}^{#1}}}
\DeclareMathOperator{\Conf}{Conf}

%Thick line for table
\setlength{\doublerulesep}{0pt}
\newcommand{\thickline}{\hline\hline\hline}


%circled text
\newcommand*\circled[1]{\tikz[baseline=(char.base)]{
    \node[shape=circle,draw,inner sep=0.3 pt] (char) {\scriptsize #1};}}


%Fix Algorithmicx package
\def\NoNumber#1{{\def\alglinenumber##1{}\State #1}\addtocounter{ALG@line}{-1}}

%For comments
\newcommand{\GColor}{ForestGreen}  %Damiens' color
\newcommand{\EColor}{MidnightBlue} %Elena's color

\newcommand*{\E}[1]{{\color{\EColor} #1} } 
\newcommand*{\G}[1]{{\color{\GColor} #1} } 

%Proper limit with the subscript underneath
% \newcommand{\Lim}[1]{\raisebox{0.5ex}{\scalebox{0.8}{$\displaystyle \lim_{#1}\;$}}}


%TIKZ dense dotted pattern

\pgfdeclarepatternformonly{my dots}{\pgfqpoint{-1pt}{-1pt}}{\pgfqpoint{2.0pt}{2.0pt}}{\pgfqpoint{2pt}{2pt}}%
{
	\pgfpathcircle{\pgfqpoint{0pt}{0pt}}{.35pt}
	\pgfpathcircle{\pgfqpoint{1pt}{1pt}}{.35pt}
	\pgfusepath{fill}
}


\tikzset{
	master/.style={
		execute at end picture={
			\coordinate (lower right) at (current bounding box.south east);
			\coordinate (upper left) at (current bounding box.north west);
		}
	},
	slave/.style={
		execute at end picture={
			\pgfresetboundingbox
			\path  (lower right)rectangle (upper left) ;
		}
	}
}