%!TEX root = main.tex
\section{Algebraic lattices and cryptographic applications}

La troisième partie du cours sera effectuée par Alice Pellet-Mary. Après des
éléments de théorie algébrique des nombres, elle définira les variantes
algébriques de LWE et SIS, ainsi que le problème NTRU,
et présentera leurs liens avec les réseaux algébriques. Elle construira alors
des primitives cryptographiques élémentaires à partir de ces problèmes.
Enfin, elle abordera des aspects de cryptanalyse et présentera des attaques
exploitant la structure algébrique sous-jacente à ces problèmes.


\subsection{Reminders about number theory}

Alice -- Rappels de théorie des nombres -- CM: 1h30 / TD: 1h

L'objectif de ce cours est de rappeler certains concepts de théorie algorithmique des nombres, qui seront utiles à la fois pour le cours d'isogénies et le cours sur les réseaux algébriques. Nous parlerons principalement de corps de nombres et d'idéaux (décomposition en idéaux premiers, groupe de classes, comportement des idéaux premiers dans des extensions de corps, etc), et de différents problèmes algorithmiques en lien avec ces objets, qui nous intéressent en cryptographie (manipulation des idéaux, calcul du groupe des classes, problème de l'idéal principal, etc).


\subsection{Ring and Module variants of LWE, and the NTRU problem}

Alice -- Réseaux algébriques en cryptographie -- CM: 3h / TD: 1h30

Ce cours sera consacré au variantes algébriques des problèmes LWE et SIS (définis dans le cours de Adeline Roux-Langlois). En plus des variantes algébriques de LWE et SIS, nous parlerons aussi du problème NTRU, qui n'a pas d'équivalent non-algébrique.
Le cours sera divisé en trois parties:
- nous commencerons par définir les variantes algébriques de LWE/SIS et le problème NTRU, et nous parlerons du lien entre ces problèmes et les réseaux (algébriques);
- nous verrons ensuite comment construire des primitives cryptographiques simples à partir de ces problèmes algorithmiques;
- enfin, nous parlerons de cryptanalyse et nous présenterons quelques attaques qui exploitent la structure algébrique de ces problèmes.

\subsection{Cryptographic applications}