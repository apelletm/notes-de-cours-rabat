\documentclass[11pt]{exam}
%%%%%%%%%%%%%%%%%%%%%%%%%%%%%%%%
%\noprintanswers % pour enlever les réponses
\printanswers

\unframedsolutions
\SolutionEmphasis{\small\color{blue}}
\renewcommand{\solutiontitle}{\noindent\textbf{A: }}
%%%%%%%%%%%%%%%%%%%%%%%%%%%%%%%%



\usepackage[margin=0.73in]{geometry}
%\usepackage[top=1in, bottom=1in, left=1in, right=1in]{geometry}

%\usepackage{fullpage}


\usepackage{hyperref}
\usepackage{appendix}
\usepackage{enumerate}


\usepackage{times,graphicx,epsfig,amsmath,latexsym,amssymb,verbatim}%,revsymb}
\usepackage{algorithmicx, enumitem, algpseudocode, algorithm, caption}


%%%%%%%%%%%%%%%%%%%%%
% Handling comments and versions %%%
%%%%%%%%%%%%%%%%%%%%%
\newcommand{\extra}[1]{}

\renewcommand{\comment}[1]{\texttt{[#1]}}


%%%%%%%%%%%%%%%%%%%%%%%%%%%
%% THEOREMS
%%%%%%%%%%%%%%%%%%%%%%%%%%%

\usepackage{amsmath,amssymb,amsfonts}
\usepackage{amsthm}

\newtheorem{theorem}{Theorem}[section]
\newtheorem{axiom}[theorem]{Axiom}
\newtheorem{conclusion}[theorem]{Conclusion}
\newtheorem{condition}[theorem]{Condition}
\newtheorem{conjecture}[theorem]{Conjecture}
\newtheorem{corollary}[theorem]{Corollary}
\newtheorem{criterion}[theorem]{Criterion}
\newtheorem{definition}[theorem]{Definition}
\newtheorem{lemma}[theorem]{Lemma}
\newtheorem{notation}[theorem]{Notation}
\newtheorem{proposition}[theorem]{Proposition}


\theoremstyle{definition}
\newtheorem{problem}{Problem}


\newcommand{\nc}{\newcommand}
\nc{\eps}{\varepsilon}
\nc{\RR}{{{\mathbb R}}}
\nc{\CC}{{{\mathbb C}}}
\nc{\FF}{{{\mathbb F}}}
\nc{\NN}{{{\mathbb N}}}
\nc{\ZZ}{{{\mathbb Z}}}
\nc{\PP}{{{\mathbb P}}}
\nc{\QQ}{{{\mathbb Q}}}
\nc{\UU}{{{\mathbb U}}}
\nc{\OO}{{{\mathbb O}}}
\nc{\EE}{{{\mathbb E}}}
\nc{\lat}{{{\mathcal L}}}
\nc{\GL}{{{\mathrm{GL}}}}


\pretolerance=1000

%%%%%%%%%%%%%%%%%%%%%%%%%%%%%%%%
%%%%%%%%%%%%%%%%%%%%%%%%%%%%%%%%
%% DOCUMENT STARTS
%%%%%%%%%%%%%%%%%%%%%%%%%%%%%%%%
%%%%%%%%%%%%%%%%%%%%%%%%%%%%%%%%
\usepackage{tikz}
\usetikzlibrary{automata}
\DeclareMathOperator{\Vol}{Vol}

\begin{document}

{\noindent
   \textsc{Algebraic lattices in cryptography}
   \hfill {\textsc{CIMPA school in Rabat -- October 2023}}\\
  }
  \hrule
% Titre de la feuille
  \begin{center}
    {\Large\textbf{
   \textsc{Exercises}
    } } 
  \end{center}
  \hrule \vspace{5mm}

\thispagestyle{empty}

\vspace{0.2cm}

%\Large

In all the exercises, unless specified otherwise, we take $P = X^d+1$ for $d$ a power-of-two, $K = \QQ[X]/P(X)$ and $R = \ZZ[X]/P(X)$.

\section{Canonical and coefficient embeddings}
In this exercise, we take $P = X^d+1$ for $d$ a power-of-two, $K = \QQ[X]/P(X)$ and $R = \ZZ[X]/P(X)$.

\begin{questions}
\question Show that the map from $\QQ^d$ to $\CC^d$ sending $\Sigma(a)$ to $\tau(a)$ (for $a \in K$) is a $\QQ$-linear morphism. Exhibit the matrix $M \in \mathrm{GL}_d(\CC)$ such that $\tau(a) = M \cdot \Sigma(a)$ for all $a \in K$.

\begin{solution}
If $\Sigma(a) = (a_0, \dots, a_{d-1})^T$, then $a = \sum_{i=0}^{d-1} a_i X^i$ and $\tau(a) = (a(\zeta_1), \dots, a(\zeta_d))^T$, where $\zeta_1, \dots, \zeta_d$ are the complex roots of $X^d+1$.
Define $M := (\zeta_i^{j-1})_{1\leq i,j\leq d}$. Then, we have $\tau(a) = M \cdot \Sigma(a)$. This shows that the map sending $\Sigma(a)$ to $\tau(a)$ is $\QQ$-linear.
Also, the matrix $M$ is a Vandermonde matrix, and we know that the determinant of this matrix is $\pm1 \cdot \prod_{i<j} (\zeta_i-\zeta_j)$. Since the $\zeta_i$'s are all distinct (because $P$ is irreducible over $\QQ$, so its complex roots are all distinct), we conclude that $\det(M) \neq 0$ and so $M$ is invertible.
\end{solution}

\question How can we compute $\Sigma(a)$ in polynomial time from $\tau(a)$? (this is equivalent to inverting the map $\tau$, since recovering $a$ from $\Sigma(a)$ is immediate).

\begin{solution}
We have seen that $\tau(a) = M \cdot \Sigma(a)$, and $M$ is invertible. Hence, to recover $\Sigma(a)$ from $\tau(a)$, it suffices to compute $M^{-1} $ (this can be done in polynomial time using Gaussian pivoting) and then $M^{-1} \cdot \tau(a)$.
\end{solution}

\question Show that $M \cdot M^* = d \cdot I_d$, where $M^* = \overline{M}^T$.\\
\textit{\color{gray}(Hint 1: you may want to prove first that if $\zeta \in \CC$ is a $m$-th root of unity different from $1$, then $\sum_{i = 0}^{m-1} \zeta^i = 0$)} \\
\textit{\color{gray}(Hint 2: to prove Hint 1, you can consider the equality $(\sum_{i = 0}^{m-1} X^i) \cdot (X-1) = X^m-1$}

\begin{solution}
Using the hint, observe that $(\sum_{i = 0}^{m-1} X^i) \cdot (X-1) = X^m-1$ for all $m > 0$. If $\zeta$ is a $m$-th root of unity, we then have $(\sum_{i = 0}^{m-1} \zeta^i) \cdot (\zeta-1) = \zeta^m-1 = 1-1 = 0$. If $\zeta \neq 1$, then $\zeta-1 \neq 0$ and so it must be that $\sum_{i = 0}^{m-1} \zeta^i = 0$. On the other hand, if $\zeta = 1$, then $\sum_{i = 0}^{m-1} \zeta^i = m$.

Let us now consider two rows $i$ and $j$ of $M$. The $(i,j)$-th coefficient of $M \cdot M^*$ is equal to the hermitian inner product between these two rows, i.e., it is $\sum_{k = 0}^{d-1} \zeta_i^k \overline{\zeta_j^k} = \sum_{k = 0}^{d-1} \zeta_i^k \cdot \zeta_j^{-k}$. Let us write $\zeta = \zeta_i \cdot \zeta_j^{-1}$. We have that $\zeta$ is a $d$-th root of unity. Indeed, $\zeta^d = \zeta_i^d / \zeta_j^d = -1/(-1) = 1$ (because $\zeta_i$ and $\zeta_j$ are roots of $X^d+1$). Moreover, $\zeta = 1$ if and only if $\zeta_i = \zeta_j$, which is the case if and only if $i = j$.

Hence, using the first part, we conclude that the $(i,j)$-th coefficient of $M \cdot M^*$ is equal to $\sum_{k = 0}^{d_1} \zeta^k = 0$ if $i \neq j$, and is equal to $d$ if $i = j$. Hence, $M \cdot M^* = d \cdot I_d$.
\end{solution}

\question Deduce from the previous question that we also have $M^* \cdot M = d \cdot I_d$.

\begin{solution}
We have proven that $M \cdot M^* = d \cdot I_d$. Hence, $M^* = d \cdot M^{-1}$, and so it commutes with $M$. Another way to prove this is to observe that $(M^* \cdot M) \cdot M^* = M^* \cdot (M \cdot M^*) = M^* \cdot d \cdot I_d = d M^*$. Since $M^*$ is invertible, we can divide by $M^*$ on the right and we obtain $M^* \cdot M = d \cdot I_d$.
\end{solution}

\question Conclude that $\|\tau(a)\| = \sqrt{d} \cdot \|\Sigma(a)\|$ for all $a \in K$.

\begin{solution}
For any $a \in K$, we have proven in the first question that $\tau(a) = M \cdot \Sigma(a)$. Hence, $\|\tau(a)\|^2 = \tau(a)^* \cdot \tau(a) = \Sigma(a)^* \cdot M^* \cdot M \cdot \Sigma(a) = d \cdot \Sigma(a)^* \cdot \Sigma(a) = d \cdot \|\Sigma(a)\|^2$. Taking the square root gives the desired result.
This equation shows that the geometry induced by the canonical embedding is the same as the geometry induced by the coefficient embedding (up to a scaling by a factor $\sqrt{d}$). Note that we crucially used here the fact that $P = X^d+1$ for $d$ a power-of-two. In general, the Vandermonde matrix $M$ is not orthogonal (even up to scaling), this is true here because the roots $\zeta_i$ are very special roots (they are power-of-two primitive roots of unity).
\end{solution}

\end{questions}

\section{Ideal lattices}
\label{exo:id-lat}

In this exercise again, we take $P = X^d+1$ for $d$ a power-of-two, $K = \QQ[X]/P(X)$ and $R = \ZZ[X]/P(X)$.

\begin{questions}
\question Show that if $a \in K$ is non-zero, then the $d$ vectors $\tau(a \cdot X^i)$ for $i = 0$ to ${d-1}$ are $\QQ$-linearly independent. \\
\textit{\color{gray}(Hint: you may want to use the fact that $\sigma: K \rightarrow \CC^d$ is injective)}

\begin{solution}
Let $z_0, \dots, z_{d-1} \in \QQ$ be such that $\sum_{i = 0}^{d-1} z_i \cdot \tau(a \cdot X^i) = 0$. We want to prove that $z_0 = \dots = z_{d-1}$ (which will prove that the $\tau(a X^i)$ are linearly independent).
We have 
\begin{align*}
0 = \sum_{i = 0}^{d-1} z_i \cdot \tau(a \cdot X^i) = \tau(\sum_{i = 0}^{d-1} z_i \cdot a \cdot X^i).
\end{align*}
Writing $z := \sum_{i = 0}^{d-1} z_i\cdot X^i$ and $y := \sum_{i = 0}^{d-1} z_i \cdot a \cdot X^i = a \cdot z$, we have that $\tau(y) = 0$, and so $y = 0$ by injectivity of $\tau$. Since $K$ is a field and $a \neq 0$ by assumption, this implies that $z = 0$, i.e., $z_0 = \dots = z_{d-1} = 0$ as desired.
\end{solution}

\question Show that for any $a,b \in R$ with $a \neq b$, then $\|\tau(a) - \tau(b)\| \geq \sqrt{d}$. \\
\textit{\color{gray}(Hint: you may want to use the fact that $\|\tau(x)\| = \sqrt{d} \cdot \|\Sigma(x)\|$)}

\begin{solution}
Let $c = a-b$ Since $a \neq b$, then $c \neq 0$, and we have $\|\tau(a) - \tau(b)\| = \|\tau(c)\|$. Hence, it suffices to prove that for all non zero $c \in R$, we have $\|\tau(c)\| \geq \sqrt{d}$. We know that $\|\tau(c)\| = \sqrt{d} \cdot \|\Sigma(c)\|$. Also, $\Sigma(c) \in \ZZ^d$ (because $R = \ZZ[X]/(X^d+1)$) and $\Sigma(c) \neq 0$ (because $c \neq 0$). Hence, $\|\Sigma(c)\| \geq 1$ (it has one coordinate which is non-zero and is an integer, so is $\geq 1$ in absolute value). And we conclude that $\|\tau(c)\| = \sqrt{d} \|\Sigma(c)\| \geq \sqrt{d}$ as desired.
\end{solution}

\question Conclude that for any non-zero ideal $\mathfrak{a}$, the set $\tau(\mathfrak{a})$ is a lattice of rank $d$ in $\CC^d$.

\begin{solution}
The set $\tau(\mathfrak{a})$ is stable by addition and subtraction since $\mathfrak{a}$ is (and $\tau$ is linear). Moreover, we have just proven that $\tau(\mathfrak{a})$ is discrete. It is then a discrete subgroup of $\CC^d \sim \RR^{2d}$, hence a lattice. To obtain the rank of this lattice, we use question $1$. We know that $\tau(\mathfrak{a}) \subseteq \tau(R)$, which is a lattice of rank $d$. So the rank of $\tau(\mathfrak{a})$ is at most $d$.
Let $a \in \mathfrak{a}$ be non-zero (such $a$ exists because $\mathfrak{a}$ is non-zero). Let $t_i = a \cdot X^{i-1}$ for $i = 1, \dots, d$. Since $X^{i-1} \in R$ and $\mathfrak{a}$ is an ideal, we know that $t_i \in \mathfrak{a}$ for all $i$'s. Moreover, we know from question $1$ that the vectors $\tau(t_i)$ are linearly independent. Hence, the rank of $\tau(\mathfrak{a})$ is $\geq d$, and we conclude that it is exactly $d$.
\end{solution}

\question Show that in any non-zero ideal $\mathfrak{a}$, it holds that $\lambda_1(\tau(\mathfrak{a})) = \cdots = \lambda_d(\tau(\mathfrak{a}))$. \\
\textit{\color{gray}(Hint: you may want to use question $1$ again.)}

\begin{solution}
Let $s \in \mathfrak{a}$ be an element reaching the first minimum, i.e., $\|\tau(s)\| = \lambda_1(\tau(\mathfrak{a}))$. Define $s_i = s \cdot X^{i-1}$ for $i = 1$ to $d$. From question $1$, we know that the $s_i$'s are linearly independent, so $\lambda_d(\tau(\mathfrak{a})) \leq \max_i \|\tau(s_i)\|$.
Recall that $\tau(s_i)$ is the product coordinate-wise of the vectors $\tau(s)$ and $\tau(X^{i-1})$. Hence, $\|\tau(s_i)\| \leq \|\tau(s)\| \cdot \|\tau(X^{i-1})\|_\infty$ (write it). By definition $\tau(X^{i-1}) = (\zeta_1^{i-1}, \dots, \zeta_d^{i-1})$ with $\zeta_j$ roots of unity (they are roots of $P = X^d+1$). Hence, $|\zeta_j^{i-1}| = 1$ and so $\|\tau(X^{i-1})\|_\infty \leq 1$. We conclude that $\lambda_d(\tau(\mathfrak{a})) \leq \|\tau(s)\| = \lambda_1(\tau(\mathfrak{a}))$. Since $\lambda_1(L) \leq \lambda_2(L) \leq \dots \leq \lambda_d(L)$ for any lattice $L$, we conclude that $\lambda_1(\tau(\mathfrak{a})) = \cdots = \lambda_d(\tau(\mathfrak{a}))$ as desired.
\end{solution}

\question Prove that if one knows a solution to SVP$_\gamma$ in $\mathfrak{a}$, then one can construct in polynomial time a solution to SIVP$_\gamma$ in $\mathfrak{a}$.

\begin{solution}
Let $s \in \mathfrak{a}$ be a solution to SVP$_\gamma$ in $\mathfrak{a}$. Construct $s_i = s \cdot X^{i-1}$ as in the previous question. We have seen in the previous question that these vectors $\tau(s_i)$ are linearly independent and $\|\tau(s_i)\| = \|\tau(s)\| \leq \gamma \cdot \lambda_1(\mathfrak{a}) = \gamma \cdot \lambda_d(\mathfrak{a})$ for all $i$'s. Hence, they form a solution to SIVP$_\gamma$ in $\mathfrak{a}$, and they are efficiently computable from $s$.
\end{solution}


\question Prove that the reciprocal is also true: if one knows a solution to SIVP$_\gamma$ in $\mathfrak{a}$, then one can construct in polynomial time a solution to SVP$_\gamma$ in $\mathfrak{a}$.

\begin{solution}
Let $s_1, \dots, s_d$ be a solution to SIVP$_\gamma$ in $\mathfrak{a}$. The $s_i$'s are non-zero and satisfy $\|\tau(s_i)\| \leq \gamma \lambda_d(\tau(\mathfrak{a})) = \gamma \lambda_1(\tau(\mathfrak{a}))$. Hence, any $s_i$ is a solution to SVP$_\gamma$ in $\tau(\mathfrak{a})$.
\end{solution}

\end{questions}

\section{Albrecht-Deo's reduction}
In this exercise again, we take $P = X^d+1$ for $d$ a power-of-two, $K = \QQ[X]/P(X)$ and $R = \ZZ[X]/P(X)$.

We will admit in this exercise that if $q$ is a prime integer and if the $a_i$ are sampled uniformly in $R_q^n$ for some $n > 1$, then with overwhelming probability, it suffices to sample a polynomial number of $a_i$'s to be able to extract $n$ of them $a'_1, \dots, a'_n$ such that the matrix whose rows are the $a'_i$'s is invertible in $R_q$.

\begin{questions}

\question Assume that we have access to an oracle computing samples from the distribution $D^{\textrm{MLWE}}_{n,q,\chi}(s)$ (for some $s \in R_q^n$), and assume that we have $n$ samples $(a_i, b_i)$ from $D^{\textrm{MLWE}}_{n,q,\chi}(s)$ such that
the matrix $\overline{A}$ whose rows are the $a_i$ is invertible in $R_q$. Let $b_i = a_i s + e_i$ with $e_i \leftarrow \chi$ and let us write $\overline{e} = (e_1, \dots, e_n)^T$ and $\overline{b} = (b_1, \dots, b_n)^T$.

Let $(a,b) \leftarrow D^{\textrm{MLWE}}_{n,q,\chi}(s)$. Define $a' = a \cdot \overline{A}^{-1}$ and $b' = \langle a', \overline{b} \rangle - b$. Show that $(a', b')$ is a sample from $D^{\textrm{MLWE}}_{n,q,\chi}(\overline{e})$.

\question Conclude that there is a polynomial time reduction from MLWE$_{n,q,\chi}$ to HNF-MLWE$_{n,q,\chi}$, which is a variant of MLWE where the secret is sampled from the distribution $\chi^n$ instead of being chosen uniformly in $R_q^n$.

\question Let $e \in R$ be such that $\|\tau(e) \| \leq \beta$ for some $\beta > 0$. Let $X = \{x \in R\,|\, \|\Sigma(x)\|_\infty \leq \beta'\}$ and $Y = \{x \in R\,|\, \|\Sigma(x-e)\|_\infty \leq \beta'\}$ for some $\beta' > 0$ not in $\ZZ$. Show that $|X| \leq (2\beta')^d$ and that $|X \cap Y| \geq (2(\beta'-\beta))^d$.

\question Let $\chi'$ be the uniform distribution over $\{x \in R\,|\, \|\Sigma(x)\|_\infty \leq \beta'\}$ where $\beta' = \beta \cdot 2^{d+1}d$. Assume that $\beta' \not \in \ZZ$, using the previous question, show that the statistical distance between $\chi'$ and $e+\chi'$ is $\leq 2^{-d}$.

\question Conclude the proof of Albrecht-Deo's reduction from the course.
\end{questions}

\section{Subfields and automorphisms}

In this exercise, we take $P = X^4+1$, $K = \QQ[X]/P(X)$ and $R = \ZZ[X]/P(X)$.
\begin{questions}
\question What are the automorphisms of $K$? Show that $\mathrm{Gal}(K)$ is isomorphic as a group to $(\ZZ/2\ZZ \times \ZZ/2\ZZ,+)$.
\question Deduce from the previous question that $K$ admits one subfield of degree $1$, three subfields of degree $2$ and one subfield of degree $4$.
\question Exhibit a basis for all the subfields from the previous question.

\end{questions}

\section{Canonical embedding and automorphisms}

In this exercise, we take $P = X^d+1$ for $d$ a power-of-two, $K = \QQ[X]/P(X)$ and $R = \ZZ[X]/P(X)$.

Let $\zeta$ be a primitive $(2d)$-th root of unity in $\CC$, and define, for $i = 1$ to $d$ the maps
\begin{align*}
\sigma_i: \ \ \ K &\rightarrow \CC \\
a(X) & \mapsto a(\zeta^{2i-1})
\end{align*}
These maps $\sigma_i$ are called the complex embeddings of $K$.
Recall that the canonical embedding $\tau$ of an element $a \in K$ is defined as $\tau(a) := (\sigma_1(a), \dots, \sigma_d(a)) \in \CC^d$.
%
Let $\varphi$ be some automorphism of $K$.

\begin{questions}
\question Show that for all $i \in \{1, \dots, d\}$, there exists $k \in \{1, \dots, d\}$ such that $\sigma_i \circ \varphi = \sigma_k$.
\question Show that the map $\sigma \mapsto \sigma \circ \varphi$ is actually a permutation over the set of complex embeddings $(\sigma_i)_{1 \leq i \leq d}$.
\question Conclude that for all $a \in K$, it holds that $\|\tau(\varphi(a))\| = \|\tau(a)\|$.
\end{questions}

\section{Short vectors in special ideals}

Let $\mathfrak{a}$ be an ideal of $K = \QQ[X]/(X^d+1)$. Let $H$ be its decomposition group (i.e., $H = \{\varphi \in \mathrm{Gal}(K)\,|\, \varphi(\mathfrak{a}) = \mathfrak{a}\}$), and let $K_H$ be the fixed field of $H$ (i.e., $K_H = \{x \in K \,|\, \varphi(x) = x, \, \forall \varphi \in H\}$).

\begin{questions}
\question Let $x \in \mathfrak{a}$ be non-zero. Define $w_i = \mathrm{Tr}_{K/K_H}(x \cdot X^{i-1})$ for $i \in \{1, \dots, d\}$. Show that there exists an index $i_0$ for which $w_{i_0}$ is non-zero.\\
\textit{\color{gray}(Hint: recall from Exercise~\ref{exo:id-lat} that the vectors $x \cdot X^{i-1}$ are linearly independent, hence they form a $\QQ$-basis of $K$.)}

\question Prove that, for all $i$, it holds that $\|\tau(w_i)\| \leq |H| \cdot \|\tau(x)\|$.\\
\textit{\color{gray}(Hint: recall that $\|\tau(y \cdot X)\| = \|\tau(y)\|$ for all $y \in K$.)}

\question Show that $w_i \in \mathfrak{b} := \mathfrak{a} \cap K_H$ for all $i$'s.\\
\textit{\color{gray}(Hint: this is where we use that $H$ is the decomposition group of $\mathfrak{a}$.)}

\question Conclude that $\lambda_1(\tau(\mathfrak{b})) \leq |H| \cdot \lambda_1(\tau(\mathfrak{a}))$.
\end{questions}

\section{NTRU}

For the first 3 questions, assume that $K = \QQ$, and that $q$ is a prime integer.

\begin{questions}
\question Let $(f,g) \in \ZZ^2$ with $(f,g) \neq (0,0)$ and $|f|, |g| < q$. Let $h, h' \in \ZZ$ such that $gh = f \bmod q$ and $g h' = f \bmod q$. Show that $h = h' \bmod q$.\\
\textit{\color{gray}(Hint: it may be useful to prove that $q$ is invertible modulo $q$.)}

\question Show that for any $B > 0$, the number of pairs $(f,g) \in \ZZ^2$ with $|f|,|g| \leq B$ is at most $(2B+1)^2$.

\question Deduce from the previous two questions that for $B < q$, the proportion of NTRU$_{q,B}$ instances in $\ZZ$ is $\leq \frac{(2B+1)^2}{q}$.\\
\textit{\color{gray}(Hint: observe that if $h$ is an NTRU$_{q,B}$ instance, then any $h' = h \bmod q$ is also an NTRU$_{q,B}$ instance, so it suffices to consider the $h$ in $\{0,\dots, q-1\}$.)}
\end{questions}

From now on, $K = \QQ[X]/(X^d+1)$ as usual, and let $q \geq 5$. Let $\chi$ be the uniform distribution over polynomials of $\ZZ[X]/(X^d+1)$ with coefficients in $\{-1,0,1\}$, and let $\psi$ be the distribution obtained by sampling $f,g \leftarrow \chi$ until $g$ is invertible mod $q$, and returning $h = f/g \bmod q$. Note that $\psi$ is a distribution over NTRU$_{q,B}$ instances for $B = d$ (because $\|\tau(f)\| = \sqrt{d} \cdot \|\Sigma(f)\| \leq d$ if $f \leftarrow \chi$).
Recall that the dec-NTRU$_{q,B,\psi}$ problem asks to distinguish $h \leftarrow \psi$ from $h \leftarrow \mathcal{U}(R_q)$.
\begin{questions}
\question Show that dec-NTRU$_{q,B,\psi}$ would be easy to solve if we had taken $h = f \bmod q$ instead of $h = f/g \bmod q$.
\question Show that dec-NTRU$_{q,B,\psi}$ would be easy to solve if we had taken $h = 1/g \bmod q$ instead of $h = f/g \bmod q$.
\end{questions}

\end{document}


