\documentclass[11pt]{exam}
%%%%%%%%%%%%%%%%%%%%%%%%%%%%%%%%
\noprintanswers % pour enlever les réponses
%\printanswers

\unframedsolutions
\SolutionEmphasis{\small\color{blue}}
\renewcommand{\solutiontitle}{\noindent\textbf{A: }}
%%%%%%%%%%%%%%%%%%%%%%%%%%%%%%%%



\usepackage[margin=0.73in]{geometry}
%\usepackage[top=1in, bottom=1in, left=1in, right=1in]{geometry}

%\usepackage{fullpage}


\usepackage{hyperref}
\usepackage{appendix}
\usepackage{enumerate}


\usepackage{times,graphicx,epsfig,amsmath,latexsym,amssymb,verbatim}%,revsymb}
\usepackage{algorithmicx, enumitem, algpseudocode, algorithm, caption}


%%%%%%%%%%%%%%%%%%%%%
% Handling comments and versions %%%
%%%%%%%%%%%%%%%%%%%%%
\newcommand{\extra}[1]{}

\renewcommand{\comment}[1]{\texttt{[#1]}}


%%%%%%%%%%%%%%%%%%%%%%%%%%%
%% THEOREMS
%%%%%%%%%%%%%%%%%%%%%%%%%%%

\usepackage{amsmath,amssymb,amsfonts}
\usepackage{amsthm}

\newtheorem{theorem}{Theorem}[section]
\newtheorem{axiom}[theorem]{Axiom}
\newtheorem{conclusion}[theorem]{Conclusion}
\newtheorem{condition}[theorem]{Condition}
\newtheorem{conjecture}[theorem]{Conjecture}
\newtheorem{corollary}[theorem]{Corollary}
\newtheorem{criterion}[theorem]{Criterion}
\newtheorem{definition}[theorem]{Definition}
\newtheorem{lemma}[theorem]{Lemma}
\newtheorem{notation}[theorem]{Notation}
\newtheorem{proposition}[theorem]{Proposition}


\theoremstyle{definition}
\newtheorem{problem}{Problem}


\newcommand{\nc}{\newcommand}
\nc{\eps}{\varepsilon}
\nc{\RR}{{{\mathbb R}}}
\nc{\CC}{{{\mathbb C}}}
\nc{\FF}{{{\mathbb F}}}
\nc{\NN}{{{\mathbb N}}}
\nc{\ZZ}{{{\mathbb Z}}}
\nc{\PP}{{{\mathbb P}}}
\nc{\QQ}{{{\mathbb Q}}}
\nc{\UU}{{{\mathbb U}}}
\nc{\OO}{{{\mathbb O}}}
\nc{\EE}{{{\mathbb E}}}
\nc{\lat}{{{\mathcal L}}}
\nc{\GL}{{{\mathrm{GL}}}}


\pretolerance=1000

%%%%%%%%%%%%%%%%%%%%%%%%%%%%%%%%
%%%%%%%%%%%%%%%%%%%%%%%%%%%%%%%%
%% DOCUMENT STARTS
%%%%%%%%%%%%%%%%%%%%%%%%%%%%%%%%
%%%%%%%%%%%%%%%%%%%%%%%%%%%%%%%%
\usepackage{tikz}
\usetikzlibrary{automata}
\DeclareMathOperator{\Vol}{Vol}

\begin{document}

{\noindent
   \textsc{Lattice-based crypto}
   \hfill {\textsc{Budapest Summer School -- August 2022}}\\
  }
  \hrule
% Titre de la feuille
  \begin{center}
    {\Large\textbf{
   \textsc{Tutorial 2}
    } } 
  \end{center}
  \hrule \vspace{5mm}

\thispagestyle{empty}

\vspace{0.2cm}

%\Large

\section{Canonical and coefficient embeddings}

\begin{questions}
\question Show that the map from $\QQ^d$ to $\CC^d$ sending $\Sigma(a)$ to $\tau(a)$ (for $a \in K$) is a $\QQ$-linear morphism. Exhibit the matrix $M \in \mathrm{GL}_d(\CC)$ such that $\tau(a) = M \Sigma(a)$ for all $a \in K$.

\question How can we compute $\Sigma(a)$ in polynomial time from $\tau(a)$? (this is equivalent to inverting the map $\tau$, since recovering $a$ from $\Sigma(a)$ is immediate).

\question Show that $M \cdot M^* = \sqrt{d} \cdot I_d$, where $M^* = \overline{M}^T$.\\
\textit{\color{gray}(Hint 1: you may want to prove first that if $\zeta \in \CC$ is a $m$-th root of unity different from $1$, then $\sum_{i = 0}^{m-1} \zeta^i = 0$)} \\
\textit{\color{gray}(Hint 2: to prove Hint 1, you can consider the equality $(\sum_{i = 0}^{m-1} X^i) \cdot (X-1) = X^m-1$}

\question Deduce from the previous question that we also have $M^* \cdot M = \sqrt{d} \cdot I_d$.

\question Conclude that $\|\tau(a)\| = \sqrt{d} \cdot \|\Sigma(a)\|$ for all $a \in K$.

\end{questions}

\section{Ideal lattices}

%Let $R$ be the ring $\ZZ[X]/(X^d+1)$ where $d$ is a power-of-two.

\begin{questions}
\question Show that if $a \in K$ is non-zero, then the $d$ vectors $\tau(a \cdot X^i)$ for $i = 0$ to ${d-1}$ are linearly independent. \\
\textit{\color{gray}(Hint: you may want to use exercise $1$)}

\question Show that for any $a,b \in R$ with $a \neq b$, then $\|\tau(a) - \tau(b)\| \geq \sqrt{d}$. \\
\textit{\color{gray}(Hint: you may want to use the fact that $\|\tau(x)\| = \sqrt{d} \cdot \|\Sigma(x)\|$)}

\question Conclude that for any non-zero ideal $\mathfrak{a}$, the set $\Sigma(\mathfrak{a})$ is a lattice of rank $d$ in $\CC^d$.

\question Show that in any non-zero ideal $\mathfrak{a}$, it holds that $\lambda_1(\tau(\mathfrak{a})) = \cdots = \lambda_d(\tau(\mathfrak{a}))$.

\question Prove that if one knows a solution to SVP$_\gamma$ in $\mathfrak{a}$, then one can construct in polynomial time a solution to SIVP$_\gamma$ in $\mathfrak{a}$.


\question Prove that the reciprocal is also true: if one knows a solution to SIVP$_\gamma$ in $\mathfrak{a}$, then one can construct in polynomial time a solution to SVP$_\gamma$ in $\mathfrak{a}$.

\end{questions}

\end{document}


