\documentclass[a4paper,11pt]{article}
\usepackage[english]{babel}
\usepackage[utf8]{inputenc}
\usepackage{amsmath,amssymb,amsthm,amsfonts}
\usepackage{comment}
\usepackage{fullpage}
\usepackage{xcolor}
\usepackage{url}
\usepackage{tikz}

%\usepackage[section]{algorithm}
%\usepackage{algorithmic}
\usepackage{algorithm,algorithmic}
\renewcommand{\algorithmicrequire}{\textbf{Input:}}
\renewcommand{\algorithmicensure}{\textbf{Output:}}
\renewcommand{\algorithmiccomment}[1]{{\color{gray}\textit{#1}}}


%\usepackage[a4paper, total={5in, 8in}]{geometry}



%opening
\title{Lattice-based cryptography}
\author{Elena Kirshanova\footnote{Technology Innovation Institute, Abu Dhabi, UAE}, Alice Pellet-Mary\footnote{CNRS \& Université de Bordeaux, Bordeaux, France}, Adeline Roux-Langlois\footnote{CNRS, GREYC, Caen, France}, Damien Stehlé\footnote{CryptoLab Inc., Lyon, France}}


%% Theorems environment
%% Only one counter for all, and depending on the section number
\newtheorem{theorem}{Theorem}[subsection]
\newtheorem{lemma}[theorem]{Lemma} %% the [theorem] says to use the same counter as for theorem
\newtheorem{properties}[theorem]{Properties}
\newtheorem{proposition}[theorem]{Proposition}
\newtheorem{heuristic}[theorem]{Heuristic}
\newtheorem*{remark}{Remark}
\theoremstyle{definition}
\newtheorem{definition}[theorem]{Definition} 
\newtheorem{example}[theorem]{Example}
\newtheorem{exercise}[theorem]{Exercise}
\newtheorem{notations}[theorem]{Notations}

%% To have the table of content appearing on the left
\usepackage{hyperref} 
\hypersetup{
    colorlinks = true,
    citecolor = blue,
}
\setcounter{tocdepth}{3}


\def\labelitemi{$\bullet$}

\newcommand{\mZ}{\mathbb{Z}}
\newcommand{\mR}{\mathbb{R}}
\newcommand{\mQ}{\mathbb{Q}}
\newcommand{\eps}{\varepsilon}
\newcommand{\NN}{\mathbb{N}}
\newcommand{\ZZ}{\mathbb{Z}}
\newcommand{\QQ}{\mathbb{Q}}
\newcommand{\RR}{\mathbb{R}}
\newcommand{\CC}{\mathbb{C}}
\newcommand{\EE}{\mathbb{E}}
\renewcommand{\Pr}{\mathbb{P}}
\newcommand{\Var}{\text{Var}}
\renewcommand{\vec}{\mathbf}
\DeclareMathOperator{\poly}{poly}
\DeclareMathOperator{\adv}{Adv}
\DeclareMathOperator{\dist}{dist}
\renewcommand{\O}{\mathcal{O}}
\newcommand{\Tr}{\mathrm{Tr}}
\newcommand{\N}{\mathcal{N}}
\newcommand{\A}{\mathcal{A}}

\newcommand{\im}{\mathrm{im}}
\renewcommand{\ker}{\mathrm{ker}}

\newcommand{\ps}[2]{\langle#1,#2\rangle}

%%% Alice's macro
\newcommand{\coeff}{\Sigma}
\newcommand{\mink}{\tau}
\DeclareMathOperator{\Aut}{Aut}
\DeclareMathOperator{\Gal}{Gal}


%%% Elena's macro
\newcommand{\smallo}{o}


% Adeline's macro
\newcommand{\norm}[1]{\Vert #1 \Vert}


\begin{document}


\maketitle

%{\color{red} [Alice: remove comments below]}

%Some conventions (?)
%\begin{itemize}
%\item $n$ the dimension of the lattice
%\item bold vectors and matrices (using \texttt{$\setminus$vec})
%\item column vectors
%\end{itemize}

%{\color{red} [Alice: I changed the default setting to have all theorems/lemmas/definitions/... use the same counter. And I made the counter start by the section number. Complain if you prefer something else.]}

\tableofcontents

%!TEX root = main.tex
\section{Background on Lattices}
\label{se:back}



This first section provides some background on Euclidean lattices, with illustrations
preparing the following sections.  


\subsection{Definitions}

We start with elementary definitions pertaining to Euclidean lattices. 
We refer the reader to~\cite{Siegel89} for additional material, including missing proofs.


A Euclidean lattice is the set of all integral linear combinations of a set if linearly independent vectors in a Euclidean space. 
An illustration is provided in Figure~\ref{fig:lattice1}. The lattice is the set of intersection points in the grid. 
The following is a more formal definition.

\begin{definition}
\label{def:lattice1}
Let~$m\geq n >0$. Let~${\bf b}_1, \ldots, {\bf b}_n$ be linearly independent vectors in~$\mR^m$.
The lattice that they span is
\[
 L\left({\bf b}_1, \ldots, {\bf b}_n\right) = \sum_{1 \leq i \leq n} \mZ \cdot {\bf b}_i = \left\{\sum_i x_i \cdot {\bf b}_i: \forall i, x_i \in \mZ\right\}. 
\]
If~$n=m$, then the lattice is said full-rank. If it is contained in~$\mZ^n$, then it is said integral. 
\end{definition}

The prototypical lattice is~$\mZ^n$, which can be written as the set of all integer linear combinations
of the canonical basis vectors~${\bf b}_i = (0^{i-1}, 1, 0^{n-i})^T$ for~$i \in \{1,\ldots,n\}$.  

Definition~\ref{def:lattice1} implies that a lattice is a subgroup of~$(\mR^m,+)$, i.e., for any~${\bf c}$ and~${\bf c}'$ in the lattice, the difference~${\bf c} - {\bf c}'$ also belongs to the lattice. It also implies that a lattice must be discrete, i.e., every sequence of lattice points that converges must be ultimately constant. (This can be seen by proved by considering the $i$-th coordinates of 
the vectors in the sequence, for any $i \leq n$, and using the discreteness of~$\mZ$.)  Graphically, this means that 
a lattice cannot have points of accumulation, i.e., that there is always a minimal distance separating any two given points of a
considered lattice. Figure~\ref{fig:lattice2} gives a subgroup of~$(\mR^2,+)$ that is not a lattice. 

In fact, being a discrete subgroup of~$(\mR^m,+)$ is an alternative definition of Euclidean lattices

\begin{lemma}
\label{def:lattice2}
Let~$m >0$. A set~$L \subset \mR^m$ is a lattice if and only if the following two conditions hold:
\begin{itemize}
\item[$\bullet$] for any ${\bf b}, {\bf b}' \in L$, we have~${\bf b} - {\bf b}' \in L$,
\item[$\bullet$] for any converging sequence~$({\bf c}_j)_{j=1,2,\ldots}$ with~${\bf c}_j \in L$ for all~$j$, there exists~$j^\star$ such that~${\bf c}_j = {\bf c}_{j^\star}$ for all~$j \geq j^\star$. 
\end{itemize}
\end{lemma}

For example, the set~$S= \mZ \cdot 1 + \mZ \cdot \sqrt{2}$ is not a lattice. Consider the continued fraction convergent~$p_k/q_k \in \mathbb{Q}$ of~$\sqrt{2}$, for~$k \geq 0$. Then it holds that~$|\sqrt{2} - p_k /q_k| \leq 1/q_k^2$ and $q_k \rightarrow_k +\infty$. Now, note that the element~$c_k := -p_k + q_k \cdot  \sqrt{2}$ belongs to~$S$ and satisfies~$|c_k| \rightarrow_k 0$. However, as~$\sqrt{2}$ is not rational, none of the $c_k$'s is $0$ and the sequence cannot be ultimately constant.

We now introduce two families of lattices playing a central role in lattice-based cryptography.

\begin{definition}
\label{def:qary}
Let~$m \geq n >0$ and~$q \geq 2$ be integers. Let~${\bf A} \in (\mZ/q\mZ)^{m \times n}$. 
\begin{itemize} 
\item[$\bullet$] The image lattice associated to~${\bf A}$ is defined as
\[
\Lambda({\bf A}) = {\bf A} \cdot (\mZ/q\mZ)^n + q \cdot \mZ^m = \{ {\bf y} \in \mZ^m: \exists {\bf s} \in (\mZ/q\mZ)^n, {\bf y} = {\bf A}\cdot {\bf s} \bmod q\};
\]
\item[$\bullet$] The image lattice associated to~${\bf A}$ is defined as
\[
\Lambda^\perp({\bf A}) = \{ {\bf x} \in \mZ^m: {\bf x}^T \cdot {\bf A} = {\bf 0} \bmod q\}.
\]
\end{itemize}
\end{definition}
{\color{red} [Alice: you said "image lattice associated to $A$" for both $\Lambda$ and $\Lambda^\perp$, is that a typo?]}

In the following, for the sake of compactness, the notation~$\mZ_q$ will refer to $\mZ/q\mZ$ (note that this does not refer 
to $q$-adic integers, which do not play any role in these lecture notes). 

By using Lemma~\ref{def:lattice2}, it can be checked that the sets $\Lambda({\bf A})$ and~$\Lambda^\perp({\bf A})$ are indeed lattices. As they are defined modulo~$q$, they are sometimes called~$q$-ary lattices. Also, due to their relevance in lattice-based cryptography, the lattice $\Lambda({\bf A})$ is often called the LWE lattice related to~${\bf A}$ and the lattice $\Lambda^\perp({\bf A})$ is often called the SIS lattice related to~${\bf A}$. The LWE and SIS problems are defined in Section~\ref{se:LBC}. The $\Lambda({\bf A})$ lattice is also known as the construction~A lattice 
applied to the linear code~${\bf A} \cdot \mZ_q^n$. The~``A'' in ``construction~A'' is unrelated to the choice of notation for matrix~${\bf A}$: construction~A refers to the first construction out of several, to obtain lattices from linear codes~\cite[Chapter~7]{CS99}. 
The following result shows that the two constructions are equivalent. 


\begin{lemma}
\label{le:qary_eq}
Let~$m \geq n >0$ and~$q \geq 2$ be integers, and~${\bf A} \in \mZ_q^{m \times n}$.
Let~${\bf H} \in \mZ_q^{m \times k}$ be such that~${\bf H} \cdot \mZ_q^k = \{ {\bf x} \in  \mZ_q^m : {\bf x}^T \cdot {\bf A} = {\bf 0} \bmod q\}$. Then:
\[
\Lambda^\perp({\bf A}) = \Lambda({\bf H}) \ \ \mbox{ and } \ \ \Lambda({\bf A}) =  \Lambda^\perp({\bf H}).
\]
\end{lemma}

\begin{proof}
Note  that the matrix~${\bf H}$ is well-defined and that its column span ${\bf H} \cdot \mZ_q^k$ is  the kernel of the 
map~$\phi_{\bf A}: {\bf x} \mapsto {\bf A}\cdot {\bf x} \bmod q$. This provides the first equality. Now, note that~${\bf A} \cdot \mZ_q^n$
is  the kernel of the map~$\phi_{\bf H}$. It is indeed contained in it by definition of~${\bf H}$ and equality follows by a cardinality argument
based on the isomorphism~$\im(\phi) \simeq \mZ_q^m / \ker (\phi)$ which holds for any linear map~$\phi$ whose domain is~$\mZ_q^m$:
\[
|\ker(\phi_{\bf H})| = \frac{q^m}{|\im(\phi_{\bf H})|} = \frac{q^m}{|\ker(\phi_{\bf A})|} = |\im( \phi_{\bf A})|.
\]
This gives the second equality.
\end{proof}

The statement of the lemma and its proof above are a little cumbersome, because we considered a modulus~$q$ of arbitrary arithmetic shape. If $q$ is assumed to be prime, then~$\mZ_q$ is a field and we are reduced to studying vector spaces over a finite field. For statements involving these lattices, we will often assume that~$q$ is prime. Most often, the results also hold up to minor modifications for more general moduli, but their proofs tend to be more technical.


We now introduce the notion of lattice duality. 

\begin{definition}
\label{def:dual}
Let $L$ be a lattice. We define the dual of $L$ as 
\[
\widehat{L} = \left\{ \widehat{\bf b} \in \mbox{span}_\mR (L): \forall {\bf b} \in L, \ps{{\bf b}}{\widehat{\bf b}} \in \mZ \right\}.
\]
\end{definition}

This can be seen as a generalization of the inverse over~$\mR \setminus \{0\}$, as $\widehat{x \cdot \mZ} = \frac{1}{x} \cdot \mZ$ for any~$x \in \mR \setminus \{0\}$. We have the following elementary properties on lattice duals. 

\begin{lemma}
\label{le:dual_props}
Let~$L$ be a lattice. Then~$\widehat{L}$ is a lattice and~$\widehat{\widehat{L}} = L$.
\end{lemma}

We now come back to our running example of construction A lattices.

\begin{lemma}
\label{le:constA_dual}
Let~$q\geq 2$ prime, $m>n>0$ and~${\bf A} \in \mZ_q^{m \times n}$. Then we have:
\[\widehat{\Lambda({\bf A})} = \frac{1}{q} \cdot \Lambda^\perp({\bf A}).\] 
\end{lemma}

\begin{proof}
We start with the inclusion $\widehat{\Lambda({\bf A})} \subseteq  (1/q) \cdot \Lambda^\perp({\bf A})$. 
Let $\widehat{\bf b} \in \widehat{\Lambda({\bf A})}$. By definition, for 
all~${\bf s} \in \mZ_q^n$ and all ${\bf y} \in \mZ^m$ such that~${\bf y} = {\bf A}\cdot {\bf s} \bmod q$, 
we have~$\ps{\widehat{\bf b}}{{\bf y}} \in \mZ$. Taking~${\bf s}= 0$ and setting~${\bf y} = (0^{i-1},q,0^{m-i})^T$ for all~$i \leq m$, we obtain that $\widehat{\bf b} \in (1/q) \cdot \mZ^m$. Let us write $\widehat{\bf b} = (1/q) \cdot {\bf k}$ for~${\bf k} \in \mZ^m$. 
By definition of~$\widehat{\Lambda({\bf A})}$, for 
all~${\bf s} \in \mZ_q^n$ and all ${\bf y} \in \mZ^m$ such that~${\bf y} = {\bf A}\cdot {\bf s} \bmod q$, 
we have~$\ps{{\bf k}}{{\bf y}} \in q \cdot \mZ$ and hence~${\bf k}^T \cdot {\bf A} \cdot {\bf s} = 0 \bmod q$. By setting~${\bf s} = (0^{i-1},1,0^{n-i})^T$ for all~$i \leq n$, we finally obtain that~${\bf k}^T \cdot {\bf A}= {\bf 0} \bmod q$.

We now prove the reverse inclusion. For this purpose, take~${\bf k} \in \mZ^m$ such that ${\bf k}^T \cdot {\bf A}= {\bf 0} \bmod q$.
We would like to show that for all~${\bf s} \in \mZ_q^n$ and all~${\bf y} \in \mZ^m$ such that~${\bf y} = {\bf A} \cdot {\bf s} \bmod q$, we have~$\ps{(1/q) \cdot {\bf k}}{{\bf y} } \in \mZ$. Note that the latter is equivalent to~${\bf k}^T \cdot {\bf A} \cdot {\bf s} = 0 \bmod q$: it is implied by the assumption that  ${\bf k}^T \cdot {\bf A}= {\bf 0} \bmod q$.
\end{proof}



\subsection{Bases and volume}

Definition~\ref{def:lattice1} states that for any lattice~$L \subset \mR^m$, there exist ${\bf b}_1, \ldots, {\bf b}_n$  
linearly independent in~$\mR^m$ such that~$L= \sum_{i \leq n} \mZ\cdot {\bf b}_i$. The vectors~${\bf b}_1, \ldots, {\bf b}_n$  
are said to form a \emph{basis} of~$L$. If~${\bf B} \in \mR^{m \times n}$ is the matrix whose columns are the~${\bf b}_i$'s (in any order), then~${\bf B}$ has rank~$n$, $L = {\bf B} \cdot \mZ^n$ and~${\bf B}$ is called a \emph{basis matrix} of~$L$.

Bases are not unique (one can always replace a basis vector by its opposite), but they all share the same cardinality~$n$, called the \emph{dimension} of~$L$. Recall that~$n$ can be lower than the embedding dimension~$m$.  

\begin{lemma}
\label{le:dim}
If ${\bf b}_1, \ldots, {\bf b}_n$ and ${\bf c}_1, \ldots, {\bf c}_{n'}$ are two bases of the same lattice~$L$, then we have that~$n=n'$. 
Further, there exists~${\bf U} \in \mZ^{n \times n}$ with~$\det ({\bf U}) = \pm 1$ such that 
\[
\left({\bf c}_1|\ldots |{\bf c}_n \right) = \left({\bf b}_1|\ldots |{\bf b}_n \right) \cdot {\bf U}.
\]
\end{lemma}

Note that~${\bf U}$ is multiplied from the right, as we consider column vectors. 
A matrix~${\bf U} \in \mZ^{n \times n}$ with~$\det ({\bf U}) = \pm 1$ is called \emph{unimodular}. Such matrices are exactly
the integer matrices whose inverses are integer matrices, i.e., the elements of~$\mbox{GL}_n(\mZ)$. Examples of unimodular matrices include permutation matrices, diagonal matrices with entries equal to~$\pm 1$ and upper triangular integer matrices with~$1$'s on the diagonal. In fact, it can be proved that any unimodular matrix can be written as a product of such matrices. 

Lemma~\ref{le:dim} implies
that a 1-dimensional lattice~$L = x \cdot \mZ$ with~$x \in \mR\setminus \{0\}$ has exactly  two bases: $x$ and~$-x$. When~$n \geq 2$, 
Lemma~\ref{le:dim} implies that a given lattice~$L$ admits infinitely many lattice bases: indeed, unimodular matrices include integer upper-triangular matrices with $1$'s on the diagonal, and all of these give different bases.  

As the dimension~$n$ is shared across all bases of a given lattice, it is called a \emph{lattice invariant}. 
In the rest of this subsection, we will introduce a few other such lattice invariants. 

\begin{definition}
\label{def:vol}
Let~$L$ be a lattice with a basis matrix~${\bf B}$. The determinant of~$L$ is defined as
\[
\det(L) = \sqrt{\det( {\bf B}^T \cdot {\bf B})}.
\]
\end{definition}

The determinant of a lattice is sometimes called the volume or co-volume. Note that when~$L$ is full-rank, the definition can 
be simplified to~$\det(L) = |\det ({\bf B})|$. The determinant is indeed a lattice invariant, as formally stated in the 
following lemma. 

\begin{lemma}
Let~${\bf B}$ and~${\bf C}$ be two matrix bases of the same lattice~$L$. Then we have
 \[
 \sqrt{\det( {\bf B}^T \cdot {\bf B})} =  \sqrt{\det( {\bf C}^T \cdot {\bf C})}.
\]
\end{lemma}

The proof follows by considering a unimodular matrix~${\bf U}$ such that ${\bf C} = {\bf B} \cdot {\bf U}$. 
Geometrically, and as illustrated in Figure~\ref{fig:det}, the determinant is the volume of the parallelepiped~${\bf B} \cdot [0,1]^n$. 

The following result focuses on dual lattices. 

\begin{lemma}
\label{le:dual_det}
Let~$L$ be a lattice with a basis matrix~${\bf B}$. Then~$\widehat{{\bf B}} = {\bf B} \cdot ( {\bf B}^T \cdot {\bf B})^{-T}$ 
is a basis of~$\widehat{L}$, and~$\det(\widehat{L}) = 1/\det(L)$.  
\end{lemma}
 
Note that when the lattice is full-rank, the definition of~$\widehat{{\bf B}}$ simplifies to~$\widehat{{\bf B}} =  {\bf B}^{-T}$. 

We now return to our running example of Construction~A lattices. 

\begin{lemma}
\label{le:consA_dimdet}
Let~$q\geq 2$ prime, $m>n>0$ and~${\bf A} \in \mZ_q^{m \times n}$. 
We have 
\begin{eqnarray*}
\dim (\Lambda({\bf A})) = m \ \ &\mbox{ and }& \ \ \det(\Lambda({\bf A})) = q^{m - r}, \\
\dim (\Lambda^{\perp}({\bf A})) = m \ \ &\mbox{ and }& \ \ \det(\Lambda^{\perp}({\bf A})) = q^{r},
\end{eqnarray*}
where~$r$ is the rank of~${\bf A}$. Further, if~${\bf A}$ is sampled uniformly, then its 
rank is~$n$ with probability~$\geq 1 - q^{-m+n}$. 
\end{lemma}

We stress that the dimension of these lattices is indeed~$m$ (even though in the rest of this section, the lattice dimension is denoted
by~$n$).\footnote{DS: les notations sont sous-optimales ici} 


\begin{proof}
The dimension equalities follow from the fact that~$q \mZ^m$ is contained in 
both~$\Lambda({\bf A})$ and~$\Lambda^{\perp}({\bf A})$. By Lemmas~\ref{le:constA_dual} and~\ref{le:dual_det},
the determinant equality for~$\Lambda({\bf A})$ and~$\Lambda^{\perp}({\bf A})$ are equivalent. We prove the first one
by explicitly constructing a basis for~$\Lambda({\bf A})$. 

As the rank of~${\bf A}$ is~$r$, there exist rank-$r$ matrices~${\bf A}' \in \mZ_q^{m \times r}$ and~${\bf T} \in \mZ_q^{r \times n}$ such  that~${\bf A} = {\bf A}' \cdot {\bf T} \bmod q$. Up to reordering the rows of~${\bf A}'$, we may assume that its first~$r$ rows form 
an invertible matrix. By left-multiplying~${\bf T}$ (resp.\ right-multiplying~${\bf A}'$) by this matrix (resp.\ its inverse), we can assume that the first~$r$ rows of~${\bf A}'$ form the identity matrix. Let us write~${\bf A}' = ({\bf Id} | {\bf A}''^T)^T$ for~${\bf A}'' \in \mZ_q^{(m-r) \times r}$. By using the definition of~$\Lambda({\bf A})$, we can write:
\[
\Lambda({\bf A}) = {\bf A} \cdot \mZ_q^n + q\cdot \mZ^m = \left[ \begin{array}{c|c|c} 
{\bf Id} & q \cdot {\bf Id} & {\bf 0} \\ \hline
{\bf A}'' & {\bf 0} &  q \cdot {\bf Id}
\end{array}\right] \cdot \mZ^{m+r},
\]
where the first block-row has~$r$ rows and $r+r+(m-r)$ columns and the coefficients of~${\bf A}''$ are viewed as integers. 
Note that this matrix has linearly dependent columns and hence
cannot be a basis of~$\Lambda({\bf A})$. However, the second block-column (i.e., columns $r+1$ to~$2r$) can be seen to be linearly 
dependent with the others. For example, the $(r+1)$-th column is $q$ times the first column to which one has subtracted the  columns of the third block  mutiplied by the first column of~${\bf A}''$. The second block is hence superfluous and we can write:
  \[
\Lambda({\bf A}) =  \left[ \begin{array}{c|c} 
{\bf Id} &  {\bf 0} \\ \hline
{\bf A}'' &   q \cdot {\bf Id}
\end{array}\right] \cdot \mZ^{m}.
\]
As the matrix is lower triangular with non-zero diagonal coefficients, its columns are linearly independent. They hence form a basis of~$\Lambda({\bf A})$, and we can compute the lattice determinant by counting the number of occurrences of~$q$ on the diagonal. 

Now, assume that~${\bf A}$ is sampled uniformly in~$\mZ_q^{m \times n}$. It is of rank~$n$ if and only if the $i$-th column is linearly independent from the previous ones, for each~$i \leq n$. The probability~$p$ that this occurs satisfies:
\[
1- p \leq \sum_{i=1}^n q^{-m+i-1}  = q^{-m} \frac{q^n-1}{q-1} \leq q^{-m+n}.
\]
Re-arranging terms allows to complete the proof.
\end{proof}

\textcolor{red}{Damien: Do we need the HNF? How to get a short basis from short vectors?}
\textcolor{red}{Alice: I don't need it}



\subsection{QR-factorisation and Gram-Schmidt orthogonalisation}

Given a full-rank matrix~${\bf B} \in \mR^{m \times n}$, there exist~${\bf Q} \in \mR^{m \times n}$ 
and~${\bf R} \in \mR^{n \times n}$ such that:
\begin{itemize}
\item[$\bullet$] ${\bf B} = {\bf Q} \cdot {\bf R}$;
\item[$\bullet$] the columns of~${\bf Q}$ are orthonormal vectors, i.e., ${\bf Q}^T \cdot {\bf Q} = {\bf Id}$;
\item[$\bullet$] ${\bf R}$ is upper-triangular with positive diagonal coefficients. 
\end{itemize}
Further, there exists a unique pair~$({\bf Q}, {\bf R})$ such that the above three properties hold. The matrices~${\bf Q}$ and~${\bf R}$ are respectively called the \emph{Q-factor} and \emph{R-factor} of~${\bf B}$, and the pair is referred to as its \emph{QR-factorisation}. 

The QR-factorisation is closely related to the Gram-Schmidt orthogonalisation of the column 
vectors~$({\bf b}_1,\ldots, {\bf b}_n)$ of~${\bf B}$. For all~$i$, we define:
\[
\widetilde{{\bf b}}_i = {\bf b}_i - \sum_{j <i} \mu_{ij}\cdot \widetilde{{\bf b}}_j, \ \ \mbox{ where } \ \ 
 \mu_{ij} = \frac{\ps{{\bf b}_i}{\widetilde{{\bf b}}_j}}{\|\widetilde{{\bf b}}_j\|^2}.
\]
The vectors~$\widetilde{{\bf b}}_1, \ldots, \widetilde{{\bf b}}_n$ are called the \emph{Gram-Schmidt orthogonalisation}
of~$({\bf b}_1,\ldots, {\bf b}_n)$. We stress that the Gram-Schmidt orthogonalisation depends on the order of the considered vectors
(and that the Gram-Schmidt orthogonalisation of permuted vectors is not the permutation of the Gram-Schmidt orthogonalisation). 
The vectors $\widetilde{{\bf b}}_i$ are pair-wise orthogonal. 

The correspondence between QR-factorisation and Gram-Schmidt orthogonalisation is formalized in the following lemma. 

\begin{lemma}
Let~${\bf B} \in \mR^{m \times n}$ be full-rank with columns~$({\bf b}_1,\ldots, {\bf b}_n)$. Let~$({\bf Q}, {\bf R})$ denote the QR-factorisation of~${\bf B}$.  Let~$(\widetilde{{\bf b}}_1, \ldots, \widetilde{{\bf b}}_n)$ be the Gram-Schmidt orthogonalisation of~$({\bf b}_1,\ldots, {\bf b}_n)$ and~$\mu_{ij} =  \ps{{\bf b}_i}{\widetilde{{\bf b}}_j} / \|\widetilde{{\bf b}}_j\|^2$ for~$i \geq j$. The following relations hold.
\begin{itemize}
\item[$\bullet$] for every $i$, the $i$-th column of~${\bf Q}$ is $\widetilde{{\bf b}}_i / \|\widetilde{{\bf b}}_i\|$;
\item[$\bullet$] for every $i$, we have $\|\widetilde{{\bf b}}_i\| = r_{ii}$;
\item[$\bullet$] for every $i \geq j$, we have $\mu_{ij} = r_{ji}/r_{jj}$.
\end{itemize}
\end{lemma} 
 
Given the above equivalency, one may prefer to work with the QR-factorization or the Gram-Schmidt orthogonalisation. 
In these notes, we will typically prefer to work with the QR-factorization, as it is typically more compact. One advantage of the 
Gram-Schmidt orthogonalisation over the QR-factorization is that if the matrix~${\bf B}$ is integral, then the $\widetilde{{\bf b}}_i$'s 
and~$\mu_{ij}$'s are rationals. Further, the bit-sizes of the numerators and denominators is polynomial in the bit-sizes of the 
coefficients of~${\bf B}$ and the Gram-Schmidt orthogonalization can be computed exactly in polynomial-time 
(with respect to $m$ and  the bit-sizes of the coefficients of~${\bf B}$). Oppositely, the QR-factorisation may not be rational. 
However, one could represent the QR-factorisation with the Gram-Schmidt orthogonalisation. More interestingly, in practice, the 
rationals involved in the Gram-Schmidt orthogonalisation are very large, and one prefers to resort to floating-point computations. 
As the numerical stability of the QR-factorisation algorithms has been extensively studied (see, e.g., \cite{Higham02}), in the context 
of such approximate computations, it is typically more convenient to use the QR-factorisation. 

The lattice determinant can be derived from the QR-factorization of any lattice basis. 

\begin{lemma}
Let~$L \subset \mR^m$ be an $n$-dimensional lattice. Let~${\bf B}$ be the matrix representation of a basis of~$L$, and~${\bf R}$ 
be its R-factor. Then we have:
\[\det(L) = \prod_{i \leq n} r_{ii}.\]
\end{lemma} 


\textcolor{red}{Damien: Do we need the R-factor of the dual?}\\
\textcolor{red}{Alice: I don't.}




\subsection{Lattice minimum}

We have already introduced the lattice determinant, which is an invariant of the lattice (i.e., it does not depend on the 
specific choice of the basis). Note that if the lattice is integral and given by a basis, then the determinant can be computed 
in polynomial time. 
We now focus on another lattice invariant, namely the lattice minimum, which is not known to be computable efficiently.  

\begin{definition}
Let~$L$ be a lattice. Its minimum~$\lambda_1(L)$ is the (Euclidean) norm of any shortest non-zero vector of~$L$. 
\end{definition}

The existence of a shortest non-zero vector in~$L$ is guaranteed by the discreteness of the lattice (Lemma~\ref{def:lattice2}). 
This implies that~$\lambda_1(L)$ is well-defined. Note that the minimum is always reached at least twice: if a vector is in a lattice 
and has a given norm, so does its opposite. It may be reached more times. The choice of the Euclidean norm is somewhat arbitrary:
it is mostly justified by technical convenience.  

The litterature sometimes considers the packing radius of a lattice, defined as the largest real~$r>0$ such that any two balls of radius~$r$ centered in lattice points do not overlap more than on a single point. This is exactly half the lattice minimum.  

The QR-factorisation provides an efficiently computable lower bound for the lattice minimum. 

\begin{lemma}
\label{le:lambda_lower_bound}
Let~$L \subset \mR^m$ be an $n$-dimensional lattice. Let~${\bf B}$ be the matrix representation of a basis of~$L$, and~${\bf R}$ 
be its R-factor. Then we have:
\[ \lambda_1(L) \geq  \min_{i \leq n} r_{ii}.\]
\end{lemma}

\begin{proof}
Let~${\bf s} \in L$ with~$\|{\bf s}\| = \lambda_1(L)$. 
There exists a non-zero  vector~${\bf k} \in \mZ^n$ such that~${\bf s} = {\bf B} \cdot {\bf k}$. 
As the columns of the Q-factor of~${\bf B}$ are orthonormal, we have that
\[
\lambda_1(L) = \|{\bf s}\| = \|{\bf B} \cdot {\bf k}\| = \|{\bf R} \cdot {\bf k}\|. 
\]
Let~$i$ be the index of the last non-zero coefficient of~${\bf k}$. Then the $i$-th coefficient of  ${\bf R} \cdot {\bf k}$ 
is $r_{ii} \cdot k_i$. We must then have
\[
\lambda_1(L) = \|{\bf R} \cdot {\bf k}\| \geq r_{ii} \cdot |k_i|.
\]
As $k_i$ is a non-zero integer, we have~$|k_i| \geq 1$, leading to the inequality~$\lambda_1(L) \geq r_{ii}$.
\end{proof}



Minkowski's theorem gives an upper bound for the minimum of a lattice, as a function of its determinant. If the lattice is integral, the
upper bound can be computed in polynomial-time to an arbitrary precision. 

\begin{theorem}
\label{th:Minko}
For every lattice~$L$ of dimension~$n$, we have~$\lambda_1(L) \leq \sqrt{n} \cdot \det(L)^{1/n}$.
\end{theorem}

Note that the minimum can be arbitrarily small compared to the $n$-th root of the determinant, as can be seen from the lattice
spanned by the columns of the diagonal matrix~$\mbox{diag}(\eps, 1/\eps)$ for~$\eps>0$ going to~$0$. This does not contradict Lemma~\ref{le:lambda_lower_bound}, as $r_{11} = \eps$. We note that the upper bound from Theorem~\ref{th:Minko} can 
also be expressed in terms of the R-factor~${\bf R}$  of an arbitrary basis of~$L$: 
indeed, the quantity $\det(L)^{1/n}$ is the geometric mean of the diagonal coefficients~$r_{ii}$ (for~$i \leq n$). 

We will prove Minkowski's theorem for the Construction~A lattices, up to a loss of a factor~2. (Recall that their determinants were computed in Lemma~\ref{le:consA_dimdet}.)

\begin{lemma}
\label{le:Minko_constA}
Let~$q\geq 2$ prime, $m>n>0$ and~${\bf A} \in \mZ_q^{m \times n}$. Let~$r$ 
denote the rank of~${\bf A}$. 
We have 
\[
\lambda_1( \Lambda({\bf A}))  \leq 
 2\sqrt{m}  \cdot  q^{1-r/m} \ \  \mbox{ and } \ \ 
\lambda_1 (\Lambda^{\perp}({\bf A})) \leq  2 \sqrt{m} \cdot  q^{r/m}.
\]
\end{lemma}


\begin{proof}
 Let us consider all the vectors~${\bf x} \in \mZ^m$ with~$\|{\bf x}\|_\infty \leq \rho$, for some~$\rho$ to be determined later. If there are more than~$q^r$ such vectors, then, by the pigeon hole principle,  
there exist ${\bf x}, {\bf x}' \in \mZ^m$ distinct such that~${\bf x}^T \cdot {\bf A} =  {\bf x'}^T \cdot {\bf A} \bmod q$ and~$\|{\bf x}\|_\infty,\|{\bf x}'\|_\infty \leq \rho$. The vector~${\bf x} - {\bf x}'$ is hence non-zero, belongs to $\Lambda^{\perp}({\bf A})$ and has infinity norm~$\leq 2 \sqrt{m} \cdot \rho$. This implies that  $\lambda_1(\Lambda^{\perp}({\bf A})) \leq 2 \sqrt{m} \cdot  \rho$. 
Setting~$\rho = q^{r/m}$ gives the result. 

Lemma~\ref{le:qary_eq} allows to obtain the desired bound for~$\Lambda({\bf A})$.
\end{proof}

By using the specificities of the Construction~A lattices, we can obtain bounds that can be lower for some ranges of parameters. 

\begin{lemma}
\label{le:Minko_constA_optimized}
Let~$q\geq 2$ prime, $m>n>0$ and~${\bf A} \in \mZ_q^{m \times n}$. Let~$r$ 
denote the rank of~${\bf A}$. 
We have 
\[
\lambda_1( \Lambda({\bf A}))  \leq 
 \min \left( 2\sqrt{m}  \cdot  q^{(m-r)/m}, q \right) \ \  \mbox{ and } \ \ 
\lambda_1 (\Lambda^{\perp}({\bf A})) \leq  \min\left( \min_{1 \leq  \mu \leq m} 2 \sqrt{\mu} \cdot  q^{r/\mu}, q\right).
\]
\end{lemma}

Note that the function $\mu \mapsto 2 \sqrt{\mu} \cdot  q^{r/\mu}$ first decreases but then increases with~$\mu$. As a result, 
for~$m < 2 r \log q$, the second bound is $\min(2 \sqrt{m} \cdot  q^{r/m}, q)$ and for $m \geq 2 r \log q$ and~$q \geq 3$, 
it is $\leq \min(4\sqrt{r \log q}, q)$.

\begin{proof}
These lattices always contain~$\mZ_q^m$, which explains why their minimum is no greater than~$q$. 
For the $\Lambda^{\perp}({\bf A})$ lattice, one may further observe that $\Lambda^{\perp}({\bf A}') \subseteq \Lambda^{\perp}({\bf A})$  for any~${\bf A}' \in \mZ_q^{\mu \times n}$ made of $\mu$ out of the~$m$ rows of~${\bf A}$, and further that the rank of~${\bf A}'$ is no larger than the rank of~${\bf A}$. 
\end{proof}

\begin{comment}
2 (1/2sqr(x) q^r/x + sqrt(x)*r*logq*q^r/x*(-1/x^2))
= q^(r/x) x^(-3/2)  ( x - 2r*logq )
=> decreasing up to x = floor (2r log q)
2sqrt(2rlogq) * exp(rlog q / floor (2r log q))
\end{comment}

It turns out that Minkowski's theorem is essentially sharp if we consider random lattices for an appropriate distribution over the 
set of lattices. This is known as the Minkowski-Hlawka theorem. We prove a variant for Construction~A lattices.

\begin{theorem}
\label{thm:Minkowski-Hlawka}
Let $q\geq 2$ prime, $m  > n>0$ and~${\bf A}$ sampled uniformly in~$\mZ_q^{m \times n}$. Then we have
\begin{eqnarray*}
            \lambda_1( \Lambda({\bf A})) & \geq  & \min \left(\sqrt{m} \cdot q^{1-n/m}/10 -\sqrt{m/2}, q \right), \\
            \lambda_1( \Lambda^{\perp}({\bf A})) & \geq & \min \left(\sqrt{m} \cdot q^{n/m}/10 -\sqrt{m/2}, q \right),
\end{eqnarray*}
where each bound holds with probability~$\geq 1-2^{-m}$.
\end{theorem}
Note that for some values of the parameters, notably $m$ very large in the second case, 
the lower bound on the lattice minimum may be negative. In such a case, the statement is vacuous. 
For many parameters, the lower bounds are within a factor~10 of the upper bounds given by Minkowski's theorem.

\begin{proof}
For~$\rho < q$, the event~$\lambda_1(\Lambda({\bf A})) \leq \rho$ occurs exactly if there exists~${\bf s} \in \mZ_q^n \setminus \{\bf 0\}$ 
such that $0 < \|{\bf A} \cdot {\bf s} \bmod q \|\leq \rho$. Here the reduction modulo~$q$ takes values in~$[-q/2,q/2)$. 
By the union bound, we have:
\begin{eqnarray*}
\Pr_{\bf A} \left[\lambda_1(\Lambda({\bf A})) \leq \rho \right] & = & \Pr_{\bf A} \left[\exists {\bf y} \in \mathcal{B}({\bf 0},\rho) \cap \mZ^m, \exists {\bf s} \in \mZ_q^n \setminus \{\bf 0\}: {\bf A} \cdot {\bf s} = {\bf y} \bmod q \right] \\
&\leq& 
\sum_{\substack{{\bf y} \in \mathcal{B}({\bf 0},\rho) \cap \mZ^m\\ {\bf s} \in \mZ_q^n \setminus \{\bf 0\}}} 
 \Pr_{\bf A} \left[{\bf A} \cdot {\bf s} = {\bf y} \bmod q \right].
\end{eqnarray*}
As the rows of~${\bf A}$ are statistically independent, ${\bf s}$ is non-zero and~$q$ is prime, we have that $\Pr_{\bf A} [{\bf A} \cdot {\bf s} = {\bf y} \bmod q ] = q^{-m}$. This implies the following upper bound:
\[
\Pr_{\bf A} \left[\lambda_1(\Lambda({\bf A})) \leq \rho \right] \leq \left|\mathcal{B}({\bf 0},\rho) \cap \mZ^m\right| \cdot q^{n-m}.
\]
By enlarging the radius by $\sqrt{m/2}$, the zero-centered ball contains all the unit hypercubes centered in the points of $\mathcal{B}({\bf 0},\rho) \cap \mZ^m$. This gives that 
\[
|\mathcal{B}_m({\bf 0}, \rho) \cap \mZ^m| \ \leq  \ \mbox{vol}(\mathcal{B}_m({\bf 0}, \rho +\sqrt{m/2}))
\ \leq \  (\rho +\sqrt{m/2})^m \cdot \left(\frac{2\pi e}{m}\right)^{m/2},
\]
from which the proof can be completed.

We now focus on the~$\lambda_1(\Lambda^{\perp}({\bf A}))$ lattice. For~$\rho < q$, we have:
\begin{eqnarray*}
\Pr_{\bf A} \left[\lambda_1(\Lambda^{\perp}({\bf A})) \leq \rho \right] & = & \Pr_{\bf A} \left[\exists {\bf x} \in \mathcal{B}({\bf 0},\rho) \cap \mZ^m \setminus \{{\bf 0}\}: {\bf x}^T \cdot {\bf A} = {\bf 0} \bmod q \right] \\
&\leq& 
\sum_{{\bf x} \in \mathcal{B}({\bf 0},\rho) \cap \mZ^m \setminus \{\bf 0\}}  \Pr_{\bf A} \left[{\bf x}^T \cdot {\bf A} = {\bf 0} \bmod q \right].
\end{eqnarray*}
Now, as~$q$ is prime and~${\bf x}$ is non-zero modulo~$q$, we have that $ \Pr_{\bf A} [{\bf x}^T \cdot {\bf A} = {\bf 0} \bmod q] = q^{-n}$. This gives:
\[
\Pr_{\bf A} \left[\lambda_1(\Lambda^{\perp}({\bf A})) \leq \rho \right] \leq  \left|\mathcal{B}({\bf 0},\rho) \cap \mZ^m\right| \cdot q^{-n}.
\]
We can conclude by using the above bound on $|\mathcal{B}_m({\bf 0}, \rho) \cap \mZ^m|$.
\end{proof}

\begin{comment}
r^{m} (2pie/m)^(m/2) (1+sqrt(m/2)/r)^m q^(n-m) \leq 2^(-m)
r +sqrt(m/2) <= q^(1-n/m) * sqrt(m/8pie)
\end{comment}

\subsection{Other lattice invariants}

Other lattice invariants may be considered, depending on the context. An important example is the covering radius~$\rho(L)$, 
defined as the smallest~$r$ such that the balls of radius~$r$ centered in the points of~$L$ cover~$\mbox{span}_\mR(L)$.
The successive minima, defined below, are also frequently considered. The measure the lattice `discreteness' in all possible dimensions
rather than only one, as does the lattice minimum. 

\begin{definition}
Let~$L \subset \mR^m$ be a lattice of dimension~$n$. For any~$i \leq n$, we define the~$i$-th minimum of~$L$ by:
\[
\lambda_i(L) = \min \left( r: \dim(\mbox{span}_\mR( L \cap \mathcal{B}_m ({\bf 0}, r))) \geq i \right). 
\]
\end{definition}

Note that the minima are always reached. 

\begin{lemma}
For any lattice~$L \subset \mR^m$ of dimension~$n$, there exist~${\bf s}_1, \ldots, {\bf s}_n \in L$ linearly independent such that~$\|{\bf s}_i\| = \lambda_i(L)$, for all~$i \leq n$. 
\end{lemma}

We stress that the ${\bf s}_i$'s may not form a basis of the lattice~$L$. Indeed, there exist lattices for which no basis reaches 
the minima. 

We have the following transference bounds. The lower bound can be proved using elementary manipulations of bases, 
whereas the upper bound was proven by Banaszcyk in~\cite{Bana93}. 

\begin{theorem}
Let~$L \subset \mR^m$ be a lattice of dimension~$n$, and $\widehat{L}$ be its dual. We have
\[ 
1 \leq \lambda_1(L) \cdot  \lambda_n(\widehat{L}) \leq n. 
\]
\end{theorem}

\subsection{Lattice Gaussians}

\textcolor{red}{
Damien: What do we need here? Only def and tail bound, and a statement for Klein/GPV? \\
Which normalization? I tend to prefer Vadim's, for convenience \\
A proof of Klein/GPV? The definition of the smoothing parameter?
}

\textcolor{red}{
Alice: No discrete Gaussian needed for me
}

\subsection{Selected hard lattice problems}

Many computational problems on lattices can be defined and studied. We refer to~\cite{NSD15} for a partial landscape. 
Here we will introduce just a few problems, which play a more direct role in lattice-based cryptography.

\begin{definition}
Let~$\gamma \geq 1$. 
The bounded distance decoding problem (BDD) is as follows: given as input a lattice~$L$ and a vector~${\bf t} \in \mbox{span}_\mR(L)$ such that~$\mbox{dist}({\bf t}, L) \leq \lambda_1(L) / \gamma$, the goal is to find~${\bf b} \in L$ such that~$\|{\bf b} - {\bf t}\| \leq \lambda_1(L) / \gamma$. 
\end{definition}

In the definition above, the distance $\mbox{dist}({\bf t}, L)$ from~${\bf t}$ to~$L$ is the minimum over all~${\bf b}$'s in~$L$ of the distance between~${\bf t}$ and~${\bf b}$. This minimum is well-defined, by discreteness of the lattice.
Note that when~$\gamma > 2$, there exists only one solution to the problem. We stress that some references use different normalizations: 
the input may be promised to satisfy $\mbox{dist}({\bf t}, L) \leq \lambda_1(L) / (2 \cdot \gamma)$. 

\begin{definition}
Let~$\gamma \geq 1$. 
The shortest vector problem (SVP) is as follows: given as input a lattice~$L$, the goal is to find~${\bf s} \in L$ such that~$0 < \|{\bf s}\| \leq \gamma \cdot \lambda_1(L)$. 
\end{definition}

\begin{definition}
Let~$\gamma \geq 1$. 
The shortest independent vectors problem (SIVP) is as follows: given as input a lattice~$L$ of dimension~$n$, the goal is to 
find~${\bf s}_1,\ldots,{\bf s}_n \in L$ such that~$\max_i \|{\bf s}_i \|  \leq \gamma \cdot \lambda_n(L)$. 
\end{definition}

These three problems become no easier when the parameter~$\gamma$ decreases or the lattice dimension~$n$ increases. 
The parameter~$\gamma$ is typically referred to as the \emph{approximation factor}. 




%!TEX root = main.tex
\section{LWE, SIS, and cryptographic applications}
\label{se:LBC}

Dans un deuxième temps, Adeline Roux-Langlois définira les problèmes
{\emph Learning With Errors} (LWE)
et {\emph Small Integers Solution} (SIS), qui sont au c{\oe}ur de la
cryptographie
reposant sur les réseaux euclidiens. Ensuite, elle décrira les chiffrements
Primal-Regev et Dual-Regev et prouvera leur sécurité à partir du
problème LWE. Enfin, elle décrira deux familles de signatures reposant
sur les réseaux euclidiens, dont sont issus les candidats Dilithium et Falcon
au processus de standardisation du NIST.
%!TEX root = main.tex
\section{Algebraic lattices and cryptographic applications}

%La troisième partie du cours sera effectuée par Alice Pellet-Mary. Après des
%éléments de théorie algébrique des nombres, elle définira les variantes
%algébriques de LWE et SIS, ainsi que le problème NTRU,
%et présentera leurs liens avec les réseaux algébriques. Elle construira alors
%des primitives cryptographiques élémentaires à partir de ces problèmes.
%Enfin, elle abordera des aspects de cryptanalyse et présentera des attaques
%exploitant la structure algébrique sous-jacente à ces problèmes.

In this section, we will discuss about structured lattices, and the corresponding structured variants of the LWE and SIS problem. We will also see a new algorithmic problem, called the NTRU problem.\footnote{This problem was originally defined as a structured lattice problem, and later generalized to a less structured (and even unstructured) version of itself.}

One of the main reason for introducing structure in lattices and lattice problems is to improve the efficiency of the cryptographic schemes based on lattices. Let's be clear here, schemes based on unstructured lattice problems such as (plain) LWE and SIS are \emph{not} inefficient, they are already somewhat efficient. But adding structure to the lattice can make them even more efficient, which makes everyone happier.
In order to see why the structure can make scheme more efficient, imagine that you have a lattice which has one basis $\vec B$ which is a circulant matrix, i.e., it is of the form
\begin{align*}
\vec B = \begin{pmatrix}
b_{1} & b_{2} & \dots & b_n \\
b_n & b_1 & \dots & b_{n-1} \\
\vdots & \ddots & \ddots & \vdots \\
b_{2} & b_3 & \dots & b_1
\end{pmatrix}.
\end{align*}
Then, storing this basis of you lattice only requires storing $n$ coefficients, instead of $n^2$ for an unstructured matrix. Also, multiplying this matrix with a vector can be done in quasi-linear time $O(n \cdot \poly( \log n) )$, instead of quadratic time $O(n^2)$.
Let's remark here that a structured lattice only need to have a few structured bases. Not all bases need to have this special structure (and in general, not all bases will have this visible structure indeed). But as long as we can compute a nicely structured basis, then we can enjoy the benefits of the structure on the storage space, and the running time of basic operations (like matrix-vector multiplication).

On the negative side, adding structure to the lattices may render the algorithmic problems based on these lattices easier to solve. From a cryptographic perspective, this would mean making the cryptographic constructions based on these problems weaker against attackers. It is then important to choose wisely the structure one adds to the lattices, and to study the impact of this structure on potential attacks.
There have been examples of poorly chosen structure, leading to attacks on the structured schemes \cite{BLA}.
%aller voir la thèse de Carl : https://www.esat.kuleuven.be/cosic/publications/thesis-399.pdf
% (une attaque de Gentry sur NTRU avec les sous corps de X^n-1, et ses propres articles sur le sujet)
On the other hand, we also have examples of good choice of structure,\footnote{Understand here that they are good \emph{so far}. They might become ``poorly chosen'' in a few months/years if someone manages to exploit the structure to find better attacks.} for which the best known attack is the same as the best known attack against unstructured lattices (i.e., for the moment, the only way we know to solve some algorithmic problems in these lattice is to forget about the structure, and use standard attacks against unstructured lattice problems).

The structure we will consider in this section is a special case of what is called \emph{modules over the ring of integers of a number field}. In order to make this name shorter, we will simply call these \emph{module lattices} in the rest of these lecture notes, as is commonly done in the cryptographic community. We will see in details what those objects are in Section~\ref{sec:number-theory} below. Module lattices are, by far, the most widely used structured lattices in lattice-based cryptography at the moment. If chosen appropriately, they may fall in the category of ``good structured lattices'' defined above, in the sens that, at the moment, we don't know how to solve standard algorithmic problems in these lattices (such at SVP) more efficiently than in non structured lattices.
Before moving to the formal definition of module lattices, let us mention the existence of other kinds of structured lattices (which won't be covered here), such as ideals over a maximal order of a cyclic algebra~\cite{CLWE}.


\subsection{Background on Number Theory}
\label{sec:number-theory}
In this section, we review some definitions and results of number theory that will be useful for the rest of the course. The last subsections are a little bit more advanced, and will be used only for the cryptanalysis part (Section~\ref{sec:cryptanalysis}). In this section, we will state most of the results without their proofs. Most of the proofs for the results about number theory can be found in Marcus' book~\ref{Marcus}. For the algorithmic results, or the results about modules, most of them are in Cohen's books~\ref{Cohen1, Cohen2}.
\subsubsection{Number fields}

A number field $K$ is a field, containing $\QQ$, and which has \emph{finite dimension} when seen as a vector space over $\QQ$.\footnote{Any field containing $\QQ$ can be seen as a vector space over $\QQ$, but not all of them have finite dimension as a $\QQ$-vector space. For instance, $\RR$ and $\CC$ are $\QQ$-vector spaces of infinite dimension.} The dimension of $K$ over $\QQ$ is called the \emph{degree} of the number field $K$, and denoted by $[K:\QQ]$. In these lecture notes, by convention, we will use the letter $d$ to denote the degree of a number field (when there is no conflict of notation).

There are multiple ways to represent number fields. The most frequent ones are seeing them as subsets of $\CC$, or as polynomial rings. In these lecture notes, we will adopt the polynomial ring representation, which will be more convenient.

\begin{lemma}
\label{le:nb-field-eq-def}
Let $P \in \QQ[X]$ be an irreducible polynomial of degree $d$. Then the polynomial ring $\QQ[X]/P(X)$ is a number field of degree $d$.
Conversely, if $K$ is a number field of degree $d$, then there exists an irreducible polynomial $P$ of degree $d$ such that $K$ is isomorphic (as a field) to $\QQ[X]/P(X)$.
\end{lemma}

In the rest of these lecture notes, let us fix an irreducible monic polynomial $P$ of degree $d$, and define $K = \QQ[X]/P(X)$. Note that for any polynomial $f(X) \in \QQ[X]$, there is a unique polynomial $g(X) \in \QQ[X]$ with degree $< d$ and such that $f(X) = g(X) \bmod P$. We will represent elements of $K$ by the unique representative of their class which has degree $< d$. In other words, we see $K$ as the set $\{\sum_{i = 0}^{d-1} x_i X^i\,|\, x_i \in \QQ\}$ (where operations are performed modulo $P$). With this representation, we see that $\QQ$ is indeed contained in $K$, as the set of all polynomials of degree $0$.

\begin{example}\ 
\begin{enumerate}
\item Taking $P = 1$ leads to $K = \QQ$, the simplest of all number field. This is not a very interesting one, but keeping it in mind can be useful sometimes, to gain some intuition about what is happening.
\item A very common choice of polynomial $P$ in cryptography is $P = X^d+1$, where $d$ is a power-of-two. This polynomial is a cyclotomic polynomial (its roots are the primitive $(2d)$-th roots of unity), and the field $K$ associated to it is called a cyclotomic field. Cyclotomic fields have been extensively studied and they enjoy nice properties, that can be useful in cryptography, both for constructing cryptographic primitives, and for cryptanalysis. Most of the schemes using structured lattices today rely on power-of-two cyclotomic fields.
\item An other famous choice of polynomial is $P = X^p-X-1$, with $p$ a prime number. These polynomials are used in the NTRU Prime construction~\cite{NTRUPrime}, and we will call the associated field $K$ an NTRU Prime field. An interesting property of these fields is that they have no subfields, i.e., there exist no field $L$ such that $\QQ \varsubsetneq L \varsubsetneq K$.
\end{enumerate}
\end{example}

\paragraph{Ring of integers.} A number field $K$ contains a special sub-ring, called the \emph{ring of integers} of the number field (also called sometimes the maximal order of the number field). It is defined as follows.

\begin{definition}
\label{def:ring-of-integers}
Let $K$ be a number field. The ring of integers $\O_K$ of $K$ is the subset of $K$ formed by all the elements that are integral over $\ZZ$, i.e., all elements $x$ of $K$ such that there exists a monic irreducible polynomial $P_x \in \ZZ[X]$ satisfying $P_x(x) = 0$.
\end{definition}

It can be proven that this set $\O_K$ is indeed a ring (but not a field). Moreover, $\O_K$ is a $\ZZ$-module of rank $d$, i.e., there exists $d$ elements $\alpha_1, \dots, \alpha_d \in \O_K$ such that any element $x \in \O_K$ can be uniquely written as $x = \sum_i x_i \alpha_i$, with $x_i \in \ZZ$. Such elements $(\alpha_1, \dots, \alpha_d)$ also form a $\QQ$-basis of the $\QQ$-vector space $K$. They are called an \emph{integral basis} of $K$.

\begin{example}\ 
\begin{enumerate}
\item If $K = \QQ$, then $\O_K = \ZZ$, and $(1)$ is an integral basis of $K$.
\item If $K = \QQ[X]/(X^d+1)$ with $d$ a power-of-two, then $\O_K = \ZZ[X]/(X^d+1)$. Moreover, $(1,X, \dots, X^{d-1})$ is an integral basis of $K$.
\item In full generality, if $K = \QQ[X]/P(X)$ with $P$ irreducible, then $\ZZ[X]/P(X) \subseteq \O_K$, but we might not have equality.
\end{enumerate}
\end{example}

One of the consequence of the existence of integral bases is that elements of $K$ can be seen as fractions, with numerators in $\O_K$ and denominators in $\ZZ$ (by default, the denominators would have been in $\O_K$ too), as stated in the following lemma.
\begin{lemma}
Any $x \in K$ can be written as $x = \frac{z}{n}$ with $z \in \O_K$ and $n \in \ZZ_{>0}$.
\end{lemma}
\begin{proof}
Take an integral basis $(\alpha_1, \dots, \alpha_d)$ of $K$ and let $x_i \in \QQ$ be such that $x = \sum_i x_i \alpha_i$. Then taking $n$ to be the least common multiple of all the denominators of the $x_i$'s gives the desired result.
\end{proof}

\paragraph{Complex embeddings.} There exists always $d$ fields homomorphisms from $K$ to $\CC$,\footnote{We impose here that the field homomorphism sends $1$ to $1$, which avoids the trivial function sending every element of $K$ to $0$.} called the \emph{complex embeddings} of $K$. Note that field homomorphisms are always injective, and that they fix $\QQ$ point-wise. Let us write $\sigma_1, \dots, \sigma_d$ these complex embeddings. Some of them may have their image lying fully in $\RR \subset \CC$. We call those real embeddings, and we will write $d_\RR$ their number. If $\sigma$ is a complex embedding whose image is not fully contained in $\RR$, then $\overline{\sigma}$ (defined by $\overline{\sigma}(x) = \overline{\sigma(x)}$) is also a complex embedding. We say that $\sigma$ and $\overline{\sigma}$ are conjugate. Let $d_{\CC}$ be the number of such pairs of conjugate complex embeddings. Then we have $d = d_\RR + 2 d_\CC$. In the rest of this course, we order the complex embeddings such that $\sigma_1, \dots, \sigma_{d_\RR}$ are real embeddings, and $(\sigma_i, \sigma_{i+d_\CC})$ are pairs of conjugate embeddings for $d_\RR < i \leq d_\RR+d_\CC$.

\begin{lemma}
Let $K = \QQ[X]/P(X)$ with $P$ irreducible, and let $\alpha_1, \dots, \alpha_d$ be the $d$ roots of $P$ in~$\CC$. Then, the complex embeddings of $K$ are exactly the functions
\begin{align*}
\sigma_i: \ \ \ K & \rightarrow \CC \\
                f(X) & \mapsto f(\alpha_i)
\end{align*}
for $i \in \{1, \dots, d\}$.\footnote{Here, $f$ is a polynomial in $\QQ[X]$ of degree $< d$. We use the fact that $P(\alpha_i) = 0$, which implies that reduction modulo $P$ does not change the output of the function, and so the function is well defined over $\QQ[X]/P(X)$.}
\end{lemma}

\begin{proof}
First of all, one can check that the functions $\sigma_i$ defined in the lemma are indeed field homomorphisms, with image in $\CC$, as desired. Moreover, since $P$ is irreducible, then all the roots $\alpha_i$ are distinct, and so the functions $\sigma_i$ are distinct too (since they have distinct evaluations at~$X$).

If we admit that there are only $d$ complex embeddings for $K$, then we are done. However, since the proof is instructive, we will also prove that any complex embedding of $K$ is one of the~$\sigma_i$. This will also prove that there are exactly $d$ complex embeddings (which we have admitted above).

Let $\sigma$ be a complex embedding. Since it is a field homomorphism and since it fixes $\QQ$, then the knowledge of $\sigma(X)$ uniquely determines $\sigma$. Indeed, if $\beta = \sigma(X) \in \CC$, then for any $f(X) = \sum_i f_i X^i \in K$ (with $f_i \in \QQ$), we have $\sigma(\sum_i f_i X^i) = \sum_i f_i \beta^i = f(\beta)$.
Moreover, since $P(X) = 0$, then we should have $P(\beta) = P(\sigma(X)) = \sigma(P(X)) = \sigma(0) = 0$. So $\beta$ is a complex root of $P$, i.e., it is one of the $\alpha_i$'s. This proves that $\sigma$ is equal to one of the $\sigma_i$'s.
\end{proof}

\begin{example}
If $K = \QQ[X]/(X^d+1)$ with $d$ a power-of-two, then $d_\RR = 0$, $d_\CC = d/2$ and the $d$ complex embeddings of $K$ are obtained by sending $X$ to all the primitive $(2d)$-th roots of unity in $\CC$ (there are $\varphi(2d) = d$ such primitive roots, which are exactly the roots of $X^d+1$).
\end{example}

\paragraph{Trace and norm.} The trace $\Tr$ and algebraic norm $\N$ are functions from $K$ to $\CC$ defined as follows. For any $x \in K$
\begin{align*}
\Tr(x) &= \sum_{i = 1}^d \sigma_i(x) \\
\N(x) &= \prod_{i = 1}^d \sigma_i(x)
\end{align*}
\begin{lemma}
For any $x \in K$, the trace and the norm of $x$ are in $\QQ$ (and not only in $\CC$). Moreover, if $x \in \O_K$, then $\Tr(x)$ and $\N(x)$ are in $\ZZ$.
\end{lemma}

\begin{example}
If $K = \QQ[X]/(X^2+1)$ and $x = a+bX$, then
\begin{align*}
\Tr(x) &= (a+ib) + (a-ib) = 2a \\
\N(x) &= (a+ib) \cdot (a-ib) = a^2+b^2
\end{align*}
\end{example}

\begin{lemma}
The trace is additive and the algebraic norm is multiplicative. I.e., for any $x, y \in K$, it holds that
\begin{align*}
\Tr(x+y) &= \Tr(x) + \Tr(y) \\
\N(xy) &= \N(x) \cdot \N(y)
\end{align*}
\end{lemma}

\subsubsection{Ideals} For any commutative ring $R$, an \emph{ideal} $I$ of $R$ is a subset of which is an additive group, and is stable by multiplication by an element of $R$, i.e., for all $x in I$ and $\alpha \in R$, it holds that $\alpha x \in I$. In our context with number fields $K$ and ring of integers $\O_K$, we will consider ideals of $\O_K$, as well as a generalization of those.

\begin{definition}
Let $K$ be a number field and $\O_K$ its ring of integers.
\begin{itemize}
\item An \emph{integral ideal} of $K$ is an ideal of $\O_K$, for the definition given above. By convention, we will use Gothic letters $\mathfrak{a}, \mathfrak{b}, \cdots$ to denote integral ideals.
\item A \emph{fractional ideal} of $K$ is a subset $I \subset K$ of the form $x \cdot \mathfrak{a} := \{x \cdot a\,|\, a \in \mathfrak{a}\}$ for some $x \in K^*$ and $\mathfrak{a}$ an integral ideal of $K$. By convention, we will use capital letters $I, J, \cdots$ to denote fractional ideals.
\end{itemize}
\end{definition}

Note that $K$ is a ring, and so one can define the ideals of $K$ when seen as a ring. However, since $K$ is also a field, it only has two ideals: $K$ and $\{0\}$. Those are not very interesting, and we will never consider ideals of $K$ in these lecture notes. Instead, we will sometimes use the terminology ``ideal of $K$'' to refer to fractional ideals of $K$ (and we will always specify ``\emph{integral} ideal if $K$'' when they are integral).

\begin{example}
For $K = \QQ$ and $\O_K = \ZZ$, the set $\mathfrak{a} = \{3x\,|\, x \in \ZZ\}$ is an integral ideal of $K$. And the set $I = \{5x/7 \,|\, x \in \ZZ\}$ is a fractional ideal of $K$.
\end{example}

\begin{definition}
A fractional ideal $I$ of $K$ is said to be \emph{principal} if it is of the form $\alpha \O_K := \{ \alpha \cdot x\,|\, x \in \O_K\}$ for some $\alpha \in K$. In this case, $\alpha$ is called a \emph{generator} of $I$ (it is usually not unique).
\end{definition}

\begin{example}
The two ideals from the previous example are principal, with generators $3$ and $5/7$ respectively. This is not a coincidence, all ideals in $\ZZ$ are principal, we say that $\ZZ$ is a principal ideal domain (PID).
In general, the ring $\O_K$ may not be principal, meaning that there exists ideals that are not generated by a single element.
\end{example}

\begin{lemma}
\label{lemma:two-elm-rep}
Any fractional ideal $I$ can be generated by at most $2$ elements of $K$, i.e., there exists $\alpha, \beta \in K$ such that $I = \{ \alpha \cdot x + \beta \cdot y \,|\, x, y \in \O_K\}$.
\end{lemma}

\paragraph{Arithmetic over ideals.} In a number field $K$, we will want to do arithmetic over the fractional ideals. We will see below that, while the ring $\O_K$ might not have the same nice arithmetic properties as the ring $\ZZ$ (for instance, we usually do not have euclidean division in $\O_K$, and sometimes not even unique factorization), everything works better if we consider ideals of $\O_K$, rather than elements of $\O_K$.

\begin{definition}
Let $I$ and $J$ be two fractional ideals. We define the addition and multiplication of $I$ and $J$ as
\begin{align*}
I + J &= \{x + y \,|\, x \in I, y \in J\} \\
I \cdot J &= \{ \sum_{i = 1}^r x_i \cdot y_i \,|\, r > 0, x_i \in I, y_i \in J\}.
\end{align*}
\end{definition}

\begin{example}
When $K = \QQ$ and $\O_K = \ZZ$, we have $6\ZZ + 8 \ZZ = 2\ZZ$, and $2 \ZZ \cdot 3 \ZZ = 6 \ZZ$.

When the ideals are integral, the addition of two ideals corresponds to their gcd, i.e., if $x, y \in \ZZ$, then $x \ZZ + y \ZZ = \gcd(x,y) \ZZ$. More generally, in an arbitrary number field $K$, the sum of two integral ideals $\mathfrak{a}$ and $\mathfrak{b}$ is the smallest (for inclusion) integral ideal containing both $\mathfrak{a}$ and $\mathfrak{b}$.

In any number field $K$, if $I = \alpha \O_K$ and $J = \beta \O_K$ are principal, then we always have $I\cdot J = (\alpha \cdot \beta)\O_K$.
\end{example}

\begin{lemma}
The sum and product of two fractional ideals is still a fractional ideal. Moreover, the set of non-zero fractional ideal forms a multiplicative group with neutral element $\O_K$, i.e., for any non-zero fractional ideal $I$, there exists a non-zero fractional ideal $J$ such that $I \cdot J = \O_K$. This ideal $J$ is called the inverse of $I$, and we write it $I^{-1}$.
\end{lemma}

There exists also a notion of prime ideals in $K$.

\begin{definition}
A \emph{prime ideal} $\mathfrak{p}$ of $K$ is an \emph{integral} ideal such that for any two elements $\alpha, \beta \in \O_K$, if $\alpha \cdot \beta \in \mathfrak{p}$, then it must be that either $\alpha \in \mathfrak{p}$ or $\beta \in \mathfrak{p}$.
We will also assume in these lecture notes that a prime ideal is non-zero.\footnote{The convention usually consists in accepting $\{0\}$ as a prime ideal, and writing everywhere ``non-zero prime ideal''. We exclude $\{0\}$ from the beginning to avoid unnecessary heavy notations.}
\end{definition}

\begin{example}
Over $\ZZ$, the prime ideals are exactly the ideals $p \cdot \ZZ$, where $p$ is a prime integer.

In $K = \QQ[X]/(X^2+1)$ and $\O_K = \ZZ[X]/(X^2+1)$, the principal ideal $\mathfrak{a} := (1+X) \cdot \O_K$ is prime. To see this, one can first check that an element $a + bX \in \O_K$ is divisible by $1+X$ if and only if $a = b \bmod 2$. Hence, $\mathfrak{a} = \{a + bX \,|\, a, b \in \ZZ, a = b \bmod 2\}$. Let $x = a+bX$ and $y = c+dX$ be in $\O_K$ and such that $x \cdot y \in \mathfrak{a}$. We have $x \cdot y = (ac-bd) + (ad+bc)X$, and so by the property above, we know that $ac-bd = ad+bc \bmod 2$. One can check that this equation implies that either $a = b \bmod 2$ or $c = d \bmod 2$ (for instance, by testing all the $2^4$ possible values of $a,b,c$ and $d$ modulo $2$). Hence, either $x$ of $y$ is in $\mathfrak{a}$, as desired.
\end{example}

\begin{lemma}
Any \emph{integral} ideal of $K$ can be factored in a unique way (up to permutation of the terms) as a product of prime ideals. In other words, for any integral ideal $\mathfrak{a}$, there exists a unique sequence of non-negative integers $(n_\mathfrak{p})_{\mathfrak{p}}$, ranging over all prime ideals and non-zero only for a finite number of prime ideals, such that $\mathfrak{a} = \prod_{\mathfrak{p}} \mathfrak{p}^{n_\mathfrak{p}}$.

\noindent For fractional ideals $I$, there is also a unique factorization, but exponents may be negative, i.e., $I = \prod_{\mathfrak{p}} \mathfrak{p}^{x_\mathfrak{p}}$ where $x_\mathfrak{p} \in \ZZ$ is zero for almost all primes.
\end{lemma}

\begin{example}
In $K = \QQ[X]/(X^2+1)$, we have $2 \O_K = \mathfrak{p}^2$, where $\mathfrak{p} = (1+X) \cdot \O_K$ is the prime ideal from the previous example. To see this, note that $(1+X)^2 = 2X \bmod X^2+1$, so $\mathfrak{p}^2 = (2X) \cdot \O_K$. But since $X$ is inertible in $\O_K$ (with inverse $-X$), the ideals $(2X) \cdot \O_K$ and $2 \O_K$ are equal.
\end{example}

\begin{lemma}
An integral ideal $\mathfrak{a}$ is prime if and only if the quotient space $\O_K/\mathfrak{a}$ has no divisors of zero. In our case, this is equivalent to $\O_K/\mathfrak{a}$ being a field (because it is a finite, commutative ring).
\end{lemma}

\paragraph{Algebraic norm.} The algebraic norm of an non-zero integral ideal $\mathfrak{a}$ is $\N(a) = |\O_K/\mathfrak{a}|$, i.e., it is the index of $\mathfrak{a}$ in $\O_K$. This is an positive integer. For a non-zero fractional ideal $I = x \cdot \mathfrak{a}$, with $x \in K$ and $\mathfrak{a}$ an integral ideal, the algebraic norm is defined as $\N(I) = |\N(x)| \cdot |\N(\mathfrak{a})|$. This is a positive rational number, and it does not depend on the choice of $x$ and $\mathfrak{a}$.

\begin{properties}
\label{prop:norm-ideal}
\begin{enumerate}
\item if $I = \alpha \O_K$ is principal, then $\N(I) = |\N(\alpha)|$;
\item the norm is multiplicative, i.e., for $I$ and $J$ fractional ideals, we have $\N(I \cdot J) = \N(I) \cdot \N(J)$;
\item if $\mathfrak{p}$ is a prime ideal, then $\N(\mathfrak{p}) = p^k$ for some $k \geq 1$ and $p$ a prime integer;
\item for an integral ideal $\mathfrak{a}$, we have $\N(\mathfrak{a}) \in \mathfrak{a}$ (when seeing $\N(\mathfrak{a})$ as an element of $K$ via the inclusion $\ZZ \subset K$).
\end{enumerate}
\item if $I \subseteq J$, then $I = \mathfrak{a} \cdot J$ with $\mathfrak{a}$ a fractional ideal. This implies in particular that $\N(J) |\N(I)$ (because $\N(\mathfrak{a})$ is an integer).
\end{properties}


\paragraph{Computational aspects.} $\ZZ$-basis of ideals. The fact that we can go from $2$-elements representation to basis in polynomial time and conversely. The fact that we can add and multiply efficiently, and factor ideals if we can factor integers.

\subsubsection{Modules}
Let $m \geq 1$ be an integer and let $M \subset K^m$. We say that $M$ is a finitely generated $\O_K$-module if and only if there exists vectors $\vec b_1, \cdots, \vec b_t \in K^m$ such that
\[ M = \{ \sum_{i = 1}^t x_i \vec b_i \,|\, x_1, \dots, x_t \in \O_K\}. \]
We say that such elements $(\vec b_1, \dots, \vec b_t)$ are a generating set of $M$ (they are not unique).
In the rest of these lecture notes, we will say that $M$ is a ($\O_K$-)module, instead of the longer (but more accurate) terminology ``finitely generated $\O_K$-module included in $K^m$ for some $m \geq 1$''.

\begin{lemma}
If $m = 1$, then a $M \subset K$ is a module if and only if it is a fractional ideal of $K$.
\end{lemma}
Ideals are the simplest cases of modules, when the dimension is $1$.

\begin{proof}
Let $I$ be a fractional ideal. From Lemma~\ref{lemma:two-elm-rep}, we know that $I = \{ x_1 \alpha + x_2 \beta \,|\, x_1, x_2 \in \O_K\}$ for some $\alpha, \beta \in K$. This proves that $I$ is a module in $K$ for our definition above.
On the other hand, let $M \subset K$ be a module. By definition, there exists $b_1, \dots, b_t \in K$ such that $M = b_1 \O_K + \dots + b_t \O_K$. In other words, $M$ is the sum of all the principal ideals $b_i \O_K$. We have seen that a sum of ideals is still an ideal, which concludes the proof.
\end{proof}

\begin{definition}
The rank of a module $M$ is the dimension of the $K$-vector space spanned by the elements of $M$ (which is a subspace $\leq K^m$).
\end{definition}

By setting $K = \QQ$ and $\O_K = \ZZ$, one can see that a $\ZZ$-module is simply a lattice. In a sense, modules are generalizations of lattices, to more general rings (here, the ring $\O_K$). There are still some differences between modules and lattices. The first one is that, in lattices, we often care about geometric questions. For this, we need to have a notion of ``size'' for the vectors of our lattice, which is obtained by taking their euclidean norm (or infinity norm, or any other norm). When replacing $\QQ$ by $K$, it is nor clear anymore what the size of an element of $K$ is, let alone a vector of $K^m$. We will discuss this in more details in Sections~\ref{sec:embeddings} and~\ref{sec:id-mod-lat}.

Another difference between modules and lattices is that lattices always have bases, but modules may not. So far, we have defined a module by a generating set, but it might not be possible to extract a set of linearly independent basis vectors from this generating set. There exists however a notion of pseudo-basis, which can be used somehow analogously to the notion of basis (up to some more technicalities).

\begin{lemma}
Let $M$ be a module of rank $r$ included in $K^m$. There exists $r$ vectors $\vec b_1, \dots, \vec b_r \in K^m$ that are $K$-linearly independent, and $r$ fractional ideals $I_1, \dots, I_r$ such that
\[ M = \sum_{i = 1}^r I_i \cdot \vec b_i := \{ \sum_i x_i \vec b_i \,|\, x_i \in I_i \text{ for all }i\}.\]
The list of pairs $\big((\vec b_i, I_i)\big)_{1 \leq i \leq r}$ is called a \emph{pseudo-basis} of the module $M$.
\end{lemma}

\begin{definition}
When $M$ admits a pseudo-basis $\big((\vec b_i, I_i)\big)_{1 \leq i \leq r}$ with all ideals $I_i$ equal to $\O_K$, we say that $M$ is free.
\end{definition}
When $I_i = \O_K$ for all $i$'s, then the $(\vec b_i)_i$ for a basis of $M$, for the usual definition of a basis (any element of $M$ can be uniquely written as a linear combination over $\O_K$ of those vectors). Observe that, in our definition, the module $M$ is free if \emph{at least one} of its pseudo-basis is a basis, but we do not require that all pseudo-bases have $I_i = \O_K$ (this would not be possible anyway).

\subsubsection{Embeddings}
\label{sec:embeddings}
We have seen above that it would be convenient to be able to define a notion of ``size'' of an element of $K$ (which we would then extend to a notion of ``size'' for the vectors of $K^m$). A usual way to obtain this is to embed $K$ into $\CC^d$, and then use, e.g., the hermitian norm over $\CC$, or the infinity norm. There are two frequently used embeddings of $K$ into $\CC^d$, leading to two different notions of size for a elements of $K$.

\begin{definition}
Let $K = \QQ[X]/P(X)$ be a number field. The coefficient embedding of $K$ is defined as
\begin{align*}
\coeff: \hspace{4mm} K & \rightarrow \QQ^d \\
\sum_{i=0}^{d-1} a_i X^i \mapsto	(a_0, \dots, a_{d-1}),
\end{align*}
and the canonical embedding (or Minkowski's embedding) is defined as
\begin{align*}
\mink: \hspace{4mm} K & \rightarrow \CC^d \\
x \mapsto	(\sigma_1(x), \dots, \sigma_d(x)),
\end{align*}
where $\sigma_1, \dots, \sigma_d$ are the complex embeddings of $K$, defined above.
\end{definition}

Note that the canonical embedding depends on our choice of polynomial $P$. If we use a different polynomial $P'$ defining an isomorphic number field $K'$, then the coefficient embedding can change, whereas the canonical embedding will stay the same (up to permutation of the coordinates, which are not uniquely ordered).

\begin{example}
In $K = \QQ[X]/(X^2+1)$, we have $\coeff(1+X) = (1,1) \in \QQ^2$ and $\mink(1+X) = (1+i, 1-i) \in \CC^2$.
\end{example}

One of the advantage of the coefficient embedding is that it lives in $\QQ^d$ instead of $\CC^d$, which makes it easier to manipulate elements on a computer. On the other hand, the canonical embedding is more intrinsic to the field, and enjoy nicer mathematical properties, which can be useful for analysis.

Let's also remark that the image of $\mink(K)$ lives in a $\RR$-vector subspace of $\CC^d$, with dimension $d$ over $\RR$ (when $\CC^d$ has dimension $2d$ over $\RR$). This is the space
\[\{x \in \CC^d \,|\, x_1, \dots, x_{d_\RR} \in \RR, \text{ and } x_{d_\RR+1} = \overline{x_{d_\RR+d_\CC+1}}, \dots, x_{d_\RR+d_\CC} = \overline{x_d} \}.\]

Now that we have functions to embed $K$ into $\CC^d$, we can define the ``size'' of an element of $K$ as the euclidean norm of its coefficient embedding, or of its canonical embedding. This usually gives different notion of sizes, but in the specific case of power-of-two cyclotomic field, one can prove that both notions are strongly linked.

\begin{proposition}
Let $K = \QQ[X]/X^d+1$ with $d$ a power-of-two. Then for all $x \in K$, we have $\|\mink(x)\| = \sqrt{d} \|\coeff(x)\|$.
\end{proposition}
In this specific case, the canonical embedding is a scaled isometry of the coefficient embedding. For other cyclotomic fields, one can also relate the coefficient and canonical embeddings, but those are usually not scaled isometries of one another (see e.g.~\cite{Blanco}).

Finally, let us introduce some convenient notation: for $\vec b = (b_1, \dots, b_m)^T \in K^m$, we will write $\mink(\vec b) = (mink(b_1), \dots, \mink(b_m))^T \in \CC^{dm}$. Similarly, we write $\coeff(\vec b) \in \QQ^{dm}$ for the application of $\coeff$ coordinate-wise.

\paragraph{Some useful properties.} As said above, the canonical embedding enjoys nice mathematical properties. One of them is that it can be related to the algebraic norm.

\begin{lemma}
\label{lemma:AM-GM-inequality}
Let $x \in K$, then $\|\mink(x)\| \geq \sqrt{d} \cdot |\N(x)|^{1/d}$. In particular, for any $x \in \O_K$ non-zero, we have $\|\mink(x)\| \geq \sqrt{d}$.
\end{lemma}

\begin{proof}
Applying the inequality of arithmetic and geometric means to the vector $(|\sigma_1(x)|^2, \dots, |\sigma_d(x)|^2)$ leads
\[ 1/d \cdot \|\mink(x)\|^2 = 1/d \cdot \sum_i |\sigma_i(x)|^2 \geq \Big(\prod_i |\sigma_i(x)|^2\Big)^{1/d} = |\N(x)|^{2/d},\]
which gives the desired inequality. The second part of the statement is obtained by noticing that the algebraic norm of an non-zero element of $\O_K$ is an integer, thus it is $\geq 1$ in absolute value.
\end{proof}

The norm induced over $K$ by the canonical embedding satisfies the triangular inequality (which is a requirement for a norm), but it is also sub-multiplicative.

\begin{lemma}
For any $x,y \in K$, we have
\[\|\mink(x \cdot y)\| \leq \|\mink(x)\|_\infty \cdot \|\mink(y)\| \leq \|\mink(x)\| \cdot \|\mink(y)\|.\]
\end{lemma}

\begin{proof}
This properties follows from the fact that, by definition of $\mink$, the vector $\mink(x \cdot y)$ is the product coordinate-wise of the vectors $\mink(x)$ and $\mink(y)$.
\end{proof}

\subsubsection{Ideal and module lattices}
\label{sec:id-mod-lat}
Ideal and module lattices and ideals and modules to which we add a lattice structure, by using one of the embeddings defined above.

\begin{proposition}
Let $M$ a module in $K^m$ of rank $r$. Then, the sets 
\[\coeff(M) := \{ \coeff(\vec b)\,|\, \vec b\in M\} \subset \QQ^{md}\]
and 
\[\mink(M) := \{ \mink(\vec b)\,|\, \vec b\in M\} \subset \CC^{md}\]
are lattices of rank $dr$.\footnote{Note that we usually define a lattice as a subset of $\RR^t$, but our set $\mink(M)$ lives in $\CC^{md}$. Extending the definition of lattices to subset of $\CC^t$ (using the hermitian norm of $\CC$) is usually pretty harmless, but if one prefers to keep all lattices in $\RR^t$, it suffices to map $\CC^{md}$ to $\RR^{2md}$ isometrically, and consider $\mink(M)$ as a (non full-rank) lattice in this space.}
They are called \textit{module lattices}. If $m = 1$, then $M = I$ is an ideal, and we call $\coeff(I)$ and $\mink(I)$ ideal lattices.
\end{proposition}

Since $\O_K$ is an ideal, the proposition above also applies to it, and we have that $\mink(\O_K)$ is a lattice of rank $d$. Its volume square $\det(\mink(\O_K))^2$ is a quantity called the \textit{discriminant} of the number field $K$, and written $\Delta_K$.

\begin{example}
Let $K = \QQ[X]/(X^2+1)$, then $\mink(\O_K)$ is the rank-$2$ lattice spanned by $\mink(1)$ and $\mink(X)$. Hence, it has a basis which is $\begin{pmatrix}
1 & i \\
1 & -i
\end{pmatrix}$, and its volume is $|-2i| = 2$. So $\Delta_K = \det(\mink(\O_K))^2 = 4$.

More generally, for a power-of-two cyclotomic field $K = \QQ[X]/(X^d+1)$, the discriminant is $\Delta_K = d^d$.
\end{example}

\begin{proposition}
\label{prop:volume-id-mod}
Let $I$ be a fractional ideal, then
\[ \det(\mink(I)) = \N(I) \cdot \sqrt{\Delta_K}.\]
More generally, if $M$ is a module of rank $r$ in $K^r$ with pseudo-basis $((\vec b_i, I_i))_i$, we have
\[\det(\mink(I)) = |\N\big(\det_K(\vec B)\big)| \cdot \prod_i \N(I_i) \cdot \Delta_K^{r/2},\]
where $\vec B$ is the matrix whose column vectors are the $\vec b_i$, and $\det_K$ is the determinant over $\mathcal{M}_r(K)$.\footnote{We could also define an analogous formula for modules that do not have full rank (i.e., modules included in $K^m$ with $m > r$), but this requires defining a conjugation over $K$, which is slightly technical, and won't be needed here.}
\end{proposition}

\begin{example}
Let $K = \QQ[X]/(X^2+1)$ and take the ideal $I = (1+X) \O_K$. The lattice $\mink(I)$ is generated by $\mink(1+X)$ and $\mink((1+X)\cdot X) = \mink(-1+X)$. Hence, it has a basis $\begin{pmatrix}
1+i & -1+i \\ 1-i & -1-i
\end{pmatrix}$ and its volume is the absolute value of the determinant of this matrix, which is $|(1+i)^2 - (1-i)^2| = 4$. We have seen in an example above that $\N(I) = 2$, which gives us that $\det(\mink(I)) = 4 = \N(I) \cdot \sqrt{\Delta_K}$, as stated in Proposition~\ref{prop:volume-id-mod}.
\end{example}

Ideal lattices are lattices, but their algebraic structure constrain their geometry quite a lot. In particular, the first minimum of an ideal lattice cannot be arbitrarily small (in a given field $K$), and its last minimum cannot be arbitrarily large.

\begin{lemma}
For any fractional ideal $I$ of $K$ we have
\begin{align*}
\lambda_1(\mink(I)) &\geq \sqrt{d} \cdot \N(I)^{1/d} \\
\lambda_d(\mink(I)) & \leq \Delta_K^{1/d} \cdot \lambda_1(\mink(I)) \leq \Delta_K^{3/(2d)} \cdot \sqrt{d} \cdot \N(I)^{1/d}.
\end{align*}
\end{lemma}
All the successive minima of the lattice $\mink(I)$ are concentrated in the interval $[1, \Delta_K^{3/(2d)}] \cdot \sqrt{d} \cdot \N(I)^{1/d}$.

\begin{proof}
For the lower bound on $\lambda_1$, let us take $x \in I$, non-zero, and such that $\|\mink(x)\| = \lambda_1(\mink(I))$. By Lemma~\ref{lemma:AM-GM-inequality}, we know that $\|\mink(x)\| \geq \sqrt{d} \cdot |\N(x)|^{1/d}$. It then suffices to prove that for any non-zero $x \in I$, $|\N(x)| \geq \N(I)$. This follows from the fact that $x \O_K$ is included in $I$, and so, by Proposition~\ref{prop:norm-ideal}, $\N(I)$ divides $\N(x \O_K) = |\N(x)|$.

For the upper bound on $\lambda_d$, let us take $x \in I$ non-zero and reaching $\lambda_1(\mink(I))$, and $y_1, \dots, y_d$ be in $\O_K$, $\QQ$-linearly independent and such that $\|\mink(y_i)\|_\infty \leq \lambda_d^{(\infty)}(\mink(\O_K))$ (here, we take the last minimum of $\mink(\O_K)$ for the \emph{infinity} norm instead of the euclidean norm). The elements $z_i = x \cdot y_i$ and in $I$ because $I$ is an ideal. They are $\QQ$-linearly independent, since otherwise we would have a linear dependence relation $\sum_i \alpha_i z_i = 0$ with the $\alpha_i$ in $\QQ$ not all zero. But since $x$ is non-zero, dividing by $x$ (over $K$) would give a linear dependence relation $\sum_i \alpha_i y_i = 0$, which is impossible by assumption on the $y_i$'s.
Hence, we know that 
\[\lambda_d(\mink(I)) \leq \max_i \|\mink(z_i)\| \leq \max_i \|z_i\|_\infty \cdot \|x\| \leq \lambda_d^{(\infty)}(\mink(\O_K)) \cdot \lambda_1(\mink(\O_K)).\]
Using the upper bound $\lambda_d^{(\infty)}(\mink(\O_K)) \leq \Delta_K^{1/d}$ from~\cite[Theorem A.4]{KoenThesis} leads the first inequality. The second inequality is obtained by applying Minkowski's first theorem on the ideal $\mink(I)$, giving us $\lambda_1(\mink(I)) \leq \sqrt{d} \det(\mink(I))^{1/d} = \sqrt{d} \cdot \Delta_K^{1/(2d)} \cdot \N(I)^{1/d}$.
\end{proof}


\begin{itemize}
%\item definition
%\item properties of ideal lattices (lower bound on $\lambda_1$ and upper bound on $\lambda_d$)
\item some comments on cryptanalysis of ideal-SVP vs module-SVP for rank > 1
\end{itemize}

\paragraph{Algorithmic problems.} Since ideal and module lattices are, in particular, lattices, we can restrict the algorithmic problems seen in the first part of the lecture notes to these lattices.

\begin{definition}
Let $K$ be a number field, $r \geq 1$ be an integer and $\gamma \geq 1$ be a real number. We call $r$-module-SVP$_\gamma$ the restriction of the shortest vector problem with approximation factor $\gamma$ to lattices of the form $\mink(M)$ where $M \subset K^r$ is a module of rank $r$.
Similarly, the restriction of SIVP with approximation factor $\gamma$ to the same module lattices is called $r$-module-SIVP$_\gamma$.
When $r = 1$, we rather use the terminology ideal-SVP$_\gamma$ and ideal-SIVP$_\gamma$.
\end{definition}

We could also define restriction of the SVP and SIVP problems to module lattices $\coeff(M)$ in coefficient embedding. However, we will not consider these problems in the rest of these notes, so we do not give them a name. From now on, we will only consider the canonical embedding $\mink$ (unless specified otherwise). Hence, to simplify notations, we may abuse notations and consider the modules $M$ directly as lattices (e.g., writing $\lambda_1(M)$, or $\det(M)$).

Note that when $K = \QQ$ and $r = n$, the problems module-SVP and module-SIVP are just the regular SVP and SIVP problems. Varying the degree $d$ of $K$ and the rank $r$ of the module allows one to obtain various problems, ranging from ideal-S(I)VP at one extremity to standard S(I)VP at the other extremity.

Regarding hardness, the module-SVP problem seems to be essentially as hard as the standard SVP problem. A notable exception is when the modules have rank $1$, i.e. for the ideal-SVP problem. In this extreme case, there exists situations (e.g., when the approximation factor $\gamma$ is very large and the field $K$ is cyclotomic; or if one allows exponentially long pre-computations) where the ideal-SVP$_\gamma$ problem is asymptotically easier to solve than the SVP$_\gamma$ problem (at least with our current knowledge).\footnote{Note that these are only special cases. At the moment, we do not have an algorithm to solve efficiently ideal-SVP in all number fields, for all approximation factors.}
On the other hand, when the rank $r$ of the module becomes larger than $2$, we currently have no algorithms solving $r$-module-SVP significantly faster than the best algorithms for SVP over all lattices.

The conclusion is that, at the moment, it seems that there might be a gap in hardness between ideal-SVP and $r$-module-SVP for $r \geq 2$, and there do not seem to be a gap in hardness between $r$-module-SVP and standard SVP. Note that the discussion here is only based on the currently known best algorithms, which only provide us with upper bounds on the hardness of the problems. This does not allow us to conclude that some problems are equivalent, nor that some problems are strictly harder than others. It only gives us some intuition on what might be happening, but this intuition may be proven false by future cryptanalysis development.




From now on, the results we review will be used only for the cryptanalysis part of the course (Section~\ref{sec:cryptanalysis}).

\subsubsection{Ideals and subfields}
split, ramify, inter

\subsubsection{Units}
log unit lattice, cyclotomic units

%Alice -- Rappels de théorie des nombres -- CM: 1h30 / TD: 1h

%L'objectif de ce cours est de rappeler certains concepts de théorie algorithmique des nombres, qui seront utiles à la fois pour le cours d'isogénies et le cours sur les réseaux algébriques. Nous parlerons principalement de corps de nombres et d'idéaux (décomposition en idéaux premiers, groupe de classes, comportement des idéaux premiers dans des extensions de corps, etc), et de différents problèmes algorithmiques en lien avec ces objets, qui nous intéressent en cryptographie (manipulation des idéaux, calcul du groupe des classes, problème de l'idéal principal, etc).

\subsection{RLWE, module-LWE and NTRU}
\label{sec:rlwe-mlwe-ntru}

\subsection{Cryptographic constructions}
\label{sec:constructions}

\subsection{Cryptanalysis}
\label{sec:cryptanalysis}

%\subsection{Ring and Module variants of LWE, and the NTRU problem}

%Alice -- Réseaux algébriques en cryptographie -- CM: 3h / TD: 1h30

%Ce cours sera consacré au variantes algébriques des problèmes LWE et SIS (définis dans le cours de Adeline Roux-Langlois). En plus des variantes algébriques de LWE et SIS, nous parlerons aussi du problème NTRU, qui n'a pas d'équivalent non-algébrique.
%Le cours sera divisé en trois parties:
%- nous commencerons par définir les variantes algébriques de LWE/SIS et le problème NTRU, et nous parlerons du lien entre ces problèmes et les réseaux (algébriques);
%- nous verrons ensuite comment construire des primitives cryptographiques simples à partir de ces problèmes algorithmiques;
%- enfin, nous parlerons de cryptanalyse et nous présenterons quelques attaques qui exploitent la structure algébrique de ces problèmes.
\input{algorithms.tex}

\bibliographystyle{alpha}
\bibliography{biblio}

\end{document}