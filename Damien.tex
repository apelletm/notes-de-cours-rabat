%!TEX root = main.tex
\section{Background on Lattices}
\label{se:back}



This first section provides some background on Euclidean lattices, with illustrations
preparing the following sections.  


\subsection{Definitions}

We start with elementary definitions pertaining to Euclidean lattices. 
We refer the reader to~\cite{Siegel89} for additional material, including missing proofs.


A Euclidean lattice is the set of all integral linear combinations of a set if linearly independent vectors in a Euclidean space. 
An illustration is provided in Figure~\ref{fig:lattice1}. The lattice is the set of intersection points in the grid. 
The following is a more formal definition.

\begin{definition}
\label{def:lattice1}
Let~$m\geq n >0$. Let~${\bf b}_1, \ldots, {\bf b}_n$ be linearly independent vectors in~$\mR^m$.
The lattice that they span is
\[
 L\left({\bf b}_1, \ldots, {\bf b}_n\right) = \sum_{1 \leq i \leq n} \mZ \cdot {\bf b}_i = \left\{\sum_i x_i \cdot {\bf b}_i: \forall i, x_i \in \mZ\right\}. 
\]
If~$n=m$, then the lattice is said full-rank. If it is contained in~$\mZ^n$, then it is said integral. 
\end{definition}

The prototypical lattice is~$\mZ^n$, which can be written as the set of all integer linear combinations
of the canonical basis vectors~${\bf b}_i = (0^{i-1}, 1, 0^{n-i})^T$ for~$i \in \{1,\ldots,n\}$.  

Definition~\ref{def:lattice1} implies that a lattice is a subgroup of~$(\mR^m,+)$, i.e., for any~${\bf c}$ and~${\bf c}'$ in the lattice, the difference~${\bf c} - {\bf c}'$ also belongs to the lattice. It also implies that a lattice must be discrete, i.e., every sequence of lattice points that converges must be ultimately constant. (This can be seen by proved by considering the $i$-th coordinates of 
the vectors in the sequence, for any $i \leq n$, and using the discreteness of~$\mZ$.)  Graphically, this means that 
a lattice cannot have points of accumulation, i.e., that there is always a minimal distance separating any two given points of a
considered lattice. Figure~\ref{fig:lattice2} gives a subgroup of~$(\mR^2,+)$ that is not a lattice. 

In fact, being a discrete subgroup of~$(\mR^m,+)$ is an alternative definition of Euclidean lattices

\begin{lemma}
\label{def:lattice2}
Let~$m >0$. A set~$L \subset \mR^m$ is a lattice if and only if the following two conditions hold:
\begin{itemize}
\item[$\bullet$] for any ${\bf b}, {\bf b}' \in L$, we have~${\bf b} - {\bf b}' \in L$,
\item[$\bullet$] for any converging sequence~$({\bf c}_j)_{j=1,2,\ldots}$ with~${\bf c}_j \in L$ for all~$j$, there exists~$j^\star$ such that~${\bf c}_j = {\bf c}_{j^\star}$ for all~$j \geq j^\star$. 
\end{itemize}
\end{lemma}

For example, the set~$S= \mZ \cdot 1 + \mZ \cdot \sqrt{2}$ is not a lattice. Consider the continued fraction convergent~$p_k/q_k \in \mathbb{Q}$ of~$\sqrt{2}$, for~$k \geq 0$. Then it holds that~$|\sqrt{2} - p_k /q_k| \leq 1/q_k^2$ and $q_k \rightarrow_k +\infty$. Now, note that the element~$c_k := -p_k + q_k \cdot  \sqrt{2}$ belongs to~$S$ and satisfies~$|c_k| \rightarrow_k 0$. However, as~$\sqrt{2}$ is not rational, none of the $c_k$'s is $0$ and the sequence cannot be ultimately constant.

We now introduce two families of lattices playing a central role in lattice-based cryptography.

\begin{definition}
\label{def:qary}
Let~$m \geq n >0$ and~$q \geq 2$ be integers. Let~${\bf A} \in (\mZ/q\mZ)^{m \times n}$. 
\begin{itemize} 
\item[$\bullet$] The image lattice associated to~${\bf A}$ is defined as
\[
\Lambda({\bf A}) = {\bf A} \cdot (\mZ/q\mZ)^n + q \cdot \mZ^m = \{ {\bf y} \in \mZ^m: \exists {\bf s} \in (\mZ/q\mZ)^n, {\bf y} = {\bf A}\cdot {\bf s} \bmod q\};
\]
\item[$\bullet$] The image lattice associated to~${\bf A}$ is defined as
\[
\Lambda^\perp({\bf A}) = \{ {\bf x} \in \mZ^m: {\bf x}^T \cdot {\bf A} = {\bf 0} \bmod q\}.
\]
\end{itemize}
\end{definition}

In the following, for the sake of compactness, the notation~$\mZ_q$ will refer to $\mZ/q\mZ$ (note that this does not refer 
to $q$-adic integers, which do not play any role in these lecture notes). 

By using Lemma~\ref{def:lattice2}, it can be checked that the sets $\Lambda({\bf A})$ and~$\Lambda^\perp({\bf A})$ are indeed lattices. As they are defined modulo~$q$, they are sometimes called~$q$-ary lattices. Also, due to their relevance in lattice-based cryptography, the lattice $\Lambda({\bf A})$ is often called the LWE lattice related to~${\bf A}$ and the lattice $\Lambda^\perp({\bf A})$ is often called the SIS lattice related to~${\bf A}$. The LWE and SIS problems are defined in Section~\ref{se:LBC}. The $\Lambda({\bf A})$ lattice is also known as the construction~A lattice 
applied to the linear code~${\bf A} \cdot \mZ_q^n$. The~``A'' in ``construction~A'' is unrelated to the choice of notation for matrix~${\bf A}$: construction~A refers to the first construction out of several, to obtain lattices from linear codes~\cite[Chapter~7]{CS99}. 
The following result shows that the two constructions are equivalent. 


\begin{lemma}
\label{le:qary_eq}
Let~$m \geq n >0$ and~$q \geq 2$ be integers, and~${\bf A} \in \mZ_q^{m \times n}$.
Let~${\bf H} \in \mZ_q^{m \times k}$ be such that~${\bf H} \cdot \mZ_q^k = \{ {\bf x} \in  \mZ_q^m : {\bf x}^T \cdot {\bf A} = {\bf 0} \bmod q\}$. Then:
\[
\Lambda^\perp({\bf A}) = \Lambda({\bf H}) \ \ \mbox{ and } \ \ \Lambda({\bf A}) =  \Lambda^\perp({\bf H}).
\]
\end{lemma}

\begin{proof}
Note  that the matrix~${\bf H}$ is well-defined and that its column span ${\bf H} \cdot \mZ_q^k$ is  the kernel of the 
map~$\phi_{\bf A}: {\bf x} \mapsto {\bf A}\cdot {\bf x} \bmod q$. This provides the first equality. Now, note that~${\bf A} \cdot \mZ_q^n$
is  the kernel of the map~$\phi_{\bf H}$. It is indeed contained in it by definition of~${\bf H}$ and equality follows by a cardinality argument
based on the isomorphism~$\im(\phi) \simeq \mZ_q^m / \ker (\phi)$ which holds for any linear map~$\phi$ whose domain is~$\mZ_q^m$:
\[
|\ker(\phi_{\bf H})| = \frac{q^m}{|\im(\phi_{\bf H})|} = \frac{q^m}{|\ker(\phi_{\bf A})|} = |\im( \phi_{\bf A})|.
\]
This gives the second equality.
\end{proof}

The statement of the lemma and its proof above are a little cumbersome, because we considered a modulus~$q$ of arbitrary arithmetic shape. If $q$ is assumed to be prime, then~$\mZ_q$ is a field and we are reduced to studying vector spaces over a finite field. For statements involving these lattices, we will often assume that~$q$ is prime. Most often, the results also hold up to minor modifications for more general moduli, but their proofs tend to be more technical.


We now introduce the notion of lattice duality. 

\begin{definition}
\label{def:dual}
Let $L$ be a lattice. We define the dual of $L$ as 
\[
\widehat{L} = \left\{ \widehat{\bf b} \in \mbox{span}_\mR (L): \forall {\bf b} \in L, \ps{{\bf b}}{\widehat{\bf b}} \in \mZ \right\}.
\]
\end{definition}

This can be seen as a generalization of the inverse over~$\mR \setminus \{0\}$, as $\widehat{x \cdot \mZ} = \frac{1}{x} \cdot \mZ$ for any~$x \in \mR \setminus \{0\}$. We have the following elementary properties on lattice duals. 

\begin{lemma}
\label{le:dual_props}
Let~$L$ be a lattice. Then~$\widehat{L}$ is a lattice and~$\widehat{\widehat{L}} = L$.
\end{lemma}

We now come back to our running example of construction A lattices.

\begin{lemma}
\label{le:constA_dual}
Let~$q\geq 2$ prime, $m>n>0$ and~${\bf A} \in \mZ_q^{m \times n}$. Then we have:
\[\widehat{\Lambda({\bf A})} = \frac{1}{q} \cdot \Lambda^\perp({\bf A}).\] 
\end{lemma}

\begin{proof}
We start with the inclusion $\widehat{\Lambda({\bf A})} \subseteq  (1/q) \cdot \Lambda^\perp({\bf A})$. 
Let $\widehat{\bf b} \in \widehat{\Lambda({\bf A})}$. By definition, for 
all~${\bf s} \in \mZ_q^n$ and all ${\bf y} \in \mZ^m$ such that~${\bf y} = {\bf A}\cdot {\bf s} \bmod q$, 
we have~$\ps{\widehat{\bf b}}{{\bf y}} \in \mZ$. Taking~${\bf s}= 0$ and setting~${\bf y} = (0^{i-1},q,0^{m-i})^T$ for all~$i \leq m$, we obtain that $\widehat{\bf b} \in (1/q) \cdot \mZ^m$. Let us write $\widehat{\bf b} = (1/q) \cdot {\bf k}$ for~${\bf k} \in \mZ^m$. 
By definition of~$\widehat{\Lambda({\bf A})}$, for 
all~${\bf s} \in \mZ_q^n$ and all ${\bf y} \in \mZ^m$ such that~${\bf y} = {\bf A}\cdot {\bf s} \bmod q$, 
we have~$\ps{{\bf k}}{{\bf y}} \in q \cdot \mZ$ and hence~${\bf k}^T \cdot {\bf A} \cdot {\bf s} = 0 \bmod q$. By setting~${\bf s} = (0^{i-1},1,0^{n-i})^T$ for all~$i \leq n$, we finally obtain that~${\bf k}^T \cdot {\bf A}= {\bf 0} \bmod q$.

We now prove the reverse inclusion. For this purpose, take~${\bf k} \in \mZ^m$ such that ${\bf k}^T \cdot {\bf A}= {\bf 0} \bmod q$.
We would like to show that for all~${\bf s} \in \mZ_q^n$ and all~${\bf y} \in \mZ^m$ such that~${\bf y} = {\bf A} \cdot {\bf s} \bmod q$, we have~$\ps{(1/q) \cdot {\bf k}}{{\bf y} } \in \mZ$. Note that the latter is equivalent to~${\bf k}^T \cdot {\bf A} \cdot {\bf s} = 0 \bmod q$: it is implied by the assumption that  ${\bf k}^T \cdot {\bf A}= {\bf 0} \bmod q$.
\end{proof}



\subsection{Bases and volume}

Definition~\ref{def:lattice1} states that for any lattice~$L \subset \mR^m$, there exist ${\bf b}_1, \ldots, {\bf b}_n$  
linearly independent in~$\mR^m$ such that~$L= \sum_{i \leq n} \mZ\cdot {\bf b}_i$. The vectors~${\bf b}_1, \ldots, {\bf b}_n$  
are said to form a \emph{basis} of~$L$. If~${\bf B} \in \mR^{m \times n}$ is the matrix whose columns are the~${\bf b}_i$'s (in any order), then~${\bf B}$ has rank~$n$, $L = {\bf B} \cdot \mZ^n$ and~${\bf B}$ is called a \emph{basis matrix} of~$L$.

Bases are not unique (one can always replace a basis vector by its opposite), but they all share the same cardinality~$n$, called the \emph{dimension} of~$L$. Recall that~$n$ can be lower than the embedding dimension~$m$.  

\begin{lemma}
\label{le:dim}
If ${\bf b}_1, \ldots, {\bf b}_n$ and ${\bf c}_1, \ldots, {\bf c}_{n'}$ are two bases of the same lattice~$L$, then we have that~$n=n'$. 
Further, there exists~${\bf U} \in \mZ^{n \times n}$ with~$\det ({\bf U}) = \pm 1$ such that 
\[
\left({\bf c}_1|\ldots |{\bf c}_n \right) = \left({\bf b}_1|\ldots |{\bf b}_n \right) \cdot {\bf U}.
\]
\end{lemma}

Note that~${\bf U}$ is multiplied from the right, as we consider column vectors. 
A matrix~${\bf U} \in \mZ^{n \times n}$ with~$\det ({\bf U}) = \pm 1$ is called \emph{unimodular}. Such matrices are exactly
the integer matrices whose inverses are integer matrices, i.e., the elements of~$\mbox{GL}_n(\mZ)$. Examples of unimodular matrices include permutation matrices, diagonal matrices with entries equal to~$\pm 1$ and upper triangular integer matrices with~$1$'s on the diagonal. In fact, it can be proved that any unimodular matrix can be written as a product of such matrices. 

Lemma~\ref{le:dim} implies
that a 1-dimensional lattice~$L = x \cdot \mZ$ with~$x \in \mR\setminus \{0\}$ has exactly  two bases: $x$ and~$-x$. When~$n \geq 2$, 
Lemma~\ref{le:dim} implies that a given lattice~$L$ admits infinitely many lattice bases: indeed, unimodular matrices include integer upper-triangular matrices with $1$'s on the diagonal, and all of these give different bases.  

As the dimension~$n$ is shared across all bases of a given lattice, it is called a \emph{lattice invariant}. 
In the rest of this subsection, we will introduce a few other such lattice invariants. 

\begin{definition}
\label{def:vol}
Let~$L$ be a lattice with a basis matrix~${\bf B}$. The determinant of~$L$ is defined as
\[
\det(L) = \sqrt{\det( {\bf B}^T \cdot {\bf B})}.
\]
\end{definition}

The determinant of a lattice is sometimes called the volume or co-volume. Note that when~$L$ is full-rank, the definition can 
be simplified to~$\det(L) = |\det ({\bf B})|$. The determinant is indeed a lattice invariant, as formally stated in the 
following lemma. 

\begin{lemma}
Let~${\bf B}$ and~${\bf C}$ be two matrix bases of the same lattice~$L$. Then we have
 \[
 \sqrt{\det( {\bf B}^T \cdot {\bf B})} =  \sqrt{\det( {\bf C}^T \cdot {\bf C})}.
\]
\end{lemma}

The proof follows by considering a unimodular matrix~${\bf U}$ such that ${\bf C} = {\bf B} \cdot {\bf U}$. 
Geometrically, and as illustrated in Figure~\ref{fig:det}, the determinant is the volume of the parallelepiped~${\bf B} \cdot [0,1]^n$. 

The following result focuses on dual lattices. 

\begin{lemma}
\label{le:dual_det}
Let~$L$ be a lattice with a basis matrix~${\bf B}$. Then~$\widehat{{\bf B}} = {\bf B} \cdot ( {\bf B}^T \cdot {\bf B})^{-T}$ 
is a basis of~$\widehat{L}$, and~$\det(\widehat{L}) = 1/\det(L)$.  
\end{lemma}
 
Note that when the lattice is full-rank, the definition of~$\widehat{{\bf B}}$ simplifies to~$\widehat{{\bf B}} =  {\bf B}^{-T}$. 

We now return to our running example of Construction~A lattices. 

\begin{lemma}
\label{le:consA_dimdet}
Let~$q\geq 2$ prime, $m>n>0$ and~${\bf A} \in \mZ_q^{m \times n}$. 
We have 
\begin{eqnarray*}
\dim (\Lambda({\bf A})) = m \ \ &\mbox{ and }& \ \ \det(\Lambda({\bf A})) = q^{m - r}, \\
\dim (\Lambda^{\perp}({\bf A})) = m \ \ &\mbox{ and }& \ \ \det(\Lambda^{\perp}({\bf A})) = q^{r},
\end{eqnarray*}
where~$r$ is the rank of~${\bf A}$. Further, if~${\bf A}$ is sampled uniformly, then its 
rank is~$n$ with probability~$\geq 1 - q^{-m+n}$. 
\end{lemma}

We stress that the dimension of these lattices is indeed~$m$ (even though in the rest of this section, the lattice dimension is denoted
by~$n$).\footnote{DS: les notations sont sous-optimales ici} 


\begin{proof}
The dimension equalities follow from the fact that~$q \mZ^m$ is contained in 
both~$\Lambda({\bf A})$ and~$\Lambda^{\perp}({\bf A})$. By Lemmas~\ref{le:constA_dual} and~\ref{le:dual_det},
the determinant equality for~$\Lambda({\bf A})$ and~$\Lambda^{\perp}({\bf A})$ are equivalent. We prove the first one
by explicitly constructing a basis for~$\Lambda({\bf A})$. 

As the rank of~${\bf A}$ is~$r$, there exist rank-$r$ matrices~${\bf A}' \in \mZ_q^{m \times r}$ and~${\bf T} \in \mZ_q^{r \times n}$ such  that~${\bf A} = {\bf A}' \cdot {\bf T} \bmod q$. Up to reordering the rows of~${\bf A}'$, we may assume that its first~$r$ rows form 
an invertible matrix. By left-multiplying~${\bf T}$ (resp.\ right-multiplying~${\bf A}'$) by this matrix (resp.\ its inverse), we can assume that the first~$r$ rows of~${\bf A}'$ form the identity matrix. Let us write~${\bf A}' = ({\bf Id} | {\bf A}''^T)^T$ for~${\bf A}'' \in \mZ_q^{(m-r) \times r}$. By using the definition of~$\Lambda({\bf A})$, we can write:
\[
\Lambda({\bf A}) = {\bf A} \cdot \mZ_q^n + q\cdot \mZ^m = \left[ \begin{array}{c|c|c} 
{\bf Id} & q \cdot {\bf Id} & {\bf 0} \\ \hline
{\bf A}'' & {\bf 0} &  q \cdot {\bf Id}
\end{array}\right] \cdot \mZ^{m+r},
\]
where the first block-row has~$r$ rows and $r+r+(m-r)$ columns and the coefficients of~${\bf A}''$ are viewed as integers. 
Note that this matrix has linearly dependent columns and hence
cannot be a basis of~$\Lambda({\bf A})$. However, the second block-column (i.e., columns $r+1$ to~$2r$) can be seen to be linearly 
dependent with the others. For example, the $(r+1)$-th column is $q$ times the first column to which one has subtracted the  columns of the third block  mutiplied by the first column of~${\bf A}''$. The second block is hence superfluous and we can write:
  \[
\Lambda({\bf A}) =  \left[ \begin{array}{c|c} 
{\bf Id} &  {\bf 0} \\ \hline
{\bf A}'' &   q \cdot {\bf Id}
\end{array}\right] \cdot \mZ^{m}.
\]
As the matrix is lower triangular with non-zero diagonal coefficients, its columns are linearly independent. They hence form a basis of~$\Lambda({\bf A})$, and we can compute the lattice determinant by counting the number of occurrences of~$q$ on the diagonal. 

Now, assume that~${\bf A}$ is sampled uniformly in~$\mZ_q^{m \times n}$. It is of rank~$n$ if and only if the $i$-th column is linearly independent from the previous ones, for each~$i \leq n$. The probability~$p$ that this occurs satisfies:
\[
1- p \leq \sum_{i=1}^n q^{-m+i-1}  = q^{-m} \frac{q^n-1}{q-1} \leq q^{-m+n}.
\]
Re-arranging terms allows to complete the proof.
\end{proof}

\textcolor{red}{Damien: Do we need the HNF? How to get a short basis from short vectors?}



\subsection{QR-factorisation and Gram-Schmidt orthogonalisation}

Given a full-rank matrix~${\bf B} \in \mR^{m \times n}$, there exist~${\bf Q} \in \mR^{m \times n}$ 
and~${\bf R} \in \mR^{n \times n}$ such that:
\begin{itemize}
\item[$\bullet$] ${\bf B} = {\bf Q} \cdot {\bf R}$;
\item[$\bullet$] the columns of~${\bf Q}$ are orthonormal vectors, i.e., ${\bf Q}^T \cdot {\bf Q} = {\bf Id}$;
\item[$\bullet$] ${\bf R}$ is upper-triangular with positive diagonal coefficients. 
\end{itemize}
Further, there exists a unique pair~$({\bf Q}, {\bf R})$ such that the above three properties hold. The matrices~${\bf Q}$ and~${\bf R}$ are respectively called the \emph{Q-factor} and \emph{R-factor} of~${\bf B}$, and the pair is referred to as its \emph{QR-factorisation}. 

The QR-factorisation is closely related to the Gram-Schmidt orthogonalisation of the column 
vectors~$({\bf b}_1,\ldots, {\bf b}_n)$ of~${\bf B}$. For all~$i$, we define:
\[
\widetilde{{\bf b}}_i = {\bf b}_i - \sum_{j <i} \mu_{ij}\cdot \widetilde{{\bf b}}_j, \ \ \mbox{ where } \ \ 
 \mu_{ij} = \frac{\ps{{\bf b}_i}{\widetilde{{\bf b}}_j}}{\|\widetilde{{\bf b}}_j\|^2}.
\]
The vectors~$\widetilde{{\bf b}}_1, \ldots, \widetilde{{\bf b}}_n$ are called the \emph{Gram-Schmidt orthogonalisation}
of~$({\bf b}_1,\ldots, {\bf b}_n)$. We stress that the Gram-Schmidt orthogonalisation depends on the order of the considered vectors
(and that the Gram-Schmidt orthogonalisation of permuted vectors is not the permutation of the Gram-Schmidt orthogonalisation). 
The vectors $\widetilde{{\bf b}}_i$ are pair-wise orthogonal. 

The correspondence between QR-factorisation and Gram-Schmidt orthogonalisation is formalized in the following lemma. 

\begin{lemma}
Let~${\bf B} \in \mR^{m \times n}$ be full-rank with columns~$({\bf b}_1,\ldots, {\bf b}_n)$. Let~$({\bf Q}, {\bf R})$ denote the QR-factorisation of~${\bf B}$.  Let~$(\widetilde{{\bf b}}_1, \ldots, \widetilde{{\bf b}}_n)$ be the Gram-Schmidt orthogonalisation of~$({\bf b}_1,\ldots, {\bf b}_n)$ and~$\mu_{ij} =  \ps{{\bf b}_i}{\widetilde{{\bf b}}_j} / \|\widetilde{{\bf b}}_j\|^2$ for~$i \geq j$. The following relations hold.
\begin{itemize}
\item[$\bullet$] for every $i$, the $i$-th column of~${\bf Q}$ is $\widetilde{{\bf b}}_i / \|\widetilde{{\bf b}}_i\|$;
\item[$\bullet$] for every $i$, we have $\|\widetilde{{\bf b}}_i\| = r_{ii}$;
\item[$\bullet$] for every $i \geq j$, we have $\mu_{ij} = r_{ji}/r_{jj}$.
\end{itemize}
\end{lemma} 
 
Given the above equivalency, one may prefer to work with the QR-factorization or the Gram-Schmidt orthogonalisation. 
In these notes, we will typically prefer to work with the QR-factorization, as it is typically more compact. One advantage of the 
Gram-Schmidt orthogonalisation over the QR-factorization is that if the matrix~${\bf B}$ is integral, then the $\widetilde{{\bf b}}_i$'s 
and~$\mu_{ij}$'s are rationals. Further, the bit-sizes of the numerators and denominators is polynomial in the bit-sizes of the 
coefficients of~${\bf B}$ and the Gram-Schmidt orthogonalization can be computed exactly in polynomial-time 
(with respect to $m$ and  the bit-sizes of the coefficients of~${\bf B}$). Oppositely, the QR-factorisation may not be rational. 
However, one could represent the QR-factorisation with the Gram-Schmidt orthogonalisation. More interestingly, in practice, the 
rationals involved in the Gram-Schmidt orthogonalisation are very large, and one prefers to resort to floating-point computations. 
As the numerical stability of the QR-factorisation algorithms has been extensively studied (see, e.g., \cite{Higham02}), in the context 
of such approximate computations, it is typically more convenient to use the QR-factorisation. 

The lattice determinant can be derived from the QR-factorization of any lattice basis. 

\begin{lemma}
Let~$L \subset \mR^m$ be an $n$-dimensional lattice. Let~${\bf B}$ be the matrix representation of a basis of~$L$, and~${\bf R}$ 
be its R-factor. Then we have:
\[\det(L) = \prod_{i \leq n} r_{ii}.\]
\end{lemma} 


\textcolor{red}{Damien: Do we need the R-factor of the dual?}




\subsection{Lattice minimum}

We have already introduced the lattice determinant, which is an invariant of the lattice (i.e., it does not depend on the 
specific choice of the basis). Note that if the lattice is integral and given by a basis, then the determinant can be computed 
in polynomial time. 
We now focus on another lattice invariant, namely the lattice minimum, which is not known to be computable efficiently.  

\begin{definition}
Let~$L$ be a lattice. Its minimum~$\lambda_1(L)$ is the (Euclidean) norm of any shortest non-zero vector of~$L$. 
\end{definition}

The existence of a shortest non-zero vector in~$L$ is guaranteed by the discreteness of the lattice (Lemma~\ref{def:lattice2}). 
This implies that~$\lambda_1(L)$ is well-defined. Note that the minimum is always reached at least twice: if a vector is in a lattice 
and has a given norm, so does its opposite. It may be reached more times. The choice of the Euclidean norm is somewhat arbitrary:
it is mostly justified by technical convenience.  

The litterature sometimes considers the packing radius of a lattice, defined as the largest real~$r>0$ such that any two balls of radius~$r$ centered in lattice points do not overlap more than on a single point. This is exactly half the lattice minimum.  

The QR-factorisation provides an efficiently computable lower bound for the lattice minimum. 

\begin{lemma}
\label{le:lambda_lower_bound}
et~$L \subset \mR^m$ be an $n$-dimensional lattice. Let~${\bf B}$ be the matrix representation of a basis of~$L$, and~${\bf R}$ 
be its R-factor. Then we have:
\[ \lambda_1(L) \geq  \min_{i \leq n} r_{ii}.\]
\end{lemma}

\begin{proof}
Let~${\bf s} \in L$ with~$\|{\bf s}\| = \lambda_1(L)$. 
There exists a non-zero  vector~${\bf k} \in \mZ^n$ such that~${\bf s} = {\bf B} \cdot {\bf k}$. 
As the columns of the Q-factor of~${\bf B}$ are orthonormal, we have that
\[
\lambda_1(L) = \|{\bf s}\| = \|{\bf B} \cdot {\bf k}\| = \|{\bf R} \cdot {\bf k}\|. 
\]
Let~$i$ be the index of the last non-zero coefficient of~${\bf k}$. Then the $i$-th coefficient of  ${\bf R} \cdot {\bf k}$ 
is $r_{ii} \cdot k_i$. We must then have
\[
\lambda_1(L) = \|{\bf R} \cdot {\bf k}\| \geq r_{ii} \cdot |k_i|.
\]
As $k_i$ is a non-zero integer, we have~$|k_i| \geq 1$, leading to the inequality~$\lambda_1(L) \geq r_{ii}$.
\end{proof}



Minkowski's theorem gives an upper bound for the minimum of a lattice, as a function of its determinant. If the lattice is integral, the
upper bound can be computed in polynomial-time to an arbitrary precision. 

\begin{theorem}
\label{th:Minko}
For every lattice~$L$ of dimension~$n$, we have~$\lambda_1(L) \leq \sqrt{n} \cdot \det(L)^{1/n}$.
\end{theorem}

Note that the minimum can be arbitrarily small compared to the $n$-th root of the determinant, as can be seen from the lattice
spanned by the columns of the diagonal matrix~$\mbox{diag}(\eps, 1/\eps)$ for~$\eps>0$ going to~$0$. This does not contradict Lemma~\ref{le:lambda_lower_bound}, as $r_{11} = \eps$. We note that the upper bound from Theorem~\ref{th:Minko} can 
also be expressed in terms of the R-factor~${\bf R}$  of an arbitrary basis of~$L$: 
indeed, the quantity $\det(L)^{1/n}$ is the geometric mean of the diagonal coefficients~$r_{ii}$ (for~$i \leq n$). 

We will prove Minkowski's theorem for the Construction~A lattices, up to a loss of a factor~2. (Recall that their determinants were computed in Lemma~\ref{le:consA_dimdet}.)

\begin{lemma}
\label{le:Minko_constA}
Let~$q\geq 2$ prime, $m>n>0$ and~${\bf A} \in \mZ_q^{m \times n}$. Let~$r$ 
denote the rank of~${\bf A}$. 
We have 
\[
\lambda_1( \Lambda({\bf A}))  \leq 
 2\sqrt{m}  \cdot  q^{1-r/m} \ \  \mbox{ and } \ \ 
\lambda_1 (\Lambda^{\perp}({\bf A})) \leq  2 \sqrt{m} \cdot  q^{r/m}.
\]
\end{lemma}


\begin{proof}
 Let us consider all the vectors~${\bf x} \in \mZ^m$ with~$\|{\bf x}\|_\infty \leq \rho$, for some~$\rho$ to be determined later. If there are more than~$q^r$ such vectors, then, by the pigeon hole principle,  
there exist ${\bf x}, {\bf x}' \in \mZ^m$ distinct such that~${\bf x}^T \cdot {\bf A} =  {\bf x'}^T \cdot {\bf A} \bmod q$ and~$\|{\bf x}\|_\infty,\|{\bf x}'\|_\infty \leq \rho$. The vector~${\bf x} - {\bf x}'$ is hence non-zero, belongs to $\Lambda^{\perp}({\bf A})$ and has infinity norm~$\leq 2 \sqrt{m} \cdot \rho$. This implies that  $\lambda_1(\Lambda^{\perp}({\bf A})) \leq 2 \sqrt{m} \cdot  \rho$. 
Setting~$\rho = q^{r/m}$ gives the result. 

Lemma~\ref{le:qary_eq} allows to obtain the desired bound for~$\Lambda({\bf A})$.
\end{proof}

By using the specificities of the Construction~A lattices, we can obtain bounds that can be lower for some ranges of parameters. 

\begin{lemma}
\label{le:Minko_constA}
Let~$q\geq 2$ prime, $m>n>0$ and~${\bf A} \in \mZ_q^{m \times n}$. Let~$r$ 
denote the rank of~${\bf A}$. 
We have 
\[
\lambda_1( \Lambda({\bf A}))  \leq 
 \min \left( 2\sqrt{m}  \cdot  q^{(m-r)/m}, q \right) \ \  \mbox{ and } \ \ 
\lambda_1 (\Lambda^{\perp}({\bf A})) \leq  \min\left( \min_{1 \leq  \mu \leq m} 2 \sqrt{\mu} \cdot  q^{r/\mu}, q\right).
\]
\end{lemma}

Note that the function $\mu \mapsto 2 \sqrt{\mu} \cdot  q^{r/\mu}$ first decreases but then increases with~$\mu$. As a result, 
for~$m < 2 r \log q$, the second bound is $\min(2 \sqrt{m} \cdot  q^{r/m}, q)$ and for $m \geq 2 r \log q$ and~$q \geq 3$, 
it is $\leq \min(4\sqrt{r \log q}, q)$.

\begin{proof}
These lattices always contain~$\mZ_q^m$, which explains why their minimum is no greater than~$q$. 
For the $\Lambda^{\perp}({\bf A})$ lattice, one may further observe that $\Lambda^{\perp}({\bf A}') \subseteq \Lambda^{\perp}({\bf A})$  for any~${\bf A}' \in \mZ_q^{\mu \times n}$ made of $\mu$ out of the~$m$ rows of~${\bf A}$, and further that the rank of~${\bf A}'$ is no larger than the rank of~${\bf A}$. 
\end{proof}

\begin{comment}
2 (1/2sqr(x) q^r/x + sqrt(x)*r*logq*q^r/x*(-1/x^2))
= q^(r/x) x^(-3/2)  ( x - 2r*logq )
=> decreasing up to x = floor (2r log q)
2sqrt(2rlogq) * exp(rlog q / floor (2r log q))
\end{comment}

It turns out that Minkowski's theorem is essentially sharp if we consider random lattices for an appropriate distribution over the 
set of lattices. This is known as the Minkowski-Hlawka theorem. We prove a variant for Construction~A lattices.

\begin{theorem}
Let $q\geq 2$ prime, $m  > n>0$ and~${\bf A}$ sampled uniformly in~$\mZ_q^{m \times n}$. Then we have
\begin{eqnarray*}
            \lambda_1( \Lambda({\bf A})) & \geq  & \min \left(\sqrt{m} \cdot q^{1-n/m}/10 -\sqrt{m/2}, q \right), \\
            \lambda_1( \Lambda^{\perp}({\bf A})) & \geq & \min \left(\sqrt{m} \cdot q^{n/m}/10 -\sqrt{m/2}, q \right),
\end{eqnarray*}
where each bound holds with probability~$\geq 1-2^{-m}$.
\end{theorem}
Note that for some values of the parameters, notably $m$ very large in the second case, 
the lower bound on the lattice minimum may be negative. In such a case, the statement is vacuous. 
For many parameters, the lower bounds are within a factor~10 of the upper bounds given by Minkowski's theorem.

\begin{proof}
For~$\rho < q$, the event~$\lambda_1(\Lambda({\bf A})) \leq \rho$ occurs exactly if there exists~${\bf s} \in \mZ_q^n \setminus \{\bf 0\}$ 
such that $0 < \|{\bf A} \cdot {\bf s} \bmod q \|< \rho$. Here the reduction modulo~$q$ takes values in~$[-q/2,q/2)$. 
By the union bound, we have:
\begin{eqnarray*}
\Pr_{\bf A} \left[\lambda_1(\Lambda({\bf A})) \leq \rho \right] & = & \Pr_{\bf A} \left[\exists {\bf y} \in \mathcal{B}({\bf 0},\rho) \cap \mZ^m, \exists {\bf s} \in \mZ_q^n \setminus \{\bf 0\}: {\bf A} \cdot {\bf s} = {\bf y} \bmod q \right] \\
&\leq& 
\sum_{\substack{{\bf y} \in \mathcal{B}({\bf 0},\rho) \cap \mZ^m\\ {\bf s} \in \mZ_q^n \setminus \{\bf 0\}}} 
 \Pr_{\bf A} \left[{\bf A} \cdot {\bf s} = {\bf y} \bmod q \right].
\end{eqnarray*}
As the rows of~${\bf A}$ are statistically independent, ${\bf s}$ is non-zero and~$q$ is prime, we have that $\Pr_{\bf A} [{\bf A} \cdot {\bf s} = {\bf y} \bmod q ] = q^{-m}$. This implies the following upper bound:
\[
\Pr_{\bf A} \left[\lambda_1(\Lambda({\bf A})) \leq \rho \right] \leq \left|\mathcal{B}({\bf 0},\rho) \cap \mZ^m\right| \cdot q^{n-m}.
\]
By enlarging the radius by $\sqrt{m/2}$, the zero-centered ball contains all the unit hypercubes centered in the points of $\mathcal{B}({\bf 0},\rho) \cap \mZ^m$. This gives that 
\[
|\mathcal{B}_m({\bf 0}, \rho) \cap \mZ^m| \ \leq  \ \mbox{vol}(\mathcal{B}_m({\bf 0}, \rho +\sqrt{m/2}))
\ \leq \  (\rho +\sqrt{m/2})^m \cdot \left(\frac{2\pi e}{m}\right)^{m/2},
\]
from which the proof can be completed.

We now focus on the~$\lambda_1(\Lambda^{\perp}({\bf A}))$ lattice. For~$\rho < q$, we have:
\begin{eqnarray*}
\Pr_{\bf A} \left[\lambda_1(\Lambda^{\perp}({\bf A})) \leq \rho \right] & = & \Pr_{\bf A} \left[\exists {\bf x} \in \mathcal{B}({\bf 0},\rho) \cap \mZ^m \setminus \{{\bf 0}\}: {\bf x}^T \cdot {\bf A} = {\bf 0} \bmod q \right] \\
&\leq& 
\sum_{{\bf x} \in \mathcal{B}({\bf 0},\rho) \cap \mZ^m \setminus \{\bf 0\}}  \Pr_{\bf A} \left[{\bf x}^T \cdot {\bf A} = {\bf 0} \bmod q \right].
\end{eqnarray*}
Now, as~$q$ is prime and~${\bf x}$ is non-zero modulo~$q$, we have that $ \Pr_{\bf A} [{\bf x}^T \cdot {\bf A} = {\bf 0} \bmod q] = q^{-n}$. This gives:
\[
\Pr_{\bf A} \left[\lambda_1(\Lambda^{\perp}({\bf A})) \leq \rho \right] \leq  \left|\mathcal{B}({\bf 0},\rho) \cap \mZ^m\right| \cdot q^{-n}.
\]
We can conclude by using the above bound on $|\mathcal{B}_m({\bf 0}, \rho) \cap \mZ^m|$.
\end{proof}

\begin{comment}
r^{m} (2pie/m)^(m/2) (1+sqrt(m/2)/r)^m q^(n-m) \leq 2^(-m)
r +sqrt(m/2) <= q^(1-n/m) * sqrt(m/8pie)
\end{comment}

\subsection{Other lattice invariants}

Other lattice invariants may be considered, depending on the context. An important example is the covering radius~$\rho(L)$, 
defined as the smallest~$r$ such that the balls of radius~$r$ centered in the points of~$L$ cover~$\mbox{span}_\mR(L)$.
The successive minima, defined below, are also frequently considered. The measure the lattice 'discreteness' in all possibly dimensions
rather than only one, as does the lattice minimum. 

\begin{definition}
Let~$L \subset \mR^m$ be a lattice of dimension~$n$. For any~$i \leq n$, we define the~$i$-th minimum of~$L$ by:
\[
\lambda_i(L) = \min \left( r: \dim(\mbox{span}_\mR( L \cap \mathcal{B}_m ({\bf 0}, r))) \geq i \right). 
\]
\end{definition}

Note that the minima are always reached. 

\begin{lemma}
For any lattice~$L \subset \mR^m$ of dimension~$n$, there exist~${\bf s}_1, \ldots, {\bf s}_n \in L$ linearly independent such that~$\|{\bf s}_i\| = \lambda_i(L)$, for all~$i \leq n$. 
\end{lemma}

We stress that the ${\bf s}_i$'s may not form a basis of the lattice~$L$. Indeed, there exist lattices for which no basis reaches 
the minima. 

We have the following transferance bounds. The lower bound can be proved using elementary manipulations of bases, 
whereas the upper bound was proven by Banaszcyk in~\cite{Bana93}. 

\begin{theorem}
Let~$L \subset \mR^m$ be a lattice of dimension~$n$, and $\widehat{L}$ be its dual. We have
\[ 
1 \leq \lambda_1(L) \cdot  \lambda_n(\widehat{L}) \leq n. 
\]
\end{theorem}

\subsection{Lattice Gaussians}

\textcolor{red}{
Damien: What do we need here? Only def and tail bound, and a statement for Klein/GPV? \\
Which normalization? I tend to prefer Vadim's, for convenience \\
A proof of Klein/GPV? The definition of the smoothing parameter?
}

\textcolor{red}{
Alice: I would say def, tail bound and statement for Klein/GPV should be sufficient for me. And I don't care about which normalization we choose. :)
}

\subsection{Selected hard lattice problems}

Many computational problems on lattices can be defined and studied. We refer to~\cite{NSD15} for a partial landscape. 
Here we will introduce just a few problems, which play a more direct role in lattice-based cryptography.

\begin{definition}
Let~$\gamma \geq 1$. 
The bounded distance decoding problem (BDD) is as follows: given as input a lattice~$L$ and a vector~${\bf t} \in \mbox{span}_\mR(L)$ such that~$\mbox{dist}({\bf t}, L) \leq \lambda_1(L) / \gamma$, the goal is to find~${\bf b} \in L$ such that~$\|{\bf b} - {\bf t}\| \leq \lambda_1(L) / \gamma$. 
\end{definition}

In the definition above, the distance $\mbox{dist}({\bf t}, L)$ from~${\bf t}$ to~$L$ is the minimum over all~${\bf b}$'s in~$L$ of the distance between~${\bf t}$ and~${\bf b}$. This minimum is well-defined, by discreteness of the lattice.
Note that when~$\gamma > 2$, there exists only one solution to the problem. We stress that some references use different normalizations: 
the input may be promised to satisfy $\mbox{dist}({\bf t}, L) \leq \lambda_1(L) / (2 \cdot \gamma)$. 

\begin{definition}
Let~$\gamma \geq 1$. 
The shortest vector problem (SVP) is as follows: given as input a lattice~$L$, the goal is to find~${\bf s} \in L$ such that~$0 < \|{\bf s}\| \leq \gamma \cdot \lambda_1(L)$. 
\end{definition}

\begin{definition}
Let~$\gamma \geq 1$. 
The shortest independent vectors problem (SIVP) is as follows: given as input a lattice~$L$ of dimension~$n$, the goal is to 
find~${\bf s}_1,\ldots,{\bf s}_n \in L$ such that~$\max_i \|{\bf s}_i \|  \leq \gamma \cdot \lambda_n(L)$. 
\end{definition}

These three problems become no easier when the parameter~$\gamma$ decreases or the lattice dimension~$n$ increases. 
The parameter~$\gamma$ is typically referred to as the \emph{approximation factor}. 



