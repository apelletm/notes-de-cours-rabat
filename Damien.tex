%!TEX root = main.tex
\section{Background on Lattices}
\label{se:back}


{\em
Dans une partie préliminaire, Damien Stehlé donnera les définitions
élémentaires portant sur les réseaux
euclidiens (bases, minima, volume), qu'il illustrera avec des exemples utiles
pour les
cours avancés (notamment les réseaux $q$-aires). Il abordera les théorèmes
de Minkowski et Minkowski-Hlawka. Ensuite, il introduira
les distributions gaussiennes sur des réseaux,
nécessaires pour les parties suivantes.
}


This first section provides some background on Euclidean lattices, with illustrations
preparing the following sections.  


\subsection{Definitions}

We start with elementary definitions pertaining to Euclidean lattices. 
We refer the reader to~\cite{Siegel89} for additional material, including missing proofs.


A Euclidean lattice is the set of all integral linear combinations of a set if linearly independent vectors in a Euclidean space. 
An illustration is provided in Figure~\ref{fig:lattice1}. The lattice is the set of intersection points in the grid. 
The following is a more formal definition.

\begin{definition}
\label{def:lattice1}
Let~$m\geq n >0$. Let~${\bf b}_1, \ldots, {\bf b}_n$ be linearly independent vectors in~$\mR^m$.
The lattice that they span is
\[
 L\left({\bf b}_1, \ldots, {\bf b}_n\right) = \sum_{1 \leq i \leq n} \mZ \cdot {\bf b}_i = \left\{\sum_i x_i \cdot {\bf b}_i: \forall i, x_i \in \mZ\right\}. 
\]
If~$n=m$, then the lattice is said full-rank. If it is contained in~$\mZ^n$, then it is said integral. 
\end{definition}

The prototypical lattice is~$\mZ^n$, which can be written as the set of all integer linear combinations
of the canonical basis vectors~${\bf b}_i = (0^{i-1}, 1, 0^{n-i})^T$ for~$i \in \{1,\ldots,n\}$.  

Definition~\ref{def:lattice1} implies that a lattice is a subgroup of~$(\mR^m,+)$, i.e., for any~${\bf c}$ and~${\bf c}'$ in the lattice, the difference~${\bf c} - {\bf c}'$ also belongs to the lattice. It also implies that a lattice must be discrete, i.e., every sequence of lattice points that converges must be ultimately constant. (This can be seen by proved by considering the $i$-th coordinates of 
the vectors in the sequence, for any $i \leq n$, and using the discreteness of~$\mZ$.)  Graphically, this means that 
a lattice cannot have points of accumulation, i.e., that there is always a minimal distance separating any two given points of a
considered lattice. Figure~\ref{fig:lattice2} gives a subgroup of~$(\mR^2,+)$ that is not a lattice. 

In fact, being a discrete subgroup of~$(\mR^m,+)$ is an alternative definition of Euclidean lattices

\begin{lemma}
\label{def:lattice2}
Let~$m >0$. A set~$L \subset \mR^m$ is a lattice if and only if the following two conditions hold:
\begin{itemize}
\item[$\bullet$] for any ${\bf b}, {\bf b}' \in L$, we have~${\bf b} - {\bf b}' \in L$,
\item[$\bullet$] for any converging sequence~$({\bf c}_j)_{j=1,2,\ldots}$ with~${\bf c}_j \in L$ for all~$j$, there exists~$j^\star$ such that~${\bf c}_j = {\bf c}_{j^\star}$ for all~$j \geq j^\star$. 
\end{itemize}
\end{lemma}

For example, the set~$S= \mZ \cdot 1 + \mZ \cdot \sqrt{2}$ is not a lattice. Consider the continued fraction convergent~$p_k/q_k \in \mathbb{Q}$ of~$\sqrt{2}$, for~$k \geq 0$. Then it holds that~$|\sqrt{2} - p_k /q_k| \leq 1/q_k^2$ and $q_k \rightarrow_k +\infty$. Now, note that the element~$c_k := -p_k + q_k \cdot  \sqrt{2}$ belongs to~$S$ and satisfies~$|c_k| \rightarrow_k 0$. However, as~$\sqrt{2}$ is not rational, none of the $c_k$'s is $0$ and the sequence cannot be ultimately constant.

We now introduce two families of lattices playing a central role in lattice-based cryptography.

\begin{definition}
\label{def:qary}
Let~$m \geq n >0$ and~$q \geq 2$ be integers. Let~${\bf A} \in (\mZ/q\mZ)^{m \times n}$. 
\begin{itemize} 
\item[$\bullet$] The image lattice associated to~${\bf A}$ is defined as
\[
\Lambda({\bf A}) = {\bf A} \cdot (\mZ/q\mZ)^n + q \cdot \mZ^m = \{ {\bf y} \in \mZ^m: \exists {\bf s} \in (\mZ/q\mZ)^n, {\bf y} = {\bf A}\cdot {\bf s} \bmod q\};
\]
\item[$\bullet$] The image lattice associated to~${\bf A}$ is defined as
\[
\Lambda^\perp({\bf A}) = \{ {\bf x} \in \mZ^m: {\bf x}^T \cdot {\bf A} = {\bf 0} \bmod q\}.
\]
\end{itemize}
\end{definition}

In the following, for the sake of compactness, the notation~$\mZ_q$ will refer to $\mZ/q\mZ$ (note that this does not refer 
to $q$-adic integers, which do not play any role in these lecture notes). 

By using Lemma~\ref{def:lattice2}, it can be checked that the sets $\Lambda({\bf A})$ and~$\Lambda^\perp({\bf A})$ are indeed lattices. As they are defined modulo~$q$, they are sometimes called~$q$-ary lattices. Also, due to their relevance in lattice-based cryptography, the lattice $\Lambda({\bf A})$ is often called the LWE lattice related to~${\bf A}$ and the lattice $\Lambda^\perp({\bf A})$ is often called the SIS lattice related to~${\bf A}$. The LWE and SIS problems are defined in Section~\ref{se:LBC}. The $\Lambda({\bf A})$ lattice is also known as the construction~A lattice 
applied to the linear code~${\bf A} \cdot \mZ_q^n$. The~``A'' in ``construction~A'' is unrelated to the choice of notation for matrix~${\bf A}$: construction~A refers to the first construction out of several, to obtain lattices from linear codes~\cite[Chapter~7]{CS99}. 
The following result shows that the two constructions are equivalent. 


\begin{lemma}
\label{le:qary_eq}
Let~$m \geq n >0$ and~$q \geq 2$ be integers, and~${\bf A} \in \mZ_q^{m \times n}$.
Let~${\bf H} \in \mZ_q^{m \times k}$ be such that~${\bf H} \cdot \mZ_q^k = \{ {\bf x} \in  \mZ_q^m : {\bf x}^T \cdot {\bf A} = {\bf 0} \bmod q\}$. Then:
\[
\Lambda^\perp({\bf A}) = \Lambda({\bf H}) \ \ \mbox{ and } \ \ \Lambda({\bf A}) =  \Lambda^\perp({\bf H}).
\]
\end{lemma}

\begin{proof}
Note  that the matrix~${\bf H}$ is well-defined and that its column span ${\bf H} \cdot \mZ_q^k$ is  the kernel of the 
map~$\phi_{\bf A}: {\bf x} \mapsto {\bf A}\cdot {\bf x} \bmod q$. This provides the first equality. Now, note that~${\bf A} \cdot \mZ_q^n$
is  the kernel of the map~$\phi_{\bf H}$. It is indeed contained in it by definition of~${\bf H}$ and equality follows by a cardinality argument
based on the isomorphism~$\im(\phi) \simeq \mZ_q^m / \ker (\phi)$ which holds for any linear map~$\phi$ whose domain is~$\mZ_q^m$:
\[
|\ker(\phi_{\bf H})| = \frac{q^m}{|\im(\phi_{\bf H})|} = \frac{q^m}{|\ker(\phi_{\bf A})|} = |\im( \phi_{\bf A})|.
\]
This gives the second equality.
\end{proof}

The statement of the lemma and its proof above are a little cumbersome, because we considered a modulus~$q$ of arbitrary arithmetic shape. If $q$ is assumed to be prime, then~$\mZ_q$ is a field and we are reduced to studying vector spaces over a finite field. For statements involving these lattices, we will often assume that~$q$ is prime. Most often, the results also hold up to minor modifications for more general moduli, but their proofs tend to be more technical.


We now introduce the notion of lattice duality. 

\begin{definition}
\label{def:dual}
Let $L$ be a lattice. We define the dual of $L$ as 
\[
\widehat{L} = \left\{ \widehat{\bf b} \in \mbox{span}_\mR (L): \forall {\bf b} \in L, \ps{{\bf b}}{\widehat{\bf b}} \in \mZ \right\}.
\]
\end{definition}

This can be seen as a generalization of the inverse over~$\mR \setminus \{0\}$, as $\widehat{x \cdot \mZ} = \frac{1}{x} \cdot \mZ$ for any~$x \in \mR \setminus \{0\}$. We have the following elementary properties on lattice duals. 

\begin{lemma}
\label{le:dual_props}
Let~$L$ be a lattice. Then~$\widehat{L}$ is a lattice and~$\widehat{\widehat{L}} = L$.
\end{lemma}

We now come back to our running example of construction A lattices.

\begin{lemma}
\label{le:constA_dual}
Let~$q\geq 2$ prime, $m>n>0$ and~${\bf A} \in \mZ_q^{m \times n}$. Then we have:
\[\widehat{\Lambda({\bf A})} = \frac{1}{q} \cdot \Lambda^\perp({\bf A}).\] 
\end{lemma}

\begin{proof}
We start with the inclusion $\widehat{\Lambda({\bf A})} \subseteq  (1/q) \cdot \Lambda^\perp({\bf A})$. 
Let $\widehat{\bf b} \in \widehat{\Lambda({\bf A})}$. By definition, for 
all~${\bf s} \in \mZ_q^n$ and all ${\bf y} \in \mZ^m$ such that~${\bf y} = {\bf A}\cdot {\bf s} \bmod q$, 
we have~$\ps{\widehat{\bf b}}{{\bf y}} \in \mZ$. Taking~${\bf s}= 0$ and setting~${\bf y} = (0^{i-1},q,0^{m-i})^T$ for all~$i \leq m$, we obtain that $\widehat{\bf b} \in (1/q) \cdot \mZ^m$. Let us write $\widehat{\bf b} = (1/q) \cdot {\bf k}$ for~${\bf k} \in \mZ^m$. 
By definition of~$\widehat{\Lambda({\bf A})}$, for 
all~${\bf s} \in \mZ_q^n$ and all ${\bf y} \in \mZ^m$ such that~${\bf y} = {\bf A}\cdot {\bf s} \bmod q$, 
we have~$\ps{{\bf k}}{{\bf y}} \in q \cdot \mZ$ and hence~${\bf k}^T \cdot {\bf A} \cdot {\bf s} = 0 \bmod q$. By setting~${\bf s} = (0^{i-1},1,0^{n-i})^T$ for all~$i \leq n$, we finally obtain that~${\bf k}^T \cdot {\bf A}= {\bf 0} \bmod q$.

We now prove the reverse inclusion. For this purpose, take~${\bf k} \in \mZ^m$ such that ${\bf k}^T \cdot {\bf A}= {\bf 0} \bmod q$.
We would like to show that for all~${\bf s} \in \mZ_q^n$ and all~${\bf y} \in \mZ^m$ such that~${\bf y} = {\bf A} \cdot {\bf s} \bmod q$, we have~$\ps{(1/q) \cdot {\bf k}}{{\bf y} } \in \mZ$. Note that the latter is equivalent to~${\bf k}^T \cdot {\bf A} \cdot {\bf s} = 0 \bmod q$: it is implied by the assumption that  ${\bf k}^T \cdot {\bf A}= {\bf 0} \bmod q$.
\end{proof}



\subsection{Bases and invariants}

Definition~\ref{def:lattice1} states that for any lattice~$L \subset \mR^m$, there exist ${\bf b}_1, \ldots, {\bf b}_n$  
linearly independent in~$\mR^m$ such that~$L= \sum_{i \leq n} \mZ\cdot {\bf b}_i$. The vectors~${\bf b}_1, \ldots, {\bf b}_n$  
are said to form a \emph{basis} of~$L$. If~${\bf B} \in \mR^{m \times n}$ is the matrix whose columns are the~${\bf b}_i$'s (in any order), then~${\bf B}$ has rank~$n$, $L = {\bf B} \cdot \mZ^n$ and~${\bf B}$ is called a \emph{basis matrix} of~$L$.

Bases are not unique (one can always replace a basis vector by its opposite), but they all share the same cardinality~$n$, called the \emph{dimension} of~$L$. Recall that~$n$ can be lower than the embedding dimension~$m$.  

\begin{lemma}
\label{le:dim}
If ${\bf b}_1, \ldots, {\bf b}_n$ and ${\bf c}_1, \ldots, {\bf c}_{n'}$ are two bases of the same lattice~$L$, then we have that~$n=n'$. 
Further, there exists~${\bf U} \in \mZ^{n \times n}$ with~$\det ({\bf U}) = \pm 1$ such that 
\[
\left({\bf c}_1|\ldots |{\bf c}_n \right) = \left({\bf b}_1|\ldots |{\bf b}_n \right) \cdot {\bf U}.
\]
\end{lemma}

Note that~${\bf U}$ is multiplied from the right, as we consider column vectors. 
A matrix~${\bf U} \in \mZ^{n \times n}$ with~$\det ({\bf U}) = \pm 1$ is called \emph{unimodular}. Such matrices are exactly
the integer matrices whose inverses are integer matrices, i.e., the elements of~$\mbox{GL}_n(\mZ)$. Examples of unimodular matrices include permutation matrices, diagonal matrices with entries equal to~$\pm 1$ and upper triangular integer matrices with~$1$'s on the diagonal. In fact, it can be proved that any unimodular matrix can be written as a product of such matrices. 

Lemma~\ref{le:dim} implies
that a 1-dimensional lattice~$L = x \cdot \mZ$ with~$x \in \mR\setminus \{0\}$ has exactly  two bases: $x$ and~$-x$. When~$n \geq 2$, 
Lemma~\ref{le:dim} implies that a given lattice~$L$ admits infinitely many lattice bases: indeed, unimodular matrices include integer upper-triangular matrices with $1$'s on the diagonal, and all of these give different bases.  

As the dimension~$n$ is shared across all bases of a given lattice, it is called a \emph{lattice invariant}. 
In the rest of this subsection, we will introduce a few other such lattice invariants. 

\begin{definition}
\label{def:vol}
Let~$L$ be a lattice with a basis matrix~${\bf B}$. The determinant of~$L$ is defined as
\[
\det(L) = \sqrt{\det( {\bf B}^T \cdot {\bf B})}.
\]
\end{definition}

The determinant of a lattice is sometimes called the volume or co-volume. Note that when~$L$ is full-rank, the definition can 
be simplified to~$\det(L) = |\det ({\bf B})|$. The determinant is indeed a lattice invariant, as formally stated in the 
following lemma. 

\begin{lemma}
Let~${\bf B}$ and~${\bf C}$ be two matrix bases of the same lattice~$L$. Then we have
 \[
 \sqrt{\det( {\bf B}^T \cdot {\bf B})} =  \sqrt{\det( {\bf C}^T \cdot {\bf C})}.
\]
\end{lemma}

The proof follows by considering a unimodular matrix~${\bf U}$ such that ${\bf C} = {\bf B} \cdot {\bf U}$. 
Geometrically, and as illustrated in Figure~\ref{fig:det}, the determinant is the volume of the parallelepiped~${\bf B} \cdot [0,1]^n$. 

The following result focuses on dual lattices. 

\begin{lemma}
\label{le:dual_det}
Let~$L$ be a lattice with a basis matrix~${\bf B}$. Then~$\widehat{{\bf B}} = {\bf B} \cdot ( {\bf B}^T \cdot {\bf B})^{-T}$ 
is a basis of~$\widehat{L}$, and~$\det(\widehat{L}) = 1/\det(L)$.  
\end{lemma}
 
Note that when the lattice is full-rank, the definition of~$\widehat{{\bf B}}$ simplifies to~$\widehat{{\bf B}} =  {\bf B}^{-T}$. 

We now return to our running example of construction~A lattices. 

\begin{lemma}
\label{le:consA_dimdet}
Let~$q\geq 2$ prime, $m>n>0$ and~${\bf A} \in \mZ_q^{m \times n}$. 
We have 
\begin{eqnarray*}
\dim (\Lambda({\bf A})) = m \ \ &\mbox{ and }& \ \ \det(\Lambda({\bf A})) = q^{m - r}, \\
\dim (\Lambda^{\perp}({\bf A})) = m \ \ &\mbox{ and }& \ \ \det(\Lambda^{\perp}({\bf A})) = q^{r},
\end{eqnarray*}
where~$r$ is the rank of~${\bf A}$. Further, if~${\bf A}$ is sampled uniformly, then its 
rank is~$n$ with probability~$\geq 1 - q^{-m+n}$. 
\end{lemma}

\begin{proof}
The dimension equalities follow from the fact that~$q \mZ^m$ is contained in 
both~$\Lambda({\bf A})$ and~$\Lambda^{\perp}({\bf A})$. By Lemmas~\ref{le:constA_dual} and~\ref{le:dual_det},
the determinant equality for~$\Lambda({\bf A})$ and~$\Lambda^{\perp}({\bf A})$ are equivalent. We prove the first one
by explicitly constructing a basis for~$\Lambda({\bf A})$. 

As the rank of~${\bf A}$ is~$r$, there exist rank-$r$ matrices~${\bf A}' \in \mZ_q^{m \times r}$ and~${\bf T} \in \mZ_q^{r \times n}$ such  that~${\bf A} = {\bf A}' \cdot {\bf T} \bmod q$. Up to reordering the rows of~${\bf A}'$, we may assume that its first~$r$ rows form 
an invertible matrix. By left-multiplying~${\bf T}$ (resp.\ right-multiplying~${\bf A}'$) by this matrix (resp.\ its inverse), we can assume that the first~$r$ rows of~${\bf A}'$ form the identity matrix. Let us write~${\bf A}' = ({\bf Id} | {\bf A}''^T)^T$ for~${\bf A}'' \in \mZ_q^{(m-r) \times r}$. By using the definition of~$\Lambda({\bf A})$, we can write:
\[
\Lambda({\bf A}) = {\bf A} \cdot \mZ_q^n + q\cdot \mZ^m = \left[ \begin{array}{c|c|c} 
{\bf Id} & q \cdot {\bf Id} & {\bf 0} \\ \hline
{\bf A}'' & {\bf 0} &  q \cdot {\bf Id}
\end{array}\right] \cdot \mZ^{m+r},
\]
where the first block-row has~$r$ rows and $r+r+(m-r)$ columns and the coefficients of~${\bf A}''$ are viewed as integers. 
Note that this matrix has linearly dependent columns and hence
cannot be a basis of~$\Lambda({\bf A})$. However, the second block-column (i.e., columns $r+1$ to~$2r$) can be seen to be linearly 
dependent with the others. For example, the $(r+1)$-th column is $q$ times the first column to which one has subtracted the  columns of the third block  mutiplied by the first column of~${\bf A}''$. The second block is hence superfluous and we can write:
  \[
\Lambda({\bf A}) =  \left[ \begin{array}{c|c} 
{\bf Id} &  {\bf 0} \\ \hline
{\bf A}'' &   q \cdot {\bf Id}
\end{array}\right] \cdot \mZ^{m}.
\]
As the matrix is lower triangular with non-zero diagonal coefficients, its columns are linearly independent. They hence form a basis of~$\Lambda({\bf A})$, and we can compute the lattice determinant by counting the number of occurrences of~$q$ on the diagonal. 

Now, assume that~${\bf A}$ is sampled uniformly in~$\mZ_q^{m \times n}$. It is of rank~$n$ if and only if the $i$-th column is linearly independent from the previous ones, for each~$i \leq n$. The probability~$p$ that this occurs satisfies:
\[
1- p \leq \sum_{i=1}^n q^{-m+i-1}  = q^{-m} \frac{q^n-1}{q-1} \leq q^{-m+n}.
\]
Re-arranging terms allows to complete the proof.
\end{proof}

We now focus on another lattice invariant, namely the lattice minimum. 

\begin{definition}
Let~$L$ be a lattice. Its minimum~$\lambda_1(L)$ is the (Euclidean) norm of any shortest non-zero vector of~$L$. 
\end{definition}

The existence of a shortest non-zero vector in~$L$ is guaranteed by the discreteness of the lattice (Lemma~\ref{def:lattice2}). This implies that~$\lambda_1(L) >0$ for any non-zero lattice~$L$. 




\begin{comment}



Definition : The first minimum of a lattice L is the (Euclidean) norm of a shortest non-zero vector of L.  lambda_1(L). (its existence is implied by the discreteness of L).

It is always reached at least twice, and may be reached more than that (the smallest known upper bound is exponential in the lattice dimension)

Remark: This is exactly twice the packing radius: r_pack = largest r s.t. balls of radii r_pack and centers the lattice points do not overlap. This is not the covering radius, which is the smallest r s.t. lattice balls of radius r cover the whole span. 

Minkowski thm: For every lattice L of rank n, we have lambda_1 ( L ) leq sqrt(n) det(L)^(1/n).

Theorem (simplified variant of Minkowski Hlawka): Let q prime and m >=n. Let A sampled uniformly in Z_q^(m x n). With probability >= 1-q^{-(m-n)}, it has rank n. When this is the case, Lambda_q(A) has determinant q^{m-n}. 
With probability >= 1 - 2^-m, it satisfies 
            lambda_1 (Lambda_q(A)) >= min (q, sqrt(m)*q^{1-n/m}/12). 

Proof. 
Now, the interesting part. lambda_1< B iff there exists s in Z_q^n \ 0 s.t. 0<||A*s mod q||<B 
                                                                                  or q < B (for s=0 mod q).
For non-zero s, does there exist t in Ball_m(B) non-zero s.t. A*s = t mod q? 
For B<=q, the probability that lambda_1 is < B is: 
Pr_A[exists b with non-zero s and norm <=B]
= Pr_A[exists s non-zero and t in Ball_m(B) : As=t mod q]
                              <= sum_s sum_t Pr_A [ A*s = t mod q ]    // union bound
                                = sum_s sum_t  Prod_i Pr_{a_i} [ <a_i,s> = t_i mod q]  // stat indep
                                = sum_s sum_t  q^{-m}  //  s non-zero, q prime
                                <= q^{n-m} * vol(ball_m(B+sqrt(m/2))   // inclusion of boxes
vol(ball_m(C)) <= (2pi e/m)^m/2 * C^m
=> vol(ball_m(B+sqrt(m/2)) <= (ctt/m)^m/2 * B^m    (using B>= sqrt(m))
Pr_A[lambda1<=B]  <= q^{n-m} * (6 * B/sqrt(m)) ^m 
QED
\end{comment}



\begin{comment}






Lemma/Definition [Construction A lattices]. 
  Let C be a linear code over Z/qZ, for a prime q. Dimension m, rank n.
  I.e., C = A * Z_q ^n, for some matrix A in Z_q^{m x n} of rank n. 
  Then L_C = C + q Z^m = A * Zq^n + Z^m = Lambda_q(A) is a lattice.    
  Similarly, 
  L^perp_C = C^perp + qZ^m  = {x in Z^m: x^T * A = 0 mod q} = Lambda_q^perp(A) 
is also a lattice.       

Proof (for the L_C lattice): 
 Use the canonical form => get a basis.  
QED
\end{comment}




\subsection{Lattice Gaussians}


\subsection{Selected Hard Lattice Problems}

