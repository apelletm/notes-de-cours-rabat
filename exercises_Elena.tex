\documentclass[11pt]{exam}
%%%%%%%%%%%%%%%%%%%%%%%%%%%%%%%%
\noprintanswers % pour enlever les réponses
%\printanswers

\unframedsolutions
\SolutionEmphasis{\small\color{blue}}
\renewcommand{\solutiontitle}{\noindent\textbf{A: }}
%%%%%%%%%%%%%%%%%%%%%%%%%%%%%%%%



\usepackage[margin=0.73in]{geometry}
%\usepackage[top=1in, bottom=1in, left=1in, right=1in]{geometry}

%\usepackage{fullpage}


\usepackage{hyperref}
\usepackage{appendix}
\usepackage{enumerate}


\usepackage{times,graphicx,epsfig,amsmath,latexsym,amssymb,verbatim}%,revsymb}
\usepackage{algorithmicx, enumitem, algpseudocode, algorithm, caption}


%%%%%%%%%%%%%%%%%%%%%
% Handling comments and versions %%%
%%%%%%%%%%%%%%%%%%%%%
\newcommand{\extra}[1]{}

\renewcommand{\comment}[1]{\texttt{[#1]}}


%%%%%%%%%%%%%%%%%%%%%%%%%%%
%% THEOREMS
%%%%%%%%%%%%%%%%%%%%%%%%%%%

\usepackage{amsmath,amssymb,amsfonts}
\usepackage{amsthm}

\newtheorem{theorem}{Theorem}[section]
\newtheorem{axiom}[theorem]{Axiom}
\newtheorem{conclusion}[theorem]{Conclusion}
\newtheorem{condition}[theorem]{Condition}
\newtheorem{conjecture}[theorem]{Conjecture}
\newtheorem{corollary}[theorem]{Corollary}
\newtheorem{criterion}[theorem]{Criterion}
\newtheorem{definition}[theorem]{Definition}
\newtheorem{lemma}[theorem]{Lemma}
\newtheorem{notation}[theorem]{Notation}
\newtheorem{proposition}[theorem]{Proposition}


\theoremstyle{definition}
\newtheorem{problem}{Problem}


\newcommand{\nc}{\newcommand}
\nc{\eps}{\varepsilon}
\nc{\RR}{{{\mathbb R}}}
\nc{\CC}{{{\mathbb C}}}
\nc{\FF}{{{\mathbb F}}}
\nc{\NN}{{{\mathbb N}}}
\nc{\ZZ}{{{\mathbb Z}}}
\nc{\PP}{{{\mathbb P}}}
\nc{\QQ}{{{\mathbb Q}}}
\nc{\UU}{{{\mathbb U}}}
\nc{\OO}{{{\mathbb O}}}
\nc{\EE}{{{\mathbb E}}}
\nc{\lat}{{{\mathcal L}}}
\nc{\GL}{{{\mathrm{GL}}}}

\renewcommand{\vec}{\mathbf}


\pretolerance=1000

%%%%%%%%%%%%%%%%%%%%%%%%%%%%%%%%
%%%%%%%%%%%%%%%%%%%%%%%%%%%%%%%%
%% DOCUMENT STARTS
%%%%%%%%%%%%%%%%%%%%%%%%%%%%%%%%
%%%%%%%%%%%%%%%%%%%%%%%%%%%%%%%%
\usepackage{tikz}
\usetikzlibrary{automata}
\DeclareMathOperator{\Vol}{Vol}

\begin{document}

{\noindent
   \textsc{Lattice-based Cryptanalysis}
   \hfill {\textsc{CIMPA Summer School -- 2023}}\\
  }
  \hrule
% Titre de la feuille
  \begin{center}
    {\Large\textbf{
   \textsc{Tutorial on Enumeration and BKZ algorithms}
    } } 
  \end{center}
  \hrule \vspace{5mm}

\thispagestyle{empty}

\vspace{0.2cm}

%\Large

\section{Enumeration}

\begin{questions}
\question Prove that the size of the enumeration tree in the algorithm described in the class is of order $2^{\mathcal{O}(n^2)}$when given on input an LLL-reduced basis.

Use the facts that for an LLL-reduced bases the following holds: 
\begin{enumerate}
	\item  $\frac{r_{1,1}}{r_{i,i}} < \alpha^{i-1}$, where you can take $\alpha = 2$ (it is the upper bound on $\alpha$);
	\item $r_{1,1} = \| \vec b_1 \| \leq \alpha^{\frac{n-1}{2}} (\det L)^{1/n}$.
	
\end{enumerate}

\end{questions}

\section{BKZ}

Let $B$ be a basis given on input to the BKZ algorithm and let $B_{[i,j]}$ for $i<j$ be the (projected) basis formed by the basis vectors $\vec b_i, \ldots \vec b_j$ projected orthogonally to the first $i-1$ basis vectors. 
\begin{questions}
	
\question What is the memory complexity of the BKZ algorithm described in class?

\question Apply Minkowski theorem to the projected lattice $B_{[i, i+\beta-1]}$ to obtain an upper bound on $	\| \tilde{\vec{b}}_i \|$. Conclude that

\begin{align} \label{ineq:q2}
	\| \tilde{\vec b}_i \|^\beta \leq \beta^{\beta/2} \prod_{j=i}^{i+\beta - 1} \| \tilde{\vec b}_j \|
\end{align}

\question With the obtained upper bounds for all $1 \leq i \leq n-\beta +1$'s, show that
\begin{align}\label{ineq:q3}
\| \tilde{\vec{b}}_1 \|^{\beta-1} \cdot \| \tilde{\vec{b}}_2 \|^{\beta-2} \cdot \ldots \cdot \| \tilde{\vec{b}}_{\beta-1} \| \leq 
\beta^{\frac{\beta(n-\beta+1)}{2}} \| \tilde{\vec{b}}_{n-\beta+2} \|^{\beta-1} \| \tilde{\vec{b}}_{n-\beta+3} \|^{\beta-2} \cdot \ldots \cdot  \|  \tilde{\vec{b}}_n\|.
\end{align}

In order to do that, apply Inequality~(\ref{ineq:q2}) to  $\prod_{i=1}^{n-\beta+1}\|\tilde{\vec b}_i \|^\beta$.

\question Using the fact that not only $B_{[1,\beta]}$ is SVP reduced, but also $B_{[1, i]}$ for $i \leq \beta$ are SVP reduced (think why this is true), conclude that (compare with Inequality~(\ref{ineq:q2})):
\begin{align}\label{ineq:q4}
	\| \tilde{\vec b}_1 \|^i \leq i^{i/2} \prod_{j=1}^{i} \| \tilde{\vec b}_j \| \quad \forall i \leq \beta
\end{align}

\question Multiply Inequalities~(\ref{ineq:q4}) for $1 \leq i \leq \beta -1$ and use Inequality~(\ref{ineq:q3}) to obtain
\begin{align}\label{ineq:q5}
\| \tilde{\vec b}_1 \|^{\frac{\beta(\beta-1)}{2}} \leq \beta^{\frac{\beta(n-1)}{2}} \cdot \| \tilde{\vec{b}}_{n-\beta+2} \|^{\beta-1} \| \tilde{\vec{b}}_{n-\beta+3} \|^{\beta-2} \cdot \ldots \cdot  \|  \tilde{\vec{b}}_n\|
\end{align}
\question 
Assume that there exist a shortest vector $\vec{v}_{\texttt{shortest}}$ whose projection orthogonal to the first $n - 1$ basis vectors is non-zero (otherwise, if all shortest vector project to zero onto the span of $\tilde{\vec b}_{n}$, then we know that all of them live in a lattice of dimension at most $n-1$ and we can remove $\vec b_n$).

This implies that $\lambda_1 = \|\vec{v}_{\texttt{shortest}} \| \geq \| \tilde{\vec b}_i \|$ for $n-\beta+2 \leq i \leq n  $ (think why). Plugging this inequality into the right-hand side of Inequality~(\ref{ineq:q5}) conclude that
\[
	\| \vec b_1 \| \leq \beta^{\frac{n-1}{\beta-1}} \lambda_1.
\]

\end{questions}

\end{document}


