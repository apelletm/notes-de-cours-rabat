\documentclass[11pt]{exam}
%%%%%%%%%%%%%%%%%%%%%%%%%%%%%%%%
%\noprintanswers % pour enlever les réponses
%\printanswers

\unframedsolutions
\SolutionEmphasis{\itshape\small}
\renewcommand{\solutiontitle}{\noindent\textbf{A: }}
%%%%%%%%%%%%%%%%%%%%%%%%%%%%%%%%

\usepackage[T2A]{fontenc}
\usepackage[utf8]{inputenc}
\usepackage[english, russian]{babel}


\usepackage[margin=0.73in]{geometry}
%\usepackage[top=1in, bottom=1in, left=1in, right=1in]{geometry}

%\usepackage{fullpage}


\usepackage{hyperref}
\usepackage{appendix}
\usepackage{enumerate}


\usepackage{times,graphicx,epsfig,amsmath,latexsym,amssymb,verbatim}%,revsymb}
\usepackage{algorithmicx, enumitem, algpseudocode, algorithm, caption}


%%%%%%%%%%%%%%%%%%%%%
% Handling comments and versions %%%
%%%%%%%%%%%%%%%%%%%%%
\newcommand{\extra}[1]{}

\renewcommand{\comment}[1]{\texttt{[#1]}}


%%%%%%%%%%%%%%%%%%%%%%%%%%%
%% THEOREMS
%%%%%%%%%%%%%%%%%%%%%%%%%%%

\usepackage{amsmath,amssymb,amsfonts}
\usepackage{amsthm}

% Landau 
\newcommand{\bigO}{\mathcal{O}}
\newcommand*{\OLandau}{\bigO}
\newcommand*{\WLandau}{\Omega}
\newcommand*{\xOLandau}{\widetilde{\OLandau}}
\newcommand*{\xWLandau}{\widetilde{\WLandau}}
\newcommand*{\TLandau}{\Theta}
\newcommand*{\xTLandau}{\widetilde{\TLandau}}
\newcommand{\smallo}{o} %technically, an omicron
\newcommand{\softO}{\widetilde{\bigO}}
\newcommand{\wLandau}{\omega}
\newcommand{\negl}{\mathrm{negl}} 


\newtheorem{theorem}{Теорема}
\newtheorem{corollary}[theorem]{Следствие}
\newtheorem{lemma}[theorem]{Лемма}
\newtheorem{observation}[theorem]{Observation}
\newtheorem{proposition}[theorem]{Предложение}

\theoremstyle{definition}
\newtheorem{definition}[theorem]{Определение}


\newcommand{\nc}{\newcommand}
\nc{\eps}{\varepsilon}
\nc{\RR}{{{\mathbb R}}}
\nc{\CC}{{{\mathbb C}}}
\nc{\FF}{{{\mathbb F}}}
\nc{\NN}{{{\mathbb N}}}
\nc{\ZZ}{{{\mathbb Z}}}
\nc{\PP}{{{\mathbb P}}}
\nc{\QQ}{{{\mathbb Q}}}
\nc{\UU}{{{\mathbb U}}}
\nc{\OO}{{{\mathbb O}}}
\nc{\EE}{{{\mathbb E}}}

\newcommand{\val}{\operatorname{val}}

\newcommand{\wt}{\ensuremath{\mathit{wt}}}
\newcommand{\Id}{\ensuremath{I}}
\newcommand{\transpose}{\mkern0.7mu^{\mathsf{ t}}}
\newcommand*{\ScProd}[2]{\ensuremath{\langle#1\mathbin{,}#2\rangle}} %Scalar Product
%\newcommand*{\eps}{\ensuremath{\varepsilon}}
\newcommand*{\Sphere}[1]{\ensuremath{\mathsf{S}^{#1}}}

\pretolerance=1000

%%%%%%%%%%%%%%%%%%%%%%%%%%%%%%%%
%%%%%%%%%%%%%%%%%%%%%%%%%%%%%%%%
%% DOCUMENT STARTS
%%%%%%%%%%%%%%%%%%%%%%%%%%%%%%%%
%%%%%%%%%%%%%%%%%%%%%%%%%%%%%%%%
\usepackage{tikz}
\usetikzlibrary{automata}



% Custom colors
\usepackage{color}
\definecolor{deepblue}{rgb}{0,0,0.5}
\definecolor{deepred}{rgb}{0.6,0,0}
\definecolor{deepgreen}{rgb}{0,0.5,0}

\usepackage{pythonhighlight}

\begin{document}
	{\noindent
		\textsc{CIMPA School}
		\hfill {E.Kirshanova, D. Stehl{\' e} //2023\\}
	\hrule
	\begin{center}
		{\Large\textbf{
				\textsc{Practice 1: Getting familiar with FpyLLL} \\[5pt]
		} } 
	\end{center}
	}
	\hrule \vspace{5mm}
	
	\thispagestyle{empty}
	
	\vspace{0.2cm}
	

	
The goal of this lab is to familiarize yourself  with basic routines implemented in the FPyLLL library~\cite{fpylll}. Documentation is available at \url{https://readthedocs.org/projects/fpylll/downloads/pdf/latest/} \\
\section{LLL}

\begin{enumerate}

\item Generating a random $q$-ary matrix.
\begin{python}
	from fpylll import *
	import copy
	A = IntegerMatrix.random(30, "qary", k=15, bits=30)
	print(A)
	dim = A.nrows
	print('dimension:', dim)
\end{python}

\item  GSO object, norm computation, and LLL reduction

\begin{python}
	M = GSO.Mat(A, float_type="d") #float_type \in {'d', 'dd', 'qd','mpfr'}
	print(M.get_r(0,0))
	print(M.d) # lattice dimension
	print(M.B) # input basis
	M.update_gso() # compute all interesting matrices (R-factor, mu, ...)
	print(M.get_r(1,1))
	print(A[0].norm()**2) # first basis norm
	print('before:', [M.get_r(i,i) for i in range(dim)]) # all r_ii
	
	L = LLL.Reduction(M, delta=0.99, eta=0.501) # creating an LLL object
	L() # run LLL
	print('after:', [M.get_r(i,i) for i in range(dim)])
	print(A)
\end{python}

\item Transformation matrix
\begin{python}
	A = IntegerMatrix.random(100, "qary", k=50, bits=30)
	dim = A.nrows
	Ac = copy.deepcopy(A) # copy of A
	U = IntegerMatrix.identity(dim)
	UinvT = IntegerMatrix.identity(dim)
	
	M_transf = GSO.Mat(Ac, float_type="qd", U = U, UinvT = UinvT) #U --  transformation matrix mapping A to LLL-reduced Ac
	print(M_transf.inverse_transform_enabled) # LLL computes U
	
	L = LLL.Reduction(M_transf, delta=0.99, eta=0.501)
	L()
	print(Ac[1])
	print((U*A)[1])
\end{python}

\end{enumerate}

\section{Enumeration}

\begin{python}
	A = IntegerMatrix.random(45, "qary", k=25, bits=30)
	M = GSO.Mat(A)
	L = LLL.Reduction(M, delta=0.99, eta=0.501)
	L()
	M.update_gso()
	
	enum = Enumeration(M, strategy=EvaluatorStrategy.BEST_N_SOLUTIONS, sub_solutions=True) #initiate the object Enumeration
	res = enum.enumerate(0, 45, 0.9*M.get_r(0, 0), 0) # launch enumeration
	print(([int(res[0][1][i]) for i in range(len(res[0][1]))]))
	#for a,b in enum.sub_solutions:
	#    print(a, b)
	
	print(enum.get_nodes()) # number of visited nodes
\end{python}

\section{BKZ}

Main source of the code: \url{https://www.maths.ox.ac.uk/system/files/attachments/lab-02.pdf}

\begin{enumerate}
	\item Part 1: Launching BKZ 
	
	\begin{python}
		from fpylll import *
		from fpylll.algorithms.bkz import BKZReduction
		from math import log
		
		from fpylll.algorithms.bkz2 import BKZReduction as BKZ2
		import matplotlib.pyplot as plt

		A = IntegerMatrix.random(60, "qary", k=30, bits=30)
		
		bkz = BKZReduction(A) #instantiation class BKZReduction
		print('before BKZ:', log(A[0].norm(),2))
		bkz(BKZ.EasyParam(20, max_loops=8)) #block size  15 ; maxloops = number of bkz tours
		print('after BKZ:', log(A[0].norm(),2))
		bkz(BKZ.EasyParam(15, max_loops=8), tracer=True)
		print(bkz.trace.get(("tour", 1)).report()) #for tracer=True
		
		
		
		k = 40
		flags = BKZ.AUTO_ABORT
		#print('flags:', flags, BKZ.AUTO_ABORT, BKZ.MAX_LOOPS,BKZ.VERBOSE)
		par = BKZ.Param(k, strategies=BKZ.DEFAULT_STRATEGY, max_loops=8, flags=flags) # instantiation class of BKZ parameters
		#bkz = BKZ2(A) #one way to instantiate BKZ # or
		#bkz = BKZ2(GSO.Mat(A)) # another way to instantiate BKZ
		bkz = BKZ2(LLL.Reduction(GSO.Mat(A))) #third way to instantiate BKZ
		_ = bkz(par) # run BKZ with flags
		print(log(A[0].norm(),2))
	\end{python}
	\item Part II: tracing how $r_{i,i}$'s change during BKZ
	\begin{python}
	A = IntegerMatrix.random(80, "qary", k=40, bits=45)
	k = 42
	tours = 3
	
	LLL.reduction(A)
	M = GSO.Mat(A)
	M.update_gso()
	d = A.nrows
	
	colours = ["#4D4D4D", "#5DA5DA", "#FAA43A", "#60BD68", "#F17CB0", "#B2912F", "#B276B2", "#DECF3F", "#F15854"]
	norms = [[log(M.get_r(i,i)) for i in range(d)]]
	plt.plot(norms[0],label="lll", color=colours[0])
	
	par = BKZ.Param(block_size=k,strategies=BKZ.DEFAULT_STRATEGY)
	bkz = BKZ2(M)
	
	for i in range(tours):
	bkz.tour(par)
	norms += [[log(M.get_r(j,j)) for j in range(d)]]
	plt.plot(norms[i+1],label="tour %d"%i, color=colours[i+1])
	
	legend = plt.legend(loc='upper center')
	plt.show()
	\end{python}
\end{enumerate}

\section{G6K}

\begin{enumerate}
	\item Launch SVP with sieving
	\begin{python}
		from fpylll import *
		from g6k import *
		from g6k.utils.stats import SieveTreeTracer
		from g6k.algorithms.bkz import pump_n_jump_bkz_tour
		from fpylll.util import gaussian_heuristic
		from math import sqrt

		n = 60
		A = IntegerMatrix.random(n, "qary", k=30, bits=30)
		g6k = Siever(A, seed=0x1337) # g6k has its own seed
		
		g6k.initialize_local(0, 0, n) # sieve the entire basis
		g6k(alg="bgj1") #alg \in {"nv", "gauss", "hk3"}
		v = g6k.best_lifts()[0][2]
		print('best lifts', g6k.best_lifts())
		w = A.multiply_left(v)  # w = v*A
		print('norms:', sum(w_**2 for w_ in w), A[0].norm()**2)
	\end{python}
\item Simulate BKZ with sieving
\begin{python}
n = 60
A = IntegerMatrix.random(n, "qary", k=30, bits=30)

g6k = Siever(A, seed=0x1337) # g6k has its own seed
g6k.initialize_local(0, 0, n)
sp = g6k.params.new()
sp.threads = 1
#g6k(alg="bgj1")

tracer = SieveTreeTracer(g6k, root_label=("bgj1"), start_clocks=True)
for b in (20, 30, 40, 50, 60): # progressive BKZ
	pump_n_jump_bkz_tour(g6k, tracer, b, pump_params={"down_sieve": True})
print(b, A[0].norm()**2)
\end{python}
\end{enumerate}

\begin{thebibliography}{9}
	\bibitem{fpylll} The {FPLLL} development team.
	\textit{{fpylll}, a {Python} wrapper for the {fplll} lattice reduction library, {Version}: 0.5.9}, 2023. Available at \url{https://github.com/fplll/fpylll}
	
\end{thebibliography}


\end{document}